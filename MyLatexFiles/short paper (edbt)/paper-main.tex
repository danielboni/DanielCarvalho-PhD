\documentclass{sig-alternate}
\begin{document}
%
% --- Author Metadata here ---
%\conferenceinfo{WOODSTOCK}{'97 El Paso, Texas USA}
%\CopyrightYear{2007} % Allows default copyright year (20XX) to be over-ridden - IF NEED BE.
%\crdata{0-12345-67-8/90/01}  % Allows default copyright data (0-89791-88-6/97/05) to be over-ridden - IF NEED BE.
% --- End of Author Metadata ---
 
\title{Query Rewriting Algorithm for Data Integration Quality on Multi-Could}
%\titlenote{(Produces the permission block, andcopyright information). For use with
%SIG-ALTERNATE.CLS. Supported by ACM.}}
%\subtitle{[Extended Abstract]
%\titlenote{A full version of this paper is available as
%\textit{Author's Guide to Preparing ACM SIG Proceedings Using
%\LaTeX$2_\epsilon$\ and BibTeX} at
%\texttt{www.acm.org/eaddress.htm}}}

\numberofauthors{5} 
 
\author{ 
% 1st. author
\alignauthor
Daniel A. S. Carvalho\\
       \affaddr{Univ. Jean Moulin Lyon 3}\\
       \affaddr{Lyon, France}\\
       \email{danielboni@gmail.com}
% 2nd. author
\alignauthor
Placido A. Souza Neto\\
       \affaddr{Instituto Federal do Rio Grande do Norte - IFRN}\\
       \affaddr{Natal, Brazil}\\
       \email{placido.neto@ifrn.edu.br}
% 3rd. author
\alignauthor 
Chirine G. Guegan\\
       \affaddr{Univ. Jean Moulin Lyon 3}\\
       \affaddr{Lyon, France}\\
       \email{chirine.ghedira-guegan@univ-lyon3.fr}
\and  % use '\and' if you need 'another row' of author names
% 4th. author
\alignauthor 
Nadia Benani\\
       \affaddr{CNRS-INSA}\\
       \affaddr{Lyon, France}\\
       \email{nadia.bennani@insa-lyon.fr}
% 5th. author
\alignauthor 
Genoveva Vargas-Solar\\
       \affaddr{CNRS-LIG-LAFMIA}\\
       \affaddr{Grenoble, France}\\
       \email{genoveva.vargas@imag.fr}
}
% There's nothing stopping you putting the seventh, eighth, etc.
% author on the opening page (as the 'third row') but we ask,
% for aesthetic reasons that you place these 'additional authors'
% in the \additional authors block, viz.
%\additionalauthors{Additional authors: John Smith (The Th{\o}rv{\"a}ld Group,
%email: {\texttt{jsmith@affiliation.org}}) and Julius P.~Kumquat
%(The Kumquat Consortium, email: {\texttt{jpkumquat@consortium.net}}).}
%\date{30 July 1999}
% Just remember to make sure that the TOTAL number of authors
% is the number that will appear on the first page PLUS the
% number that will appear in the \additionalauthors section.

\maketitle
\begin{abstract}
In this paper we describe \textit{Rhone}, a query rewriting algorithm for data
integration using SLA measures. We present a running scenarion for validate the
algorithm in a multi-cloud environment.
\end{abstract}

% A category with the (minimum) three required fields
\category{H.4}{Information Systems Applications}{Miscellaneous}
%A category including the fourth, optional field follows...
\category{D.2.8}{Software Engineering}{Metrics}[complexity measures, performance measures]

\terms{Theory}

\keywords{ACM proceedings, \LaTeX, text tagging}

\section{Introduction}
Data integration is a well-know problem in the database domain~\cite{Halevy:2001}. 
Recently, with the emergency of cloud environments, this problem has also been threated in the service-oriented domains~\cite{Barhamgi2010,Benouaret2011,ba2014}.
Generally, these kind of applications deals with query rewriting problems.
\cite{Barhamgi2010} proposed a query rewriting approach which processes queries on data provider services. \cite{Benouaret2011} introduced a service composition framework to answer preference queries. In that approach, two algorithms based on~\cite{Barhamgi2010} are presented to rank the best rewritings based on previously computed scores.
%: one (i) to produce all possible rewritings before computing their scores and other (ii)
%which uses a quality metric that combines diversity and accuracy to,
%incrementally, rank services and to build the best rewritings.

\cite{ba2014} presented an algorithm which produces and order rewritings according to user preferences.
Our work differs from the others in two aspects: (i) the user can express his
quality preferences on his queries; and (ii) we consider the service's quality
aspects defined on SLA to service selection and to produce rewritings.    

In this context, the aim of our work is to present the early stages of our
ongoing work on developing the \textit{Rhone} service-based query rewriting
algorithm.

The remainder of this paper is organized as follows. Section~\ref{sec:rhone}
describes the proposed algorithm. Section~\ref{sec:implementationandresults} describes
the running scenario and also the implementation process.
Finally, section~\ref{sec:conclusions} presents the conclusions.


\section{Service-based Query Rewriting Algorithm}
\label{sec:rhone} 
This section describes \textit{Rhone} the service-based query rewriting algorithm that we propose. Given
a set of \textit{abstract services}, a set of \textit{concrete services}, a
\textit{user query} and a set of user \textit{quality preferences}, derive a set
of service compositions that answer the query and that fulfill the quality preferences
 regarding the context of data service deployment.

The  input for  Rhone  is: (1) a query; (2) a list of concrete services defined in the following lines.

\noindent \textbf{Definition 1 (Query):} 
%A query $Q$ has the form:
A query $Q$ is defined as a set of \textit{abstract services} ($A$), a set of \textit{constraints} ($C$), and a set of \textit{user preferences}($P$) in accordance with the grammar: 

\begin{center}
$Q (\overline{I}, \overline{O}) := A_{1}(\overline{I}, \overline{O}), A_{2}(\overline{I}, \overline{O}), ..,  A_{n}(\overline{I}, \overline{O}),C_{1},C_{2}, .., C_{m}[P_{1},P_{2}, .., P_{k}]$
\end{center}  

The left side of the definition is called the \textit{head} of the query; and the right side is called the \textit{body}. 
$\overline{I}$ and $\overline{O}$ are a set of \textit{input} and \textit{output} parameters, respectively.
Input parameters in both sides of the definition are called \textit{head variables}.
In contrast, input parameters only in the query body are called \textit{local variables}.
The preferences (for short \textit{measures}) are classified as: \textit{single measures} (static measures) and \textit{composed measures} (aggregations of \textit{single measures}).
%
%The result if this step is a list of \textit{candidate concrete services} which
% may be used in the rewriting process.

\noindent \textbf{Definition 2 (Concrete service)}  ($S$):
\begin{center}
$S (\overline{I}, \overline{O}) := A_{1}(\overline{I}, \overline{O}), A_{2}(\overline{I}, \overline{O}), ..,  A_{n}(\overline{I}, \overline{O})[P_{1},P_{2}, .., P_{k}]$
\end{center}  

A concrete service ($S$) is defined as a set of abstract services ($A_{i}$), and by their quality constraints $P_{j}$. 
These quality constraints associated to the service represent the Service Level Agreement (SLA) exported by the concrete service.

The algorithm consists in four steps: (i) select candidate concrete services; (ii)
create mappings from concrete services to the query (called \textit{concrete
service description (CSD)}); (iii) combine the list of CSDs; and finally (iv)
produce rewritings.

\begin{algorithm}
\small
\caption{ - RHONE}
\label{algo-rhone}
\begin{algorithmic}[1]
\STATE \textbf{function} $\mathit{rhone} (Q, \bigS)$
 \STATE  $\bigLS \leftarrow \mathit{SelectCandidateServices}(Q, \bigS)$ \label{rhone:buildPCD}
 \STATE  $\bigLCSD \leftarrow CreateCSDs(Q, \bigLS)$
 \STATE  $I \leftarrow CombineCSDs(\bigLCSD)$
 \STATE $R\leftarrow \emptyset$
% \STATE ~\!\tqI{\agg{Q}} 
    \STATE $p \leftarrow I.next()$
    \WHILE {$p\ \neq\ \emptyset$ \AND ~\!\tqI{\agg{Q}}} 
      \IF {\textit{isRewriting}$(Q, p)$}
  \STATE $R\leftarrow R\,\cup \mathit{Rewriting}(p)$
  \STATE ~\!\tqS{\agg{Q}}
   \ENDIF
      \STATE $p \leftarrow I.\mathit{Next}()$
 \ENDWHILE
    \STATE \textbf{return} $R$
\STATE \textbf{end function}
\end{algorithmic}
\end{algorithm}

\noindent \textbf{Select candidate concrete services:} This step
 looks for concrete services that can be matched with the query (line 2). In
 this sense, there are three matching problems: 
 (i) \textit{abstract service matching}, an abstract service $A$ can be
 matched with an abstract service $B$ only if (\textit{a}) they have the same
 name, and (\textit{b}) they have a compatible number of variables;
 (ii) \textit{measure matching}, all \textit{single measures} in the query must
 exist in the concrete service, and all of them can not violate the measures in
 the query ; and 
 (iii) \textit{concrete service matching}, a concrete service can
 be matched with the query if all its abstract services satisfy the \textit{abstract service
 matching} problem and all the \textit{single measures} satisfy the \textit{measures matching} problem.
% Considering \textit{abstract service matching}, an abstract service $A$ can be
% matched with an abstract service $B$ only if (\textit{a}) they have the same
% name.  In the \textit{measure matching}, all single measures in the query must
% exist in the concrete service, and all of them can not violate the measures in
% the query. For the \textit{concrete service matching}, a concrete service can
% be matched with the query if all its abstract services can be matched with the
% abstract service in the query (satisfying the \textit{abstract service
% matching} problem) and all the single measures in the query can be matched with
% the concrete service measures (satisfying the \textit{measures matching} problem).
The result of this step is a list of \textit{candidate concrete services} which
may be used in the rewriting process.


\noindent \textbf{Creating concrete service descriptions:} This step tries to create \textit{concrete services description} (CSD) to be
 used in the rewriting process (line 3). A CSD maps abstract services and
 variables of a concrete service onto abstract services and variables of the
 query. A CSD is created 
% according to the following variable mapping rules:  (i)
% \textit{head variables}, (ii) \textit{local-head variables} and  (iii)
% \textit{local-local variables}.
according to variable mapping rules mainly based on 2 criterias: the type and the dependency (variables used as inputs on other abstract services). 
%\textit{Head} and \textit{local} variables in concrete services can be mapped
% to \textit{head} or \textit{local variables} in the query if they are of the same type.
%\textit{Local} variables in concrete services can be mapped to \textit{head} variables in the query. 
%\textit{Local} variables in concrete services can be mapped to \textit{local}
% variables in the query if: (\textit{a}) they are of the same type; and (\textit{b}) the concrete service covers all abstract services in the query that depend on this variable. The relation ``depends''  means that this \textit{local} variable is used as input in another abstract service. 
The  result of this step is a list of CSDs.
  
%    \textit{Head variables} in concrete services can be mapped to
%  \textit{head} or \textit{local variables} in the query if they are of the
%  same type. \textit{Local-head variables} in concrete services can be mapped to
%  \textit{head variables} in the query if they are of the same type.
%  \textit{Local-local variables} in concrete services can be mapped to
%  \textit{local-local variables} in the query if: (\textit{a}) they are of the
%  same type; and (\textit{b}) the concrete service covers all abstract services in
%  the query that depend on this variable. The relation "depends"  means that this local variable is used as input in another abstract service. The  result of this step is a list of CSDs.

%Regarding the example CSDs are created to ?????.
 
\noindent \textbf{Combining CSDs.} Given all produced CSDs  (line
4), they are combined among each other to generate  a list of lists of CSDs, each element representing a possible composition.

\noindent \textbf{Producing rewritings.} The final step (lines 5-13) identifies which lists of CSDs are a valid
rewritings of the user query given the list of lists of CSDs.
A combination of CSDs is a valid rewriting if: (i) they cover all abstract services in the query; and 
(ii) there is mapping to all head variables in the query (implemented by the function \textit{isRewriting}$(Q, p)$ - line 8).
The originality of our algorithm concerns the aggregation function (\agg{Q}).
It is responsible to check and increment \textit{composed measures} (if present in the query). 
This means for each element in the CSD list the value of \textit{composed measure} is incremented (line 10), and rewritings are produced while the values of these measures are respected (line 7). 
The result of this step is the list of valid 
 rewritings of the query (line 14).%, that is those the provide expected data
 % and respect quality preferences.

%\noindent \textbf{Producing rewritings.} In the final step, given the list of
% lists of CSDs, the algorithm identifies which lists of CSDs are a valid
% rewritings of the user query (lines 5-13).
%A combination of CSDs is a valid rewriting iff: (i) the number of
% \textit{abstract services} in the query is equal to the result of adding the
% number of \textit{abstract services} of each CSD; (ii) there is no duplicated abstract
% service; (iii) there is mapping to all head variables in the
% query; and (iv) if the query contains a \textit{composed measure}, that corresponds to the preferences associated to the query. Every element in the CSD list has its corresponding \textit{composed measure} (represented as the called function \textit{isRewriting}$(Q, p)$ - line 8). The result of this step is the list of valid 
% rewritings of the query (line 14), that is those the provide expected data and respect quality preferences.

% Regarding our example the query has a preference which is associated to the
% rewritings (composed measure). Its value is updated while rewriting the query. In that case, \emph{total cost} is updated by aggregating the value of \emph{price per call} of each service. The rewritings produced are below. Note that more rewritings can be produced if the \textit{composed measure} did not exists. The rewrintgs are listed in the lexicographical order considering the concrete services.
%\begin{tiny}
%\begin{verbatim}
%Q(disease?, info!, dna!) := S1(disease?,p!) S7(p?,info!) S4(p?,dna!)
%Q(disease?, info!, dna!) := S3(disease?,p!, _) S7(p?,info!) S4(p?,dna!)
%Q(disease?, info!, dna!) := S1(disease?,p!) S8(p?,info!) S4(p?,dna!)
%\end{verbatim}
%\end{tiny}

\section{Implementation and Results}
\label{sec:implementationandresults}  
Let us suppose the following medical scenario to illustrate our service-based query rewriting algorithm. 
Users can access \textit{abstract services} (basic service capabilities) to retrieve: (i) patients infected by a given disease (\textit{DiseasePatients(d?,p!)});
(ii) patient dna information \textit{PatientDNA(p?, dna!)}; and (iii) patient personal information (\textit{PatientInformation(p?, info!)}).
%To perform these function consider the \textit{abstract services}: (i)
%, (ii)  and (iii)
%.
The decorations ? and ! are used to specify input and output parameters, respectively. 

%Users can retrieve information about patients, diseases, dna information and others.
%To perform these function consider the \textit{abstract services}: (i)
%\textit{DiseasePatients(d?,p!)}, (ii) \textit{PatientDNA(p?,dna!)} and (iii)
%\textit{PatientInformation(p?,info!)}.
%The decorations ? and ! are used to specify input and output parameters, respectively. 

% 
% \begin{table}[h!]
% \center
% \begin{tabular}{|p{3.7cm}|p{4cm}|}
% \hline
% \begin{small} \textbf{\textit{Abstract Service}} \end{small} & \begin{small}\textbf{\textit{Description}} \end{small}\\ 
% \hline 
% \begin{small} \textit{DiseasePatients(d?,p!)} \end{small} & \begin{small} Given a disease \textit{d}, a list of patients \textit{p} infected by it is retrieved. \end{small}\\ 
% \hline 
% \begin{small} \textit{PatientDNA(p?,dna!)} \end{small} & \begin{small} Given a patient \textit{p}, his DNA information \textit{dna} is retrieved. \end{small}\\ 
% \hline 
% \begin{small} \textit{PatientInformation(p?,info!)} \end{small} & \begin{small} Given a patient \textit{p}, his personal information \textit{info} is retrieved. \end{small}\\ 
% \hline 
% \end{tabular} \caption{List of \textit{abstract services}}
% \end{table}\label{table:abstractservices}

%In our scenario, a \textit{query} expresses an abstract composition that
% describes the requirements of a user.
%\textit{Queries} and \textit{concrete services} are defined in terms of
% \textit{abstract services}.
%They can be associated to a single \textit{abstract service} or to a
% composition of them.

Let us consider the query: \textit{a user wants to retrieve personal and DNA information of patients who were infected by a disease `K' using services that have availability higher than 98\%, price per call less than 0.2 dollars, and total cost less than 1 dollar.} 
The query below corresponds to the example (see Definition 1).
% A query $Q$ tagged with user preferences is defined in accordance with the grammar:
% \begin{center}
% $Q (\overline{I}, \overline{O}) := A_{1}(\overline{I}, \overline{O}), A_{2}(\overline{I}, \overline{O}), ..,  A_{n}(\overline{I}, \overline{O})[P_{1},P_{2}, .., P_{k}]$
% \end{center}
% where the left side is the \textit{head} of the query; and the right side is the \textit{body}. 
% $\overline{I}$ and $\overline{O}$ are a set of \textit{input} and \textit{output} parameters, respectively.
% Input parameters present in both sides of the definition are called \textit{head variables}.
% In contrast, input parameters only in the body are called \textit{local variables}.
% $A_{1}, A_{2}, .., A_{n}$ are \textit{abstract services}.
% $P_{1}, P_{2}, .., P_{k}$ are user preferences (over the services). Preferences are in the form $x \otimes constant$ such that $\otimes \in\lbrace \geq, \leq, =, \neq, <, >\rbrace$.

%The query which express the example following our grammar is below.
%The decorations ? and ! are used to specify input and output parameters, respectively. 
\begin{small}
\begin{center}
$Q (d?, dna!) := DiseasePatients(d?, p!), PatientDNA(p?, dna!),$ \\
$[availability > 99\%, \ price \ per \ call < 0.2\$, \ total \ cost < 1\$]$
\end{center} 
\end{small}

%We highlight that in the query there are two types of preferences (let's refer
%to them as \textit{measures}): \textit{single measures} (availability and price
%per call) and \textit{composed measures} (total cost).
%The \textit{single measures} are the simplest type. It is a static measure
% which is has a name associated with an operation and a value. The\textit{ composed measure} is dynamically computed measure. It is defined as aggregations of \textit{single measures}.

%\textit{Concrete services} are defined follwing the same grammar as the
%\textit{query}. The only difference is that concrete services do not have
%\textit{composed measures}. 
  
\begin{figure}[!htb]
\centering
\includegraphics[scale=0.35]{analysis.png}
\label{fig:fig01}
\caption{Query rewriting evaluation.}
\end{figure}

We use 7 concrete services to run our approach. In this example all the queries
have 6 \textit{abstract services} and 2 \textit{single measures}. The number of local variables (dependencies) and CSDs
is being modified to see how the algorithm works under these conditions.   
 
By now, the analysis identified that the factor that influenciates the Rhone
performance is the number of CSDs versus the number of abstract services in the
query since they increase the number of possible combinations of CSDs.  
We proceeded two types of analysis for the query $Q$ (\textit{Test 1} and
\textit{Test 2} - Figure 1). The first set of tests doesn't consider quality measures,
while the second consider the measures in the rewriting process. The number of
concrete services used in the analysis is from 2 until 35 services. Our
preliminary results show that our algorithm, considering the quality measures,
presents a better result for performance and the total number of rewritings.     

%To illustrate our approach in the next section, consider the \textit{concrete
% services} in table~\ref{table:concreteservices}.

% \begin{table}[h!]
% \center
% \begin{tabular}{p{7cm}}
% \hline
% \begin{small} $S1 (a?, b!) := DiseaseInfectedPatients(a?, b!)$ \end{small}\\ 
% \begin{small} $[availability > 99\%, \ price \ per \ call = 0.2\$]$ \end{small}\\ 
% \hline 
% \begin{small} $S2 (a?, b!) := DiseaseInfectedPatients(a?, b!)$ \end{small}\\
% \begin{small} $[availability > 99\%, \ price \ per \ call = 0.1\$]$ \end{small}\\ 
% \hline 
% \begin{small} $S3 (a?, b!, c!) := DiseaseInfectedPatients(a?, b!, c!)$ \end{small}\\
% \begin{small} $[availability > 98\%, \ price \ per \ call = 0.1\$]$ \end{small}\\ 
% \hline 
% \begin{small} $S4 (a?, b!) := PatientDNA(a?, b!)$ \end{small}\\
% \begin{small} $[availability > 99.5\%, \ price \ per \ call = 0.1\$]$ \end{small}\\ 
% \hline
% \begin{small} $S5 (a?, b!) := PatientDNA(a?, b!)$ \end{small}\\
% \begin{small} $[availability > 99.7\%, \ price \ per \ call = 0.1\$]$ \end{small}\\ 
% \hline
% \begin{small} $S6 (a?, b!) := PatientInformation(a?, b!)$ \end{small} \\
% \begin{small} $[availability > 99.7\%, \ price \ per \ call = 0.1\$]$ \end{small}\\ 
% \hline
% \begin{small} $S7 (a?, b!) := PatientDNA(a?, c!),PatientInformation(c?, b!)$ \end{small}\\
% \begin{small} $[availability > 99.7\%, \ price \ per \ call = 0.1\$]$ \end{small}\\ 
% \hline
% \end{tabular} \caption{Available \textit{concrete services}}
% \end{table}\label{table:concreteservices}

\section{Implementation and Results}
\label{sec:implementationandresults}  
The algorithm is implemented in Java. 
We are currently performing experiments in order to evaluate the performance of the Rhone. 
%To perform this, a set of testcases have been produced varying the number of concrete services, CSDs, abstract services and measures. 
The figure~\ref{figure:chart1} is an example of the charts we have created.
In this example all the queries have 6 \textit{abstract services} and 2 \textit{single measures}. The number of local variables (dependencies) and CSDs is being modified to see how the algorithm works under these conditions.
\begin{center}
\begin{figure}[h!]
\includegraphics[scale=0.3]{chart1.PNG} 
\end{figure}\label{figure:chart1}
\end{center}

By now, the analysis identified that the factor that influenciates the Rhone
performance is the number of CSDs vs. the number of abstract services in the
query since they increase the number of possible combinations of CSDs.   


\section{Conclusions}
...

%ACKNOWLEDGMENTS are optional
\section{Acknowledgments}
optional..

%
% The following two commands are all you need in the
% initial runs of your .tex file to
% produce the bibliography for the citations in your paper.
\bibliographystyle{abbrv}
\bibliography{sigproc}  % sigproc.bib is the name of the Bibliography in this case

\end{document}
