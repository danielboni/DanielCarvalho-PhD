\documentclass[12pt,a4paper,oneside]{article}
\usepackage[utf8]{inputenc}
\usepackage[english]{babel}
\usepackage{amsmath}
\usepackage{amsfonts}
\usepackage{amssymb}
\usepackage{graphicx}
\usepackage[left=2cm,right=2cm,top=2cm,bottom=2cm]{geometry}
\begin{document}

\section{Slides Description}
Problem addressed in our algorithm is: given a set of abstract services, a set of concrete services, a user query and a set of user quality preferences, derive a set of service compositions that answer the query and fulfill the quality preferences.

Our related work relies on two different domains: query rewriting algorithms using views (database domain)~\cite{ref} and query algorithm in the service composition domain~\cite{ref}.

We assume that: 
\begin{itemize}
\item The query expresses an abstract composition that describes the requirements of a user. It is expressed with respect to a catalogue of abstract services
\item A concrete service is defined in terms of an abstract composition. It can be associated to a single abstract service or to a composition of abstract services
\item Concrete services are tagged quality measures. Not all services are tagged with the same measures. Every measure is defined in a catalogue
\item Each measure is written in the form: \textit{constant} (measure name) \textit{operation} ($<$, $>$, =, $\leq$, $\geq$) \textit{value} (the static value associated to the measure)
\item There are two types of measures: single and composite measures. The single measures is the simplest type. It is a static measure which is has a name associated with an operation and a value. The composite measure is dynamically computed measure. It is defined as aggregations of single measures. All measures we are using are in a catalogue. Example:
Single measure: availability, price per call, price per request, response time, location, provenance, etc.
Composite measure: total price or cost which are computed by adding price per call and price per request values of all services included in the service composition. Total response time is computed by adding the response time of all services included in the service composition.
\end{itemize}

\begin{verbatim}
<measures>
	<singlemeasure name=‘‘price per request’’ type=‘‘double’’/>
	<singlemeasure name=‘‘price per call’’ type=‘‘double’’/>
	<compositemeasure name=‘‘total price’’ type=‘‘double’’ action=‘‘sum’’>
		<singlemeasure name=‘‘price per request’’ type=‘‘double’’/>
		<singlemeasure name=‘‘price per call’’ type=‘‘double’’/>
	</compositemeasure>
</measures>
\end{verbatim}

The parameter "action" in the composite measure specifies how the single measures should be aggregated (adding, average, product, etc).

The table~\ref{example1} contains the abstract services that will be used in our example.

\begin{table}[h!]
\center
\begin{tabular}{|p{7cm}|p{7cm}|}
\hline 
\textbf{\textit{Abstract Service}} & \textbf{\textit{Description}} \\ 
\hline 
\textit{DiseaseInfectedPatients(d?,p!)} & Given a disease \textit{d}, a list of patients \textit{p} infected by it is retrieved. \\ 
\hline 
\textit{DiseaseInfectedPatients(d?,p!,op!)} & Given a disease \textit{d}, a list of patients \textit{p} infected by it is retrieved, and \textit{op} is an optional boolean output indicating if the operation proceeded well or not. \\ 
\hline 
\textit{PatientDNA(p?,dna!)} & Given a patient \textit{p}, his DNA information \textit{dna} is retrieved. \\ 
\hline 
\textit{PatientPersonalInformation(p?,info!)} & Given a patient \textit{p}, his personal information \textit{info} is retrieved. \\ 
\hline 
\end{tabular} \caption{Abstract services description}
\end{table}\label{example1}

\noindent Let us suppose the following query. 
\textit{The user wants to retrieve patient’s personal and DNA information of patients who were infected by a disease «K»	 using services that have availability higher than 98\%, price per call less than 0.2 dollars, and total cost less then 1 dollar.}

The query which express the example in terms of abstract services is specified below.
The decorations ? and ! are used to specify input and output parameters, respectively. 

\begin{center}
$Q (d?, patientinfo!, dna!) := DiseaseInfectedPatient (d?, p!), PatientPersonalInformation (p?, info!), PatientDNA (p?, dna!)$ \\ $\lbrace p="K" \rbrace [availability > 98, \ price \ per \ call < 0.2, \ total \ cost < 1]$ \\
\end{center}  

The query expressed in terms of abstract services including its constraints and preferences. In the current implementation braces ($\lbrace\rbrace$) and brackets ($[]$) are used to distinguish constraints from preferences.

The first step of the algorithm looks for concrete services that can be matched with the query. We have three matching problems associated to this step: abstract service matching, measure matching and concrete service matching.
\begin{itemize}
\item \textit{abstract service matching}: a abstract service $A$ can be matched with a abstract service $B$ only if: (i) they have the same name. In this case we are assuming they perform the same function; and (ii) the number and type of variables should be compatible. This means that the number of input and output variables of $A$ must be equal or higher than the number of input and output variables of $B$. Consider the example below.
\begin{verbatim}
1) A1(x?,c!)
	
2) A1(a?,b!,c!)
\end{verbatim}
In this example, 2 can be matched to 1, because the number of input and output variables of 2 is higher than the number of input and output variables of 1.
However 1 can not be matched to 2, because the number of input and output variables of 1 is less than the number of input and output variables of 2.
\item \textit{measures matching}: all single measures in the query must exist in the concrete service, and all of them can not violate the measures in the query.
\item concrete service matching: one concrete service can be matched with the query if all its abstract services can be matched with the abstract service in the query (satisfying the \textit{abstract service matching} problem) and all the single measures in the query can be matched with the concrete service measures (satisfying the \textit{measures matching} problem). Consider the example below.
\begin{tiny}
\begin{verbatim}
Q(x?,y!) = A1(x?,c!), A2(c?,y!)[availability > 98, price per call < 0.2, total price < 1]
	
S1(a?,b!) = A1(a?,b!) [availability > 98] (It can not be matched. Price per call is missing.)

S2(a?,b!) = A1(a?,b!) [availability > 98, price per call = 0.2] (It can not be matched. Price per call is violating
the query preference.)

S3(a?,b!) = A1(a?,c!), A3(c?,b!) [availability > 98, price per call = 0.1] (It can not be matched. A3 is not a abstract
service in the query.)

S4(a?,b!) = A1(a?,b!) [availability > 98, price per call = 0.1]

S5(a?,b!) = A1(a?,c!), A2(c?,b!) [availability > 99, price per call = 0.1]

S5(a?,b!) = A1(a?,c!), A2(c?,b!) [availability > 99, price per call = 0.1, location = “close”](Even have more single measures than the query, 
it can be matched)
\end{verbatim}
\end{tiny}
\end{itemize}

The principle of the solution for the first step of the algorithm is: given (i) a query and a set of user quality preferences; and (ii) a set of concrete services tagged with quality measures, looks for concrete services which respects the matching rules. The concrete service that matches the rules is a candidate concrete service. The result is a list of candidate concrete services which may be used in the rewriting process.

Related work on web service selection and composition considering QoS aspects~\cite{}. 

To illustrate the selection of candidate services, consider the query in the example presented before and the following concrete services.
\begin{tiny}
\begin{verbatim}
S1(a?,b!) := DiseaseInfectedPatients(a?,b!)[availability > 99, price per call = 0.1]
S2(a?,b!) := DiseaseInfectedPatients(a?,b!)[availability > 98, price per call = 0.2]
S3(a?,b!,c!) := DiseaseInfectedPatients(a?,b!)[availability > 98, price per call = 0.1]
S4(a?,b!) := PatientDNA(a?,b!)[availability > 98, price per call = 0.1]
S5(a?,b!) := PatientDNA(a?,b!)[availability > 96, price per call = 0.1]
S6(a?,b!) := DiseaseInfectedPatients(a?,k!), PatientDNA(k?,b!)[availability > 93]
S7(a?,b!) := PatientPersonalInfo(a?,b!)[availability > 99, price per call = 0.1]
S8(a?,b!) := PatientPersonalInfo(a?,b!)[availability > 98, price per call = 0.1]
S9(a?,c!,b!) := PatientPersonalInfo(a?,b!),PatientDNA(a?,c!)[availability > 98, price per call = 0.1] 
\end{verbatim}
\end{tiny}

In this example, the services S2, S5 and S6 can not be selected as candidate concrete services since they violate user preferences

The second step of the algorithm tries to create \textit{concrete services description} (CSD) to be used in the rewriting process. A CSD maps abstract services and variables of a concrete service to abstract services and variables of the query. A CSD is created according to the following variable mapping rules:
\begin{itemize}
\item \textit{Rule 1}: head variables in the concrete service can be mapped to head or local variables in the query if they are from the same type
\item \textit{Rule 2}: local variables in the concrete service can be mapped to head variables in the query if they are from the same type
\item \textit{Rule 3}: local variable in the concrete service can be mapped to a local variable in the query if: (i) they are from the same type; and (ii) the concrete service cover all abstract service in the query that depends on this variable. Depends here means that this local variable is used as input in another abstract service.
\end{itemize}

The principle of the solution for the second step is: given a list of candidate concrete services, looks for a candidate concrete services that satisfies the the variable mapping rules. For the ones who satisfies the rules, a CSD is created. As result a list of CSDs is produced.

To illustrate the selection and creation of concrete service description, consider the concrete services selected in the previous step below.
\begin{tiny}
\begin{verbatim}
S1(a?,b!) := DiseaseInfectedPatients(a?,b!)[availability > 99, price per call = 0.1]
S3(a?,b!,c!) := DiseaseInfectedPatients(a?,b!,c!)[availability > 98, price per call = 0.1]
S4(a?,b!) := PatientDNA(a?,b!)[availability > 98, price per call = 0.1]
S7(a?,b!) := PatientPersonalInfo(a?,b!)[availability > 99, price per call = 0.1]
S8(a?,b!) := PatientPersonalInfo(a?,b!)[availability > 98, price per call = 0.1]
S9(a?,c!,b!) := PatientPersonalInfo(a?,b!),PatientDNA(a?,c!)[availability > 98, price per call = 0.1] 
\end{verbatim}
\end{tiny}

In this example six CSDs are created once all concrete services satisfy the rules to create CSDs. The third step of the algorithm generates all combinations of CSDs. As result we have a list of list of CSDs.

The fifth and final step of the algorithm verifies/identifies which combination of CSDs is a valid rewriting of the user query. A combination of CSDs is a valid rewriting if:
\begin{itemize}
\item The number of abstract services in the query is equal to the result of adding the number of abstract services of each CSD
\item There is no duplicity/redundancy of abstract services in the list of CSD
\item All head variables in the query must be mapped to a variable in one of the concrete services in the list of CSDs
\item If the query has a composite measure, this measure is updated for each rewriting produced, and this measure can not be violated
\end{itemize}

As result of this step we have a list of rewriting of the query. To illustrate let us consider the example used before.
In our query we have a preference which is associated to the rewritings (composite measure) and not to a single service. Considering this preference, we have to update its value while producing the rewritings.
The value of total cost is this example is updated by aggregating the value of price per call of each service. The rewritings produced that can satisfy the user preference while aggregating these values are below. Note that more than three rewritings can be produced that composite measure did not exists. The rewrintgs are listed in the lexicographical order considering the concrete services
.
\begin{verbatim}
Q(disease?, info!, dna!) := S1(disease?,p!) S7(p?,info!) S4(p?,dna!)
Q(disease?, info!, dna!) := S3(disease?,p!, _) S7(p?,info!) S4(p?,dna!)
Q(disease?, info!, dna!) := S1(disease?,p!) S8(p?,info!) S4(p?,dna!)
\end{verbatim}




\section{Formal definition of the Rhone service based query rewriting algorithm}
\section{Basic concepts}

A user willing to perform a data integration task defines \textit{(i)} a \textsl{query}; and \textit{(ii)} a set of  \textsl{requirements} over the services or/and over the entire composition (integration). \textsl{Queries} are specified in terms of \textsl{abstract services}' definitions.

% ----------------------------------- DEFINITION 1 ABSTRACT SERVICE --------------------------------- %
\bigskip
\noindent \textbf{Definition 1 (\textsl{Abstract service})}. An \textsl{abstract service} describes the small piece of function performed by a \textsl{service} deployed by a \textsl{data provider}. For instance, retrieve weather information, book a hotel, retrieve infected patients, among others. The \textsl{abstract service} is defined as follows: $A \ (\overline{I}; \ \overline{O})$ where $A$ is the name which identifies the \textsl{abstract service}. $\overline{I}$ and $\overline{O}$ are a set of comma-separated input and output parameters, respectively.

% ----------------------------------- DEFINITION 2 QUERY --------------------------------- %
\bigskip
\noindent \textbf{Definition 2 (\textsl{Query})}.
An user \textsl{query} $Q$ is defined as a sequence of \textsl{abstract services} followed by a set of \textsl{user requirements} in accordance with the grammar:
%
\begin{center}
\begin{math}
Q (\overline{I}_{h}; \overline{O}_{h}) := A_{1}(\overline{I}_{1l};
\overline{O}_{1l}), A_{2}(\overline{I}_{2l}; \overline{O}_{2l}), ..,  A_{n}(\overline{I}_{nl}; \overline{O}_{nl}),R_{1},R_{2}, .., R_{m}
\end{math}
\end{center}
%

The \textsl{query} is defined in terms of \textsl{abstract services} ($A_{1}, A_{2}, .., A_{n}$) including a set of \textsl{user requirements} ($R_{1},R_{2}, .., R_{m}$). 
The left-hand of the definition is called the \textit{head}; it defines \textsl{query} name $Q$, a set of input $\overline{I}$ and output $\overline{O}$ variables, respectively. Variables in the \textit{head} are identified by $\overline{I}_{h}$ and $\overline{O}_{h}$, and they are called \textit{head} variables. The right-hand is the \textit{body} definition; it includes a set of \textsl{abstract services} followed by a set of \textsl{user requirements}. \textsl{Abstract services} in the \textit{body} also defines input and output variables. Variables appearing in the \textit{body} are identified by $\overline{I}_{l}$ and $\overline{O}_{l}$. \textit{Head} variables can appear in the \textit{body}, and variables appearing only in the \textit{body} are called \textit{local} variables.
%\textit{Head} variables can be accessed and shared among different services. On the other hand, \textit{local} variables can be used only by the service which define them.
%The left-hand of the definition is called the \textit{head}; and the right-hand is the \textit{body}. 
%A \textsl{query} $Q$ includes a set of input $\overline{I}$ and output $\overline{O}$ variables, respectively.
%Variables in the \textit{head} are identified by $\overline{I}_{h}$ and $\overline{O}_{h}$, and called \textit{head} variables. 
%They appear in the \textit{head} and in the \textit{body} definition. 
%Variables appearing only in the \textit{body} are identified by $\overline{I}_{l}$ and $\overline{O}_{l}$, and are called \textit{local} variables. \textit{Head} variables can be accessed and shared among different services. On the other hand, \textit{local} variables can be used only by the service which define them.
%

\bigskip
In order to be able to compare queries, we assume that there is an ontology (like the one used in~\cite{Barhamgi2010}) to describe the \textsl{abstract services} semantics. The comparison between \textsl{queries} is allowed by the query containment concept.

% ----------------------------------- DEFINITION 3 QUERY CONTAINMENT --------------------------------- %
\bigskip
\noindent \textbf{Definition 3 (\textsl{Query containment})}.
A \textsl{query} $Q_{1}$ is contained in a \textsl{query} $Q_{2}$, denoted by $Q_{1} \subset Q_{2}$, if and only if the result to $Q_{1}$ is a subset of the result to $Q_{2}$.
%

\bigskip
Intuitively, the query equivalence is defined as follows:

% ----------------------------------- DEFINITION 4 QUERY CONTAINMENT --------------------------------- %
\bigskip
\noindent \textbf{Definition 4 (\textsl{Query equivalence})}.
A \textsl{query} $Q_{1}$ is equivalent to a \textsl{query} $Q_{2}$, denoted by $Q_{1} \equiv Q_{2}$, if and only if $Q_{1} \subset Q_{2}$ and $Q_{2} \subset Q_{1}$.
%

% ----------------------------------- DEFINITION 3 USER REQUIREMENTS --------------------------------- %
\bigskip
\noindent \textbf{Definition 5 (\textsl{User requirement})}.
An \textsl{user requirement} $r$ is in the form $x \otimes c$, where $x$ is an identifier; $c$ is a constant; and $\otimes \in\lbrace \geq, \leq, =, \neq, <, >\rbrace$. 
%
The \textsl{user requirement} $r$ could concern \textit{(i)} the entire \textsl{query}, in this case noted as $r_{Q}$; or \textit{(ii)} a single \textsl{service}, noted as $r_{S}$. For instance, the \textsl{total response time} is obtained by adding the \textsl{response time} of each \textsl{service} involved in the composition.

% ----------------------------------- DEFINITION 4 USER REQUIREMENTS DOMAIN --------------------------------- %
\bigskip
\noindent \textbf{Definition 6 (\textsl{Requirement domain})}. A \textsl{requirement domain} is a set of possible values which can be assumed by an \textsl{user requirement} $r$, represented by $Dom(r)$. For instance, a \textsl{requirement domain} ``\textit{response time}'' includes the possible values associated to the \textit{response time} \textsl{user requirement}. Each \textsl{user requirement} $r_{i}$ has its own \textsl{requirement domain} $D_{i}$. 

% ----------------------------------- DEFINITION 5 EVALUATION --------------------------------- %
\bigskip
\noindent \textbf{Definition 7 (\textsl{User requirement evaluation})}. The evaluation of an \textsl{user requirement} $r$, indicated by $eval(r)$, returns a set of values $\lbrace v_{1},..,v_{i} \rbrace$ that can be assigned to $r$ such that $\lbrace v_{1},..,v_{i} \rbrace \subset Dom(r)$.

% ----------------------------------- DEFINITION 6 COMPARABLE REQUIREMENTS --------------------------------- %
\bigskip
\noindent \textbf{Definition 8 (\textsl{Comparable requirements})}. Given two \textsl{user requirements} $r_{1}$ and $r_{2}$, both can be comparable, denoted by $r_{1} \perp r_{2}$, if and only if:  $Dom(r_{1}) = Dom(r_{2})$.

% ----------------------------------- DEFINITION 7 REQUIREMENTS EQUIVALENCE --------------------------------- %
\bigskip
\noindent \textbf{Definition 9 (\textsl{User requirements equivalence})}.
A set of \textsl{user requirements} $R_{1}$ is equivalent to a set of \textsl{user requirements} $R_{2}$, represented by $R_{1} \equiv R_{2}$, if and only if: $\forall r_{i} \in R_{1}, \ \exists r_{j} \in R_{2} \ \vert \ eval (r_{i}) = eval(r_{j}) \ and \ \vert R_{1} \vert = \vert R_{2} \vert$.

% ----------------------------------- DEFINITION 8 REQUIREMENTS MORE RESTRICT --------------------------------- %
\bigskip
\noindent \textbf{Definition 10 (\textsl{User requirements more restrict})}.
Given a set of \textsl{user requirements} $R_{1}$ and $R_{2}$, $R_{1}$ is more restrict than $R_{2}$, represented by $R_{1} \ \rhd \ R_{2}$, if and only if:
\begin{flushleft}
Case 1: $\forall r_{i} \in R_{1}, \ \exists r_{j} \in R_{2}, \ \nexists r_{k} \in R_{2} \ \vert \ eval (r_{i}) \subset eval(r_{j}) \ and \ eval (r_{k}) \subset eval(r_{i}) \ and \ \vert R_{1} \vert = \vert R_{2} \vert$.
\end{flushleft}
\begin{flushleft}
Case 2: $\forall r_{i} \in R_{1}, \ \exists r_{j} \in R_{1}, \ \exists r_{k} \in R_{2}, \ \nexists r_{l} \in R_{2} \ \vert \ \neg (r_{j} \ \bot \ r_{k}) \ and \ eval (r_{l}) \subset eval(r_{i})$.
\end{flushleft}
%\begin{flushleft}
%Case 3: $\forall r_{i} \in R_{1}, \ \exists r_{j} \in R_{2} \ \vert \ eval (r_{i}) \supset eval(r_{j}) \ and \ \vert R_{1} \vert > \vert R_{2} \vert$. 
%\end{flushleft}

% ----------------------------------- DEFINITION 9 REQUIREMENTS LESS RESTRICT --------------------------------- %
%\bigskip
%\noindent \textbf{Definition 9 (\textsl{User requirements less restrict})}.
%Given a set of \textsl{user requirements} $R_{1}$ and $R_{2}$, $R_{1}$ is less restrict than $R_{2}$, represented by $R_{1} \ \lhd \ R_{2}$, if and only if:
%\begin{flushleft}
%Case 1: $\forall r_{i} \in R_{1}, \ \exists r_{j} \in R_{2} \ \vert \ eval (r_{i}) \supset eval(r_{j}) \ and \ \vert R_{1} \vert = \vert R_{2} \vert$.
%\end{flushleft}
%\begin{flushleft}
%Case 2: $\forall r_{j} \in R_{2}, \ \exists r_{i} \in R_{1} \ \vert \ eval (r_{j}) \subset eval(r_{i}) \ and \ \vert R_{1} \vert < \vert R_{2} \vert$.
%\end{flushleft}

% ----------------------------------- DEFINITION 10 PART MORE / LESS RESTRICT --------------------------------- %
\bigskip
\noindent \textbf{Definition 11 (\textsl{Part of the user requirements more restrict and part less restrict})}.
Given a set of \textsl{user requirements} $R_{1}$ and $R_{2}$, part of the \textsl{user requirements} $R_{1}$ can be more restrict and part less restrict than the \textsl{user requirements} $R_{2}$, represented by $R_{1} \ \diamond \ R_{2}$, if and only if:
\begin{flushleft}
$\forall r_{i} \in R_{1}, \ \exists r_{j} \in R_{2} \ \vert \ (eval (r_{i}) \subset eval(r_{j}) \ or \ eval (r_{i}) \supset eval(r_{j})) \ and \ \vert R_{1} \vert = \vert R_{2} \vert$.
\end{flushleft}

% ----------------------------------- DEFINITION 11 REQUIREMENTS COMPLETELY DIFFERENT --------------------------------- %
\bigskip
\noindent \textbf{Definition 12 (\textsl{User requirements different})}.
Given a set of \textsl{user requirements} $R_{1}$ and $R_{2}$, $R_{1}$ is different of $R_{2}$, represented by $R_{1} \ \neq \ R_{2}$, if and only if:
\begin{flushleft}
$\forall r_{i} \in R_{1}, \ \nexists r_{j} \in R_{2} \ \vert \ eval (r_{i}) \subset eval(r_{j}) \ or \ eval (r_{i}) \supset eval(r_{j})$.
\end{flushleft}
%
%% ----------------------------------- DEFINITION 7 QUERY EQUIVALENCE  --------------------------------- %
%\bigskip
%\noindent \textbf{Definition 11 (\textsl{Query equivalence})}.
%Definition 2 (query equivalence): a query Q1 is equivalent to a query Q2 if: 
%Q1 and Q2 have the same number of abstract services
%For each abstract services in Q1 there is an equivalent in Q2
%
%% ----------------------------------- DEFINITION 8 QUERY SUBSET  --------------------------------- %
%\bigskip
%\noindent \textbf{Definition 12 (\textsl{Query subset})}.
%Definition 10 (query subset): a query Q1 is a subset of the query Q2 if: 
%Q1 has less abstract services than Q2
%For each abstract service in Q1 there is an equivalent in Q2.
%In other words, the query Q1 is contained in the query Q2.
%
%% ----------------------------------- DEFINITION 9 QUERY ??  --------------------------------- %
%\bigskip
%\noindent \textbf{Definition 13 (\textsl{different queries but with some abstract services in common})}.
%Definition 11 (different queries but with some abstract services in common): this case occurs when given a query Q1 and a query Q2: 
%Q1 and Q2 have a different number of abstract services
%There is at least one abstract service in Q1 that has an equivalent in Q2;
%There is at least one abstract service in Q1 that has no equivalent in Q2; and
%There is at least one abstract service in Q2 that has no equivalent in Q1;


\section{Query Variations}

For all the cases described bellow, we assume that there exists a previous processed integration request including \textit{(i)} a \textsl{query} $Q_{p}$; and \textit{(ii)} a set of \textsl{user requirements} $R_{p}$ specified over it.

\subsection{Equivalent \textsl{query} and equivalent \textsl{user requirements}}
Given a user request defining a \textsl{query} $Q_{n}$ and a set of \textsl{requirements} $R_{n}$, the case in which the \textsl{queries} and the \textsl{requirements} are equivalent occurs if and only if: $Q_{n} \equiv Q_{p}$ and $R_{n} \equiv R_{p}$.

\bigskip
\noindent \textbf{Solution:} This is the best case. The previous composition generated to the integration could be re-executed if all involved \textsl{concrete services} could be enforced. Thus, a new \textsl{integration SLA} would be produced and stored to the new request.

\subsection{Equivalent \textsl{query} and more restrict \textsl{user requirements}}
Given a user request defining a \textsl{query} $Q_{n}$ and a set of \textsl{requirements} $R_{n}$, the case in which the \textsl{queries} are equivalent and the \textsl{requirements} for the new request are more restrict occurs if and only if: $Q_{n} \equiv Q_{p}$ and $R_{n} \triangleright R_{p}$.

\bigskip
\noindent \textbf{Solution:} In this case,  the previous composition generated to the integration could be re-executed if all involved \textsl{concrete services} could be enforced to the new request according to the new requirements. Otherwise, the \textsl{concrete services} that do not satisfy the \textsl{requirements} are re-allocated by the ones which satisfy. Once a new composition that is in accordance with the \textsl{requirements} is produced, a new \textsl{integration SLA} is produced and stored to the new request.

\subsection{Equivalent \textsl{query} and less restrict \textsl{user requirements}}
Given a user request defining a \textsl{query} $Q_{n}$ and a set of \textsl{requirements} $R_{n}$, the case in which the \textsl{queries} are equivalent and the \textsl{requirements} for the new request are less restrict occurs if and only if: $Q_{n} \equiv Q_{p}$ and $R_{p} \triangleright R_{n}$.

\bigskip
\noindent \textbf{Solution:} In this case,  the previous composition generated to the integration could be re-executed if all involved \textsl{concrete services} could be enforced to the new request considering the their available cloud resources. Otherwise, the \textsl{concrete services} out of resources are re-allocated by the ones which have available resources and satisfy the \textsl{user requirements}. Once a new composition that is in accordance with the \textsl{requirements} is produced, a new \textsl{integration SLA} is produced and stored to the new request.

\subsection{\textsl{Query} subset and equivalent \textsl{user requirements}}
Given a user request defining a \textsl{query} $Q_{n}$ and a set of \textsl{requirements} $R_{n}$, the case in which the the incoming \textsl{query} is a subset of the previous \textsl{query} and the \textsl{requirements} are equivalent occurs if and only if: $Q_{n} \subset Q_{p}$ and $R_{n} \equiv R_{p}$.

\bigskip
\noindent \textbf{Solution:} In this case, the previous composition generated to the previous integration will be partially reused. The subset of the composition could be re-executed if all involved \textsl{concrete services} could be enforced considering that they have available resources. Otherwise, \textsl{services} out of resources are re-allocated with the ones that satisfy the \textsl{user requirement} and have available resources. Thus, the new composition is executed and a new \textsl{integration SLA} is produced and stored to the new request.

\subsection{\textsl{Query} subset and more restrict \textsl{user requirements}}
Given a user request defining a \textsl{query} $Q_{n}$ and a set of \textsl{requirements} $R_{n}$, the case in which the the incoming \textsl{query} is a subset of the previous \textsl{query} and the \textsl{requirements} for the new request are more restrict occurs if and only if: $Q_{n} \subset Q_{p}$ and $R_{n} \triangleright R_{p}$.

\bigskip
\noindent \textbf{Solution:} In this case, the previous composition generated to the previous integration will be partially reused. The subset of the previous composition which is interesting to the new request could be re-executed if all involved \textsl{concrete services} could be enforced considering that they have available resources and that the \textsl{user requirements} are being satisfied. Otherwise, \textsl{services} out of resources or which do not satisfy the \textsl{requirements} are re-allocated with the ones that satisfy the \textsl{user requirement} and have available resources. Thus, the new composition is executed and a new \textsl{integration SLA} is produced and stored to the new request.

\subsection{\textsl{Query} subset and less restrict \textsl{user requirements}}
Given a user request defining a \textsl{query} $Q_{n}$ and a set of \textsl{requirements} $R_{n}$, the case in which the the incoming \textsl{query} is a subset of the previous \textsl{query} and the \textsl{requirements} for the new request are less restrict occurs if and only if: $Q_{n} \subset Q_{p}$ and $R_{p} \triangleright R_{n}$.

\bigskip
\noindent \textbf{Solution:} In this case, the previous composition generated to the previous integration will be partially reused. The subset of the previous composition which is interesting to the new request could be re-executed if all involved \textsl{concrete services} could be enforced considering that they have available resources. Otherwise, \textsl{services} out of resources are re-allocated with the ones that satisfy the \textsl{user requirement} and have available resources. Thus, the new composition is executed and a new \textsl{integration SLA} is produced and stored to the new request.

\subsection{\textsl{Query} superset and equivalent \textsl{user requirements}}
Given a user request defining a \textsl{query} $Q_{n}$ and a set of \textsl{requirements} $R_{n}$, the case in which the the incoming \textsl{query} is a superset of the previous \textsl{query} and the \textsl{requirements} are equivalent occurs if and only if: $Q_{p} \subset Q_{n}$ and $R_{n} \equiv R_{p}$.

\bigskip
\noindent \textbf{Solution:} In this case, the composition generated to the previous integration will be reused if all involved \textsl{concrete services} could be enforced considering that they have available resources. The previous composition will be extended to include the \textsl{services} remaining to achieve the user needs. Once the composition is produced meeting the \textsl{requirements}, it is executed. Otherwise, \textsl{services} out of resources are re-allocated with the ones that satisfy the \textsl{user requirement} and have available resources. The composition is now extended to include the \textsl{services} remaining to achieve the user needs. Once the composition is produced meeting the \textsl{requirements}, it is executed. Thus, the new composition is executed and a new \textsl{integration SLA} is produced and stored to the new request.

\subsection{\textsl{Query} subset and more restrict \textsl{user requirements}}
Given a user request defining a \textsl{query} $Q_{n}$ and a set of \textsl{requirements} $R_{n}$, the case in which the the incoming \textsl{query} is a superset of the previous \textsl{query} and the \textsl{requirements} for the new request are more restrict occurs if and only if: $Q_{p} \subset Q_{n}$ and $R_{n} \triangleright R_{p}$.

\bigskip
\noindent \textbf{Solution:} In this case, the composition generated to the previous integration will be reused if all involved \textsl{concrete services} could be enforced considering that they have available resources and the \textsl{user requirements} are met. The composition will be extended to include the \textsl{services} remaining to achieve the user needs. Once the composition is produced meeting the \textsl{requirements}, it is executed. Otherwise, \textsl{services} out of resources or violating the \textsl{requirements} are re-allocated with the ones that satisfy the \textsl{user requirement} and have available resources. The composition is now extended to include the \textsl{services} remaining to achieve the user needs. Once the composition is produced meeting the \textsl{requirements}, it is executed. Thus, the new composition is executed and a new \textsl{integration SLA} is produced and stored to the new request.

\subsection{\textsl{Query} subset and less restrict \textsl{user requirements}}
Given a user request defining a \textsl{query} $Q_{n}$ and a set of \textsl{requirements} $R_{n}$, the case in which the the incoming \textsl{query} is a superset of the previous \textsl{query} and the \textsl{requirements} for the new request are more restrict occurs if and only if: $Q_{p} \subset Q_{n}$ and $R_{p} \triangleright R_{n}$.

\bigskip
\noindent \textbf{Solution:} In this case, the composition generated to the previous integration will be reused if all involved \textsl{concrete services} could be enforced considering that they have available resources. The composition will be extended to include the \textsl{services} remaining to achieve the user needs. Once the composition is produced meeting the \textsl{requirements}, it is executed. Otherwise, \textsl{services} out of resources are re-allocated with the ones that satisfy the \textsl{user requirement} and have available resources. The composition is now extended to include the \textsl{services} remaining to achieve the user needs. Once the composition is produced meeting the \textsl{requirements}, it is executed. Thus, the new composition is executed and a new \textsl{integration SLA} is produced and stored to the new request.

\section{Query rewriting approaches}
A query rewriting algorithm which processes queries on data provider services is proposed in~\cite{Barhamgi2010}.
A query is defined using SPARQL and data services are modeled as RDF views.
Both can be seen as a graph. 
Given a query and a set of data services, the algorithm searches for relevant services and creates a mapping table for them. 
This table shows different ways of using a data service to cover part of the query. 
Then, based on the mapping table, the algorithm generates different combinations of data services to answer the query.
A valid combination (a rewriting answer) is a service composition in which the set of data service graphs fully satisfy the query graph.  

A service composition framework to answer preference queries is proposed in~\cite{Benouaret2011}. 
%that return the top-\textit{k} service compositions based on the algorithm introduced by~\cite{Barhamgi2010}  
The concept of preferences is included to SPARQL queries, and fuzzy constraints to services.
Services and service compositions are ranked according to a fuzzy dominance relationship and fuzzy scores.
Two algorithms based on~\cite{Barhamgi2010} to generate the rewriting compositions are presented: (1) the first produces all possible rewritings before computing their scores, and then return the best ones; (2) the second uses a quality metric that combines diversity and accuracy to, incrementally, rank services and to build the best rewritings.


\bibliographystyle{plain}
\bibliography{bibliography}
\end{document}