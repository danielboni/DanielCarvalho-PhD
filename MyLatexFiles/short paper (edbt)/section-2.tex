\section{Implementation and Results}
\label{sec:implementationandresults}  
Let us suppose the following medical scenario to illustrate our service-based query rewriting algorithm. Users can retrieve information about patients, diseases, dna information and others.
To perform these function consider the \textit{abstract services} in table~\ref{table:abstractservices}. 
\textit{Abstract services} are a set of basic service capabilities. 

\begin{table}[h!]
\center
\begin{tabular}{|p{3.7cm}|p{4cm}|}
\hline
\begin{small} \textbf{\textit{Abstract Service}} \end{small} & \begin{small}\textbf{\textit{Description}} \end{small}\\ 
\hline 
\begin{small} \textit{DiseasePatients(d?,p!)} \end{small} & \begin{small} Given a disease \textit{d}, a list of patients \textit{p} infected by it is retrieved. \end{small}\\ 
\hline 
\begin{small} \textit{PatientDNA(p?,dna!)} \end{small} & \begin{small} Given a patient \textit{p}, his DNA information \textit{dna} is retrieved. \end{small}\\ 
\hline 
\begin{small} \textit{PatientInformation(p?,info!)} \end{small} & \begin{small} Given a patient \textit{p}, his personal information \textit{info} is retrieved. \end{small}\\ 
\hline 
\end{tabular} \caption{List of \textit{abstract services}}
\end{table}\label{table:abstractservices}

In our scenario, a \textit{query} expresses an abstract composition that describes the requirements of a user.
\textit{Queries} and \textit{concrete services} are defined in terms of \textit{abstract services}.
They can be associated to a single \textit{abstract service} or to a composition of them.

Let us consider the following query: \textit{a user wants to retrieve patient's personal and DNA information of patients who were infected by a disease `K' using services that have availability higher than 98\%, price per call less than 0.2 dollars, and total cost less then 1 dollar.} 

A query $Q$ tagged with user preferences is defined in accordance with the grammar:
\begin{center}
$Q (\overline{I}, \overline{O}) := A_{1}(\overline{I}, \overline{O}), A_{2}(\overline{I}, \overline{O}), ..,  A_{n}(\overline{I}, \overline{O})[P_{1},P_{2}, .., P_{k}]$
\end{center}
where the left side is the \textit{head} of the query; and the right side is the \textit{body}. 
$\overline{I}$ and $\overline{O}$ are a set of \textit{input} and \textit{output} parameters, respectively.
Input parameters present in both sides of the definition are called \textit{head variables}.
In contrast, input parameters only in the body are called \textit{local variables}.
$A_{1}, A_{2}, .., A_{n}$ are \textit{abstract services}.
$P_{1}, P_{2}, .., P_{k}$ are user preferences (over the services). Preferences are in the form $x \otimes constant$ such that $\otimes \in\lbrace \geq, \leq, =, \neq, <, >\rbrace$.
The query which express the example following our grammar is below.
The decorations ? and ! are used to specify input and output parameters, respectively. 
\begin{small}
\begin{center}
$Q (d?, dna!) := DiseasePatients(d?, p!), PatientDNA(p?, dna!),$ \\
$[availability > 99\%, \ price \ per \ call < 0.2\$, \ total \ cost < 1\$]$
\end{center} 
\end{small}

We highlight that in the query there are two types of preferences (let's refer
to them as \textit{measures}): \textit{single measures} (availability and price per call) and \textit{composed measures} (total cost).
The \textit{single measures} are the simplest type. It is a static measure which is has a name associated with an operation and a value. The\textit{ composed measure} is dynamically computed measure. It is defined as aggregations of \textit{single measures}. 

\textit{Concrete services} are defined follwing the same grammar as the
\textit{query}. The only difference is that concrete services do not have
\textit{composed measures}. We use 7 concrete services to run our approach.

In this example all the queries have 6 \textit{abstract services} and 2
\textit{single measures}. The number of local variables (dependencies) and CSDs
is being modified to see how the algorithm works under these conditions.   

By now, the analysis identified that the factor that influenciates the Rhone
performance is the number of CSDs versus the number of abstract services in the
query since they increase the number of possible combinations of CSDs.   


%To illustrate our approach in the next section, consider the \textit{concrete
% services} in table~\ref{table:concreteservices}.

% \begin{table}[h!]
% \center
% \begin{tabular}{p{7cm}}
% \hline
% \begin{small} $S1 (a?, b!) := DiseaseInfectedPatients(a?, b!)$ \end{small}\\ 
% \begin{small} $[availability > 99\%, \ price \ per \ call = 0.2\$]$ \end{small}\\ 
% \hline 
% \begin{small} $S2 (a?, b!) := DiseaseInfectedPatients(a?, b!)$ \end{small}\\
% \begin{small} $[availability > 99\%, \ price \ per \ call = 0.1\$]$ \end{small}\\ 
% \hline 
% \begin{small} $S3 (a?, b!, c!) := DiseaseInfectedPatients(a?, b!, c!)$ \end{small}\\
% \begin{small} $[availability > 98\%, \ price \ per \ call = 0.1\$]$ \end{small}\\ 
% \hline 
% \begin{small} $S4 (a?, b!) := PatientDNA(a?, b!)$ \end{small}\\
% \begin{small} $[availability > 99.5\%, \ price \ per \ call = 0.1\$]$ \end{small}\\ 
% \hline
% \begin{small} $S5 (a?, b!) := PatientDNA(a?, b!)$ \end{small}\\
% \begin{small} $[availability > 99.7\%, \ price \ per \ call = 0.1\$]$ \end{small}\\ 
% \hline
% \begin{small} $S6 (a?, b!) := PatientInformation(a?, b!)$ \end{small} \\
% \begin{small} $[availability > 99.7\%, \ price \ per \ call = 0.1\$]$ \end{small}\\ 
% \hline
% \begin{small} $S7 (a?, b!) := PatientDNA(a?, c!),PatientInformation(c?, b!)$ \end{small}\\
% \begin{small} $[availability > 99.7\%, \ price \ per \ call = 0.1\$]$ \end{small}\\ 
% \hline
% \end{tabular} \caption{Available \textit{concrete services}}
% \end{table}\label{table:concreteservices}