\section{Introduction}
Data integration is a well-know problem in the database domain~\cite{Halevy:2001}. 
Recently, with the emergency of cloud environments, this problem has also been threated in the service-oriented domains~\cite{Barhamgi2010,Benouaret2011,ba2014}.
Generally, these kind of applications deals with query rewriting problems.
\cite{Barhamgi2010} proposed a query rewriting approach which processes queries on data provider services. \cite{Benouaret2011} introduced a service composition framework to answer preference queries. In that approach, two algorithms based on~\cite{Barhamgi2010} are presented to rank the best rewritings based on previously computed scores.
%: one (i) to produce all possible rewritings before computing their scores and other (ii)
%which uses a quality metric that combines diversity and accuracy to,
%incrementally, rank services and to build the best rewritings.

\cite{ba2014} presented an algorithm which produces and order rewritings according to user preferences.
Our work differs from the others in two aspects: (i) the user can express his
quality preferences on his queries; and (ii) we consider the service's quality
aspects defined on SLA to service selection and to produce rewritings.    

In this context, the aim of our work is to present the early stages of our
ongoing work on developing the \textit{Rhone} service-based query rewriting
algorithm.

The remainder of this paper is organized as follows. Section~\ref{sec:rhone}
describes the proposed algorithm. Section~\ref{sec:implementationandresults} describes
the running scenario and also the implementation process.
Finally, section~\ref{sec:conclusions} presents the conclusions.
