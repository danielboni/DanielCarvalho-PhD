\section{Service-based Query Rewriting Algorithm}
\label{sec:rhone} 
This section describes \textit{Rhone} the service-based query rewriting algorithm that we propose. Given
a set of \textit{abstract services}, a set of \textit{concrete services}, a
\textit{user query} and a set of user \textit{quality preferences}, derive a set
of service compositions that answer the query and that fulfill the quality preferences
 regarding the context of data service deployment.

The  input for  Rhone  is: (1) a query; (2) a list of concrete services defined in the following lines.

\noindent \textbf{Definition 1 (Query):} 
%A query $Q$ has the form:
A query $Q$ is defined as a set of \textit{abstract services} ($A$), a set of \textit{constraints} ($C$), and a set of \textit{user preferences}($P$) in accordance with the grammar: 

\begin{center}
$Q (\overline{I}, \overline{O}) := A_{1}(\overline{I}, \overline{O}), A_{2}(\overline{I}, \overline{O}), ..,  A_{n}(\overline{I}, \overline{O}),C_{1},C_{2}, .., C_{m}[P_{1},P_{2}, .., P_{k}]$
\end{center}  

The left side of the definition is called the \textit{head} of the query; and the right side is called the \textit{body}. 
$\overline{I}$ and $\overline{O}$ are a set of \textit{input} and \textit{output} parameters, respectively.
Input parameters in both sides of the definition are called \textit{head variables}.
In contrast, input parameters only in the query body are called \textit{local variables}.
The preferences (for short \textit{measures}) are classified as: \textit{single measures} (static measures) and \textit{composed measures} (aggregations of \textit{single measures}).
%
%The result if this step is a list of \textit{candidate concrete services} which
% may be used in the rewriting process.

\noindent \textbf{Definition 2 (Concrete service)}  ($S$):
\begin{center}
$S (\overline{I}, \overline{O}) := A_{1}(\overline{I}, \overline{O}), A_{2}(\overline{I}, \overline{O}), ..,  A_{n}(\overline{I}, \overline{O})[P_{1},P_{2}, .., P_{k}]$
\end{center}  

A concrete service ($S$) is defined as a set of abstract services ($A_{i}$), and by their quality constraints $P_{j}$. 
These quality constraints associated to the service represent the Service Level Agreement (SLA) exported by the concrete service.

The algorithm consists in four steps: (i) select candidate concrete services; (ii)
create mappings from concrete services to the query (called \textit{concrete
service description (CSD)}); (iii) combine the list of CSDs; and finally (iv)
produce rewritings.

\begin{algorithm}
\small
\caption{ - RHONE}
\label{algo-rhone}
\begin{algorithmic}[1]
\STATE \textbf{function} $\mathit{rhone} (Q, \bigS)$
 \STATE  $\bigLS \leftarrow \mathit{SelectCandidateServices}(Q, \bigS)$ \label{rhone:buildPCD}
 \STATE  $\bigLCSD \leftarrow CreateCSDs(Q, \bigLS)$
 \STATE  $I \leftarrow CombineCSDs(\bigLCSD)$
 \STATE $R\leftarrow \emptyset$
% \STATE ~\!\tqI{\agg{Q}} 
    \STATE $p \leftarrow I.next()$
    \WHILE {$p\ \neq\ \emptyset$ \AND ~\!\tqI{\agg{Q}}} 
      \IF {\textit{isRewriting}$(Q, p)$}
  \STATE $R\leftarrow R\,\cup \mathit{Rewriting}(p)$
  \STATE ~\!\tqS{\agg{Q}}
   \ENDIF
      \STATE $p \leftarrow I.\mathit{Next}()$
 \ENDWHILE
    \STATE \textbf{return} $R$
\STATE \textbf{end function}
\end{algorithmic}
\end{algorithm}

\noindent \textbf{Select candidate concrete services:} This step
 looks for concrete services that can be matched with the query (line 2). In
 this sense, there are three matching problems: 
 (i) \textit{abstract service matching}, an abstract service $A$ can be
 matched with an abstract service $B$ only if (\textit{a}) they have the same
 name, and (\textit{b}) they have a compatible number of variables;
 (ii) \textit{measure matching}, all \textit{single measures} in the query must
 exist in the concrete service, and all of them can not violate the measures in
 the query ; and 
 (iii) \textit{concrete service matching}, a concrete service can
 be matched with the query if all its abstract services satisfy the \textit{abstract service
 matching} problem and all the \textit{single measures} satisfy the \textit{measures matching} problem.
% Considering \textit{abstract service matching}, an abstract service $A$ can be
% matched with an abstract service $B$ only if (\textit{a}) they have the same
% name.  In the \textit{measure matching}, all single measures in the query must
% exist in the concrete service, and all of them can not violate the measures in
% the query. For the \textit{concrete service matching}, a concrete service can
% be matched with the query if all its abstract services can be matched with the
% abstract service in the query (satisfying the \textit{abstract service
% matching} problem) and all the single measures in the query can be matched with
% the concrete service measures (satisfying the \textit{measures matching} problem).
The result of this step is a list of \textit{candidate concrete services} which
may be used in the rewriting process.


\noindent \textbf{Creating concrete service descriptions:} This step tries to create \textit{concrete services description} (CSD) to be
 used in the rewriting process (line 3). A CSD maps abstract services and
 variables of a concrete service onto abstract services and variables of the
 query. A CSD is created 
% according to the following variable mapping rules:  (i)
% \textit{head variables}, (ii) \textit{local-head variables} and  (iii)
% \textit{local-local variables}.
according to variable mapping rules mainly based on 2 criterias: the type and the dependency (variables used as inputs on other abstract services). 
%\textit{Head} and \textit{local} variables in concrete services can be mapped
% to \textit{head} or \textit{local variables} in the query if they are of the same type.
%\textit{Local} variables in concrete services can be mapped to \textit{head} variables in the query. 
%\textit{Local} variables in concrete services can be mapped to \textit{local}
% variables in the query if: (\textit{a}) they are of the same type; and (\textit{b}) the concrete service covers all abstract services in the query that depend on this variable. The relation ``depends''  means that this \textit{local} variable is used as input in another abstract service. 
The  result of this step is a list of CSDs.
  
%    \textit{Head variables} in concrete services can be mapped to
%  \textit{head} or \textit{local variables} in the query if they are of the
%  same type. \textit{Local-head variables} in concrete services can be mapped to
%  \textit{head variables} in the query if they are of the same type.
%  \textit{Local-local variables} in concrete services can be mapped to
%  \textit{local-local variables} in the query if: (\textit{a}) they are of the
%  same type; and (\textit{b}) the concrete service covers all abstract services in
%  the query that depend on this variable. The relation "depends"  means that this local variable is used as input in another abstract service. The  result of this step is a list of CSDs.

%Regarding the example CSDs are created to ?????.
 
\noindent \textbf{Combining CSDs.} Given all produced CSDs  (line
4), they are combined among each other to generate  a list of lists of CSDs, each element representing a possible composition.

\noindent \textbf{Producing rewritings.} The final step (lines 5-13) identifies which lists of CSDs are a valid
rewritings of the user query given the list of lists of CSDs.
A combination of CSDs is a valid rewriting if: (i) they cover all abstract services in the query; and 
(ii) there is mapping to all head variables in the query (implemented by the function \textit{isRewriting}$(Q, p)$ - line 8).
The originality of our algorithm concerns the aggregation function (\agg{Q}).
It is responsible to check and increment \textit{composed measures} (if present in the query). 
This means for each element in the CSD list the value of \textit{composed measure} is incremented (line 10), and rewritings are produced while the values of these measures are respected (line 7). 
The result of this step is the list of valid 
 rewritings of the query (line 14).%, that is those the provide expected data
 % and respect quality preferences.

%\noindent \textbf{Producing rewritings.} In the final step, given the list of
% lists of CSDs, the algorithm identifies which lists of CSDs are a valid
% rewritings of the user query (lines 5-13).
%A combination of CSDs is a valid rewriting iff: (i) the number of
% \textit{abstract services} in the query is equal to the result of adding the
% number of \textit{abstract services} of each CSD; (ii) there is no duplicated abstract
% service; (iii) there is mapping to all head variables in the
% query; and (iv) if the query contains a \textit{composed measure}, that corresponds to the preferences associated to the query. Every element in the CSD list has its corresponding \textit{composed measure} (represented as the called function \textit{isRewriting}$(Q, p)$ - line 8). The result of this step is the list of valid 
% rewritings of the query (line 14), that is those the provide expected data and respect quality preferences.

% Regarding our example the query has a preference which is associated to the
% rewritings (composed measure). Its value is updated while rewriting the query. In that case, \emph{total cost} is updated by aggregating the value of \emph{price per call} of each service. The rewritings produced are below. Note that more rewritings can be produced if the \textit{composed measure} did not exists. The rewrintgs are listed in the lexicographical order considering the concrete services.
%\begin{tiny}
%\begin{verbatim}
%Q(disease?, info!, dna!) := S1(disease?,p!) S7(p?,info!) S4(p?,dna!)
%Q(disease?, info!, dna!) := S3(disease?,p!, _) S7(p?,info!) S4(p?,dna!)
%Q(disease?, info!, dna!) := S1(disease?,p!) S8(p?,info!) S4(p?,dna!)
%\end{verbatim}
%\end{tiny}