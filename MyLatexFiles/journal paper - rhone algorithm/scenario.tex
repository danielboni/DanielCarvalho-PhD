To illustrate the definition, let us suppose the set of abstract services in Table \ref{example1} and the Example 1.
%Let us suppose the following query example in order to illustrate the definition.

\begin{table}[h]
\center
\begin{tabular}{|p{7cm}|p{7cm}|}
\hline 
\textbf{\textit{Abstract Service}} & \textbf{\textit{Description}} \\ 
\hline 
\textit{DiseaseInfectedPatients(d?,p!)} & Given a disease \textit{d}, a list of patients \textit{p} infected by it is retrieved. \\ 
\hline 
\textit{PatientDNA(p?,dna!)} & Given a patient \textit{p}, his DNA information \textit{dna} is retrieved. \\ 
\hline 
\textit{PatientPersonalInformation(p?,info!)} & Given a patient \textit{p}, his personal information \textit{info} is retrieved. \\ 
\hline 
\end{tabular} \caption{Abstract services description}
\end{table}\label{example1}


\noindent \textbf{Example 1:} %Let us suppose the following simple query. 
\textit{The user wants to retrieve the DNA information from patients infected by the disease `K' using services that have availability higher than 99\%, price per call less than 0.2 dollars, and the total cost less then 1 dollar}.
%Query and concrete service are formally defined as follows.

The query which express the Example 1 according to the Definition 1 and the abstract services in Table \ref{example1} is specified below.
The decorations ? and ! are used to specify input and output parameters, respectively. 

\begin{center}
$Q (d?, dna!) := DiseaseInfectedPatients(d?, p!), PatientDNA(p?, dna!),$ \\
$d = ``K"[availability > 99\%, \ price \ per \ call < 0.2\$, \ total \ cost < 1\$]$
\end{center}  

Analyzing the query, it is possible to note that the parameters ``d?'' and ``dna!'' appear in both sides of the definition.
Due to that they are \textit{head} variables.
On the other hand, ``p!'' and ``p?'' are \textit{local} variables considering that they appear only in the body definition. 
Additionally, note that the local variables ``p!'' and ``p?'' have the same name.
Intuitively, this fact indicates a dependency between the abstract services which use these variables (in that case \textit{DiseaseInfectedPatients} and \textit{PatientDNA}).

In the example, \textit{DiseaseInfectedPatients} and \textit{PatientDNA} are abstract services that specify basic service functions which are combined to answer the query. 
%\textit{DiseaseInfectedPatients} retrieves the patients infected by a given disease.
%\textit{PatientDNA} retrieves that DNA information of a patient.
The constraint ($d = ``K''$) over the input parameter `d' will be further used while executing the query over a database (the where clause). 
\textit{Availability}, \textit{price per call} and \textit{total cost} are the user preferences over the services.

\noindent \textbf{Example 2:} Considering the query (see Example 1) and the abstract services (see Table \ref{example1}), the concrete services below are examples in accordance with the Definition 2.

\begin{flushleft}
$S1 (a?, b!) := DiseaseInfectedPatients(a?, b!)[availability > 99\%, \ price \ per \ call = 0.2\$]$ \\
$S2 (a?, b!) := DiseaseInfectedPatients(a?, b!)[availability > 99\%, \ price \ per \ call = 0.1\$]$ \\
$S3 (a?, b!, c!) := DiseaseInfectedPatients(a?, b!, c!)[availability > 98\%, \ price \ per \ call = 0.1\$]$ \\
$S4 (a?, b!) := PatientDNA(a?, b!)[availability > 99.5\%, \ price \ per \ call = 0.1\$]$ \\
$S5 (a?, b!) := PatientDNA(a?, b!)[availability > 99.7\%, \ price \ per \ call = 0.1\$]$ \\
$S6 (a?, b!) := PatientPersonalInformation(a?, b!)[availability > 99.7\%, \ price \ per \ call = 0.1\$]$ \\
$S7 (a?, b!) := PatientDNA(a?, c!),PatientPersonalInformation(c?, b!)[availability > 99.7\%, \ price \ per \ call = 0.1\$]$ \\

\end{flushleft}

For example, looking to the concrete services in the Example 2, the abstract service \textit{DiseaseInfectedPatients} in $S1$ and $S2$ are equivalent to the abstract service \textit{DiseaseInfectedPatients} in the query $Q$ (Example 1) once they have the same name and number of input/output parameters.
On the other hand, the abstract service \textit{DiseaseInfectedPatients} in $S3$ is not equivalent to the abstract service \textit{DiseaseInfectedPatients} in the query because the number of parameters are different.

For example, considering the query in the Example 1 and the concrete services in the Example 2, it is possible to see that:
%\begin{itemize}
%\item S1 and S3 are not a candidate services because they violate the user preferences.
(1) $S1$ is not a candidate service because it violates an user preference (\textit{price per call});
(2) $S3$ and $S7$ are not a candidate service because they have abstract services that are not in $Q$; and
(3) $S2$, $S4$ and $S5$ are candidate services once: all their abstract services have an equivalent in $Q$ and there is no violation in the user preference.
%\end{itemize}

%\noindent \textbf{Definition 4 (Candidate service):} Given a query $Q$ and a concrete service $S$, $S$ is a candidate service iff $\nexists A_{i} \ s.t. \ A_{i} \in S \ and \ A_{i} \not\in Q$. $\forall A_{i} \in S$ and an abstract service $A_{i}$ such that $A_{i} \in Q$ and $A_{i} \in S$, the mapping $A_{i}.Q \longmapsto A_{i}.S$ can be generated iff:
%(1) $A_{i}.Q = A_{i}.S$; and (2) $\nexists A \ s.t. \ A \in S \ and \ A \not\in Q$.

%In the next step, the algorithm builds the \textit{partial descriptors} (PD). 
%Each descriptor describes how a \textit{candidate} concrete service can be used in the query rewriting process.

\noindent \textbf{Example 3:} To illustrate the rules above consider the following example. 
\textit{The user wants to retrieve the personal information and the DNA information from patients infected by disease ``K"}.
Supposing we have the query $Q$ and the concrete services $S1$, $S2$, $S3$ and $S4$:

\begin{center}
$Q (d?, info!, dna!) := DiseaseInfectedPatients(d?, p!), PatientDNA(p?, dna!), PatientPersonalInformation(p?, info!)$ \\
$S1 (a?, b!) := DiseaseInfectedPatients(a?, c!), PatientDNA(c?, b!)$ \\
$S2 (a?, b!) := PatientPersonalInformation(a?, b!)$ \\
$S3 (a?, b!) := DiseaseInfectedPatients(a?, b!)$ \\
$S4 (a?, b!) := PatientDNA(a?, b!)$ \\
\end{center} 

In the query $Q$ it is possible to note that ``p!'' is a \textit{local} variable which is used as input (``p?'') for the abstract services $A2$ and $A3$. 
Looking to the concrete service $S1$ no CSD will be created for it because the \textit{local} variable $c!$ is mapped to the local variable $p!$, but $S1$ does not cover all abstract services which expects that variable. 
On the other hand, CSDs are constructed to the services $S2$, $S3$ and $S4$ once even existing the mapping from a local variable in the concrete service to a local variable in the query, all of them only cover one abstract service which uses that \textit{local} variable.
To be more clear about these rules, consider the rewriting below in which the CSDs for the services  $S2$, $S3$ and $S4$ are used:

\begin{center}
$Q (d?, info!, dna!) := S3(d?, p!), S4(p?, dna!), S2(p?, info!)$ \\
\end{center}  

The rewriting above is the only one possible for the query. 
However, let us suppose that a CSD for $S1$ was created violating the rule number two, consequently the wrong rewriting below would be created:

\begin{center}
$Q (d?, info!, dna!) := S1(d?, info!),S4(p?, dna!)$ \\
\end{center}  

The problem here is regarding the \textit{local} variable $p?$ which appears in $S4$, and it apparently should come from $S1$, but we can not guarantee that the same \textit{local} variable internally used in $S1$ is the one expected by $S4$. 
That is the reason the rule two exists.