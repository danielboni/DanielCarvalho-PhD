\textcolor{red}{Lembrar rhone iterator... adicionar sub-algoritmos}

\begin{algorithm}
\small
\caption{ - RHONE}
\label{algo-rhone}
\begin{algorithmic}[1]
\STATE \textbf{function} $\mathit{rhone} (Q, \bigS)$
 \STATE  $\bigLS \leftarrow \mathit{SelectCandidateServices}(Q, \bigS)$ \label{rhone:buildPCD}
 \STATE  $\bigLCSD \leftarrow CreateCSDs(Q, \bigLS)$
 \STATE  $I \leftarrow CombineCSDs(\bigLCSD)$
 \STATE $R\leftarrow \emptyset$
% \STATE ~\!\tqI{\agg{Q}} 
    \STATE $p \leftarrow I.next()$
    \WHILE {$p\ \neq\ \emptyset$ \AND ~\!\tqI{\agg{Q}}} 
      \IF {\textit{isRewriting}$(Q, p)$}
  \STATE $R\leftarrow R\,\cup \mathit{Rewriting}(p)$
  \STATE ~\!\tqS{\agg{Q}}
   \ENDIF
      \STATE $p \leftarrow I.\mathit{Next}()$
 \ENDWHILE
    \STATE \textbf{return} $R$
\STATE \textbf{end function}
\end{algorithmic}
\end{algorithm}

\noindent \textbf{Select candidate concrete services:} This step
 looks for concrete services that can be matched with the query (line 2). In
 this sense, there are three matching problems: 
 (i) \textit{abstract service matching}, an abstract service $A$ can be
 matched with an abstract service $B$ only if (\textit{a}) they have the same
 name, and (\textit{b}) they have a compatible number of variables;
 (ii) \textit{measure matching}, all \textit{single measures} in the query must
 exist in the concrete service, and all of them can not violate the measures in
 the query ; and 
 (iii) \textit{concrete service matching}, a concrete service can
 be matched with the query if all its abstract services satisfy the \textit{abstract service
 matching} problem and all the \textit{single measures} satisfy the \textit{measures matching} problem.
% Considering \textit{abstract service matching}, an abstract service $A$ can be
% matched with an abstract service $B$ only if (\textit{a}) they have the same
% name.  In the \textit{measure matching}, all single measures in the query must
% exist in the concrete service, and all of them can not violate the measures in
% the query. For the \textit{concrete service matching}, a concrete service can
% be matched with the query if all its abstract services can be matched with the
% abstract service in the query (satisfying the \textit{abstract service
% matching} problem) and all the single measures in the query can be matched with
% the concrete service measures (satisfying the \textit{measures matching} problem).
The result of this step is a list of \textit{candidate concrete services} which
may be used in the rewriting process.


\noindent \textbf{Creating concrete service descriptions:} This step tries to create \textit{concrete services description} (CSD) to be
 used in the rewriting process (line 3). A CSD maps abstract services and
 variables of a concrete service onto abstract services and variables of the
 query. A CSD is created 
% according to the following variable mapping rules:  (i)
% \textit{head variables}, (ii) \textit{local-head variables} and  (iii)
% \textit{local-local variables}.
according to variable mapping rules mainly based on 2 criterias: the type and the dependency (variables used as inputs on other abstract services). 
%\textit{Head} and \textit{local} variables in concrete services can be mapped
% to \textit{head} or \textit{local variables} in the query if they are of the same type.
%\textit{Local} variables in concrete services can be mapped to \textit{head} variables in the query. 
%\textit{Local} variables in concrete services can be mapped to \textit{local}
% variables in the query if: (\textit{a}) they are of the same type; and (\textit{b}) the concrete service covers all abstract services in the query that depend on this variable. The relation ``depends''  means that this \textit{local} variable is used as input in another abstract service. 
The  result of this step is a list of CSDs.
  
%    \textit{Head variables} in concrete services can be mapped to
%  \textit{head} or \textit{local variables} in the query if they are of the
%  same type. \textit{Local-head variables} in concrete services can be mapped to
%  \textit{head variables} in the query if they are of the same type.
%  \textit{Local-local variables} in concrete services can be mapped to
%  \textit{local-local variables} in the query if: (\textit{a}) they are of the
%  same type; and (\textit{b}) the concrete service covers all abstract services in
%  the query that depend on this variable. The relation "depends"  means that this local variable is used as input in another abstract service. The  result of this step is a list of CSDs.

%Regarding the example CSDs are created to ?????.
 
\noindent \textbf{Combining CSDs.} Given all produced CSDs  (line
4), they are combined among each other to generate  a list of lists of CSDs, each element representing a possible composition.

\noindent \textbf{Producing rewritings.} The final step (lines 5-13) identifies which lists of CSDs are a valid
rewritings of the user query given the list of lists of CSDs.
A combination of CSDs is a valid rewriting if: (i) they cover all abstract services in the query; and 
(ii) there is mapping to all head variables in the query (implemented by the function \textit{isRewriting}$(Q, p)$ - line 8).
The originality of our algorithm concerns the aggregation function (\agg{Q}).
It is responsible to check and increment \textit{composed measures} (if present in the query). 
This means for each element in the CSD list the value of \textit{composed measure} is incremented (line 10), and rewritings are produced while the values of these measures are respected (line 7). 
The result of this step is the list of valid 
 rewritings of the query (line 14).%, that is those the provide expected data
 % and respect quality preferences.
 
\textcolor{red}{Texto antigo... adicionar funcoes e ligar ao texto existente.} 
 
 The function \agg{\_} in Algorithm~\ref{algo-rhone} simply selects the aggregation conditions present in the query. 
This function is defined as follows:
\[
\agg{\mathbf{SELECT}\ \alpha\ \mathbf{FROM}\ \beta\ \mathbf{WHERE}\ \gamma\ \mathbf{SUCH\ THAT}\ \delta} = \delta 
\]

Where $\delta \in \Delta$ is defined by the following grammar rules (where $c$ denotes any constant and $x$ denotes any identifier:
\begin{eqnarray*}
\Delta & ::= & E = E \mid E \leq E \mid \dots \mid \Delta\ ;\ \Delta   \\
E      & ::= & c \mid x \mid \$\mathit{SUM(x)} \mid \$\mathit{AVG(x)} \mid \dots \mid \\
       &     & E + E \mid E - E \mid E * E \mid E / E \mid ( E )
\end{eqnarray*}

The translation functions \tqI{\_}, \tqT{\_} and \tqS{\_} take the aggregation portion of a query and produce code, respectively, for the initialization, property check and increment of the variables which implement the aggregation part of the query.
Notice that these function definitions are overloaded (they take both elements from $\Delta$ and $E$.
They are defined as follows (we use the operation ``+'' for string concatenation and $\lambda$ is the empty string):

\begin{eqnarray*}
\tqI{\delta_1 ; \ \delta_2}	& = & \tqI{\delta_1} + \hbox{``;''} + \tqI{\delta_2} \\
\tqI{E_1 \Bumpeq \ E_2} 		& = & \tqI{E_1} + \hbox{``;''} + \tqI{E_2}
									\qquad \mathit{where} \Bumpeq \in \{ =, \leq, <, \dots \}\\
\tqI{c} 						& = & \lambda \\
\tqI{x} 						& = & \lambda \\
\tqI{\$\mathit{SUM(x)}}	 	& = & \hbox{``\textbf{var} $x_{sum} = 0$''} \\
\tqI{\$\mathit{AVG(x)}}	 	& = & \hbox{``\textbf{var} $x_{sum} = 0$ ; \textbf{var} $x_{count} = 0$''} \\
\tqI{E_1 \odot \ E_2} 		& = & \tqI{E_1} + \hbox{``;''} + \tqI{E_2} 
									\qquad \mathit{where}\ \odot \in \{ +, -, *, / \}\\
\tqI{( E )} 					& = & \tqI{E} 
\end{eqnarray*}

The function \tqI{\_} generates the initialization code for the aggregates in the query. 
Notice that, for the sake of simplicity, we suppose that there is no repetition among aggregates.

The next definition generates the tests generated for each aggregation condition in the query:

\begin{eqnarray*}
\tqT{\delta_1 ; \ \delta_2} 	& = & \tqT{\delta_1} + \hbox{``$\wedge$''} + \tqT{\delta_2} \\
\tqT{E_1 \Bumpeq \ E_2} 		& = & \tqT{E_1} + \hbox{``$\wedge$''} + \tqT{E_2}
									\qquad \mathit{where} \Bumpeq \in \{ =, \leq, <, \dots \}\\
\tqT{c} 						& = & c \\
\tqT{x} 						& = & x \\
\tqT{\$\mathit{SUM(x)}} 		& = & \hbox{``$x_{sum}$''} \\
\tqT{\$\mathit{AVG(x)}} 		& = & \hbox{``$x_{sum} / x_{count}$''} \\
\tqT{E_1 \odot \ E_2}		& = & \tqI{E_1} + \hbox{``$\odot$''} + \tqI{E_2} 
									\qquad \mathit{where}\ \odot \in \{ +, -, *, / \}\\
\tqT{( E )} 					& = & \hbox{``(''} + \tqI{E} \hbox{``)''}
\end{eqnarray*}

The function \tqS{\_} is responsible for generating the operations that update the variables involved in the aggregation expressions of the query.
This function is defined as follows:

\begin{eqnarray*}
\tqS{\delta_1 ; \ \delta_2} 	& = & \tqS{\delta_1} + \hbox{``;''} + \tqS{\delta_2} \\
\tqS{E_1 \Bumpeq \ E_2} 		& = & \tqS{E_1} + \hbox{``;''} + \tqS{E_2}
									\qquad \mathit{where} \Bumpeq \in \{ =, \leq, <, \dots \}\\
\tqS{c} 						& = & \lambda \\
\tqS{x} 						& = & \lambda \\
\tqS{\$\mathit{SUM(x)}} 		& = & \hbox{``$x_{sum} := x_{sum} + x$''} \\
\tqS{\$\mathit{AVG(x)}} 		& = & \hbox{``$x_{sum}:= x_{sum} + x ; x_{count} := x_{count} + 1$''} \\
\tqS{E_1 \odot \ E_2}		& = & \tqS{E_1} + \hbox{``;''} + \tqS{E_2} 
									\qquad \mathit{where}\ \odot \in \{ +, -, *, / \}\\
\tqS{( E )} 					& = & \tqS{E}
\end{eqnarray*}

The function $\mathit{BuildPCDs}(Q, \bigS)$ (Algorithm~\ref{algo-rhone}, line~\ref{rhone:buildPCD}) is responsible for creating one PCD for each concrete service.

The iterator $I$ is responsible for producing sets of PCDs such that \textit{(i)} their combination \textit{covers} the body of the abstract query and \textit{(ii)} they are produced in a decreasing order of priority (the first set produced by the iterator maximizes the combined preference).

The iterator $I$ implements a dynamic priority queue. 
Its initialization stores the most preferred solutions in the queue (according to the order determined by the SLAs).
The operation $I.\mathit{Next}()$ de-queues the next preferred set of PCDs, calculates the successors of this set (in the pareto order determined by the SLA), inserts the successors in the priority queue and finally returns the de-queued set.