\documentclass[12pt,a4paper,oneside]{report}
\usepackage[utf8]{inputenc}
\usepackage[english]{babel}
\usepackage{amsmath}
\usepackage{amsfonts}
\usepackage{amssymb}
\usepackage{graphicx}
\usepackage[left=2cm,right=2cm,top=2cm,bottom=2cm]{geometry}
\begin{document}

\begin{center}
\textbf{\large{Report July 6th}} \\
\textbf{\large{Partial Coverage Descriptor Output using the Implementation}}
\end{center}

This document contains the PCDs created to the diseases' example. 

\bigskip

\begin{flushleft}
\textsl{\textbf{ 1. First Input data}}
\end{flushleft}

\begin{flushleft}
\textbf{Query:} \\
Q(disease, patients, dnas, birth, gen, addr, wheatherstatistic) :- A1(disease, patients), A2(patients,dnas), A3(patients, birth, gen, addr), A4(addr, wheatherstatistic) \\
\end{flushleft}

\begin{flushleft}
\textbf{Concrete services:} \\
S1(disease, patientsssn) :- A1(disease, patientsssn)\\
S2(disease, patientsssn) :- A1(disease, patientsssn)\\
S3(patientsssn, dna) :- A2(patientsssn, dna)\\
S4(patientsssn, dateofbirth, gender, address) :- A3(patientsssn, dateofbirth, gender, address)\\
S5(address, wheatherstaticalinformation) :- A4(address, wheatherstaticalinformation)\\
S6(patientsssn, diseases) :- A5(patientsssn, diseases)\\
S7(patientsssn, vaccinateddiseases) :- A6(patientsssn, vaccinateddiseases)\\
\end{flushleft}

\begin{flushleft}
As expected the \textbf{rewriting results} were: \\
\textbf{1)} Q(disease,patients,dnas,birth,gen,addr,wheatherstatistic) :- S1(disease,patients),S3(patients,dnas),S4(patients,birth,gen,addr),S5(addr,wheatherstatistic)
\textbf{2)} Q(disease,patients,dnas,birth,gen,addr,wheatherstatistic) :- S2(disease,patients),S3(patients,dnas),S4(patients,birth,gen,addr),S5(addr,wheatherstatistic)
\end{flushleft}

\begin{flushleft}
\textbf{PCDs created by the current version of the algorithm:} \\
\textbf{1)} Concrete service: S1(disease,patientsssn) :- A1(disease,patientsssn) \\
Mappings of variables: disease $\longrightarrow$ disease; patients $\longrightarrow$ patientsssn; \\

\textbf{2)} Concrete service: S2(disease,patientsssn) :- A1(disease,patientsssn) \\
Mappings of variables: disease $\longrightarrow$ disease; patients $\longrightarrow$ patientsssn; \\

\textbf{3)} Concrete service: S3(patientsssn,dna) :- A2(patientsssn,dna) \\
Mappings of variables: patients $\longrightarrow$ patientsssn; dnas $\longrightarrow$ dna; \\

\textbf{4)} Concrete service: S4(patientsssn,dateofbirth,gender,address) :- A3(patientsssn,dateofbirth,gender,address) \\
Mappings of variables: patients $\longrightarrow$ patientsssn; birth $\longrightarrow$ dateofbirth; gen $\longrightarrow$ gender; addr $\longrightarrow$ address; \\

\textbf{5)} Concrete service: S5(address,wheatherstaticalinformation) :- A4(address,wheatherstaticalinformation) \\
Mappings of variables: addr $\longrightarrow$ address; wheatherstatistic $\longrightarrow$ wheatherstaticalinformation; 
\end{flushleft}

Looking for the output generated by the algorithm is seems to be different from the definition on the paper. There are mission elements such as the set \textbf{def}, \textbf{h}, \textbf{G} and \textbf{has\_opt}. These elements are not defined and used in the MCD object (extended version from the Original minicon). As I will show in the next example, the set \textbf{h} does not express dependencies between terms in the composition, and also between elements from the head of the concrete definition (left side) to the terms in the abstract services definition (right side).

\begin{flushleft}
\textsl{\textbf{ 2. Second Input data}}
\end{flushleft}

\begin{flushleft}
\textbf{Query:} \\
Q(disease, patients, dnas, birth, gen, addr, wheatherstatistic) :- A1(disease, patients), A2(patients,dnas), A3(patients, birth, gen, addr), A4(addr, wheatherstatistic) \\
\end{flushleft}

\begin{flushleft}
\textbf{Concrete services:} \\
\textbf{\textit{S2(disease, patientsssn, dna) :- A1(disease, patientsssn), A2(patientsssn, dna)}}\\
S3(patientsssn, dna) :- A2(patientsssn, dna)\\
S4(patientsssn, dateofbirth, gender, address) :- A3(patientsssn, dateofbirth, gender, address)\\
S5(address, wheatherstaticalinformation) :- A4(address, wheatherstaticalinformation)\\
S6(patientsssn, diseases) :- A5(patientsssn, diseases)\\
S7(patientsssn, vaccinateddiseases) :- A6(patientsssn, vaccinateddiseases)\\
\end{flushleft}

In this example I removed the concrete service S1 and added a new abstract service to the definition of S2. Now S2 can solve S1 and S3.

\begin{flushleft}
The \textbf{rewriting results} were: \\
T\textbf{1) }Q(disease,patients,dnas,birth,gen,addr,wheatherstatistic) :- S2(disease,patients,\_),S2(\_,patients,dnas),S4(patients,birth,gen,addr),S5(addr,wheatherstatistic) \\
\textbf{2) }Q(disease,patients,dnas,birth,gen,addr,wheatherstatistic) :- S2(disease,patients,\_),S3(patients,dnas),S4(patients,birth,gen,addr),S5(addr,wheatherstatistic)
\end{flushleft}

\begin{flushleft}
\textbf{PCDs created by the current version of the algorithm:} \\
\textbf{1) }Concrete service: S2(disease,patientsssn,dna) :- A1(disease,patientsssn),A2(patientsssn,dna)\\
Mappings of variables: disease $\longrightarrow$ disease; patients $\longrightarrow$ patientsssn; 

\textbf{2) }Concrete service: S2(disease,patientsssn,dna) :- A1(disease,patientsssn),A2(patientsssn,dna)\\
Mappings of variables: patients $\longrightarrow$ patientsssn; dnas $\longrightarrow$ dna; 

\textbf{3) }Concrete service: S3(patientsssn,dna) :- A2(patientsssn,dna)\\
Mappings of variables: patients $\longrightarrow$ patientsssn; dnas $\longrightarrow$ dna; 

\textbf{4) }Concrete service: S4(patientsssn,dateofbirth,gender,address) :- A3(patientsssn,dateofbirth,gender,address)\\
Mappings of variables: patients $\longrightarrow$ patientsssn; birth $\longrightarrow$ dateofbirth; gen $\longrightarrow$ gender; addr $\longrightarrow$ address; 

\textbf{5) }Concrete service: S5(address,wheatherstaticalinformation) :- A4(address,wheatherstaticalinformation)\\
Mappings of variables: addr $\longrightarrow$ address; wheatherstatistic $\longrightarrow$ wheatherstaticalinformation; 
\end{flushleft}

As I mentioned in the meeting, I was suspecting that two different PCDs are defined to the same concrete service, and as we can see in the output that what happens. I believe that's why we have the first answer using the underscore and two calls for the same service. Also, there is no definition for the set \textbf{h} expressing the dependencies between the services,  but the second answer seems to be good. However, I have no idea (yet) where in the code this kind of dependency is defined.

\end{document}