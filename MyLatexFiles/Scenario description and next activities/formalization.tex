\section{Basic concepts}

A user willing to perform a data integration task defines \textit{(i)} a \textsl{query}; and \textit{(ii)} a set of  \textsl{requirements} over the services or/and over the entire composition (integration). \textsl{Queries} are specified in terms of \textsl{abstract services}' definitions.

% ----------------------------------- DEFINITION 1 ABSTRACT SERVICE --------------------------------- %
\bigskip
\noindent \textbf{Definition 1 (\textsl{Abstract service})}. An \textsl{abstract service} describes the small piece of function performed by a \textsl{service} deployed by a \textsl{data provider}. For instance, retrieve weather information, book a hotel, retrieve infected patients, among others. The \textsl{abstract service} is defined as follows: $A \ (\overline{I}; \ \overline{O})$ where $A$ is the name which identifies the \textsl{abstract service}. $\overline{I}$ and $\overline{O}$ are a set of comma-separated input and output parameters, respectively.

% ----------------------------------- DEFINITION 2 QUERY --------------------------------- %
\bigskip
\noindent \textbf{Definition 2 (\textsl{Query})}.
An user \textsl{query} $Q$ is defined as a sequence of \textsl{abstract services} followed by a set of \textsl{user requirements} in accordance with the grammar:
%
\begin{center}
\begin{math}
Q (\overline{I}_{h}; \overline{O}_{h}) := A_{1}(\overline{I}_{1l};
\overline{O}_{1l}), A_{2}(\overline{I}_{2l}; \overline{O}_{2l}), ..,  A_{n}(\overline{I}_{nl}; \overline{O}_{nl}),R_{1},R_{2}, .., R_{m}
\end{math}
\end{center}
%

The \textsl{query} is defined in terms of \textsl{abstract services} ($A_{1}, A_{2}, .., A_{n}$) including a set of \textsl{user requirements} ($R_{1},R_{2}, .., R_{m}$). 
The left-hand of the definition is called the \textit{head}; it defines \textsl{query} name $Q$, a set of input $\overline{I}$ and output $\overline{O}$ variables, respectively. Variables in the \textit{head} are identified by $\overline{I}_{h}$ and $\overline{O}_{h}$, and they are called \textit{head} variables. The right-hand is the \textit{body} definition; it includes a set of \textsl{abstract services} followed by a set of \textsl{user requirements}. \textsl{Abstract services} in the \textit{body} also defines input and output variables. Variables appearing in the \textit{body} are identified by $\overline{I}_{l}$ and $\overline{O}_{l}$. \textit{Head} variables can appear in the \textit{body}, and variables appearing only in the \textit{body} are called \textit{local} variables.
%\textit{Head} variables can be accessed and shared among different services. On the other hand, \textit{local} variables can be used only by the service which define them.
%The left-hand of the definition is called the \textit{head}; and the right-hand is the \textit{body}. 
%A \textsl{query} $Q$ includes a set of input $\overline{I}$ and output $\overline{O}$ variables, respectively.
%Variables in the \textit{head} are identified by $\overline{I}_{h}$ and $\overline{O}_{h}$, and called \textit{head} variables. 
%They appear in the \textit{head} and in the \textit{body} definition. 
%Variables appearing only in the \textit{body} are identified by $\overline{I}_{l}$ and $\overline{O}_{l}$, and are called \textit{local} variables. \textit{Head} variables can be accessed and shared among different services. On the other hand, \textit{local} variables can be used only by the service which define them.
%

\bigskip
In order to be able to compare queries, we assume that there is an ontology (like the one used in~\cite{Barhamgi2010}) to describe the \textsl{abstract services} semantics. The comparison between \textsl{queries} is allowed by the query containment concept.

% ----------------------------------- DEFINITION 3 QUERY CONTAINMENT --------------------------------- %
\bigskip
\noindent \textbf{Definition 3 (\textsl{Query containment})}.
A \textsl{query} $Q_{1}$ is contained in a \textsl{query} $Q_{2}$, denoted by $Q_{1} \subset Q_{2}$, if and only if the result to $Q_{1}$ is a subset of the result to $Q_{2}$.
%

\bigskip
Intuitively, the query equivalence is defined as follows:

% ----------------------------------- DEFINITION 4 QUERY CONTAINMENT --------------------------------- %
\bigskip
\noindent \textbf{Definition 4 (\textsl{Query equivalence})}.
A \textsl{query} $Q_{1}$ is equivalent to a \textsl{query} $Q_{2}$, denoted by $Q_{1} \equiv Q_{2}$, if and only if $Q_{1} \subset Q_{2}$ and $Q_{2} \subset Q_{1}$.
%

% ----------------------------------- DEFINITION 3 USER REQUIREMENTS --------------------------------- %
\bigskip
\noindent \textbf{Definition 5 (\textsl{User requirement})}.
An \textsl{user requirement} $r$ is in the form $x \otimes c$, where $x$ is an identifier; $c$ is a constant; and $\otimes \in\lbrace \geq, \leq, =, \neq, <, >\rbrace$. 
%
The \textsl{user requirement} $r$ could concern \textit{(i)} the entire \textsl{query}, in this case noted as $r_{Q}$; or \textit{(ii)} a single \textsl{service}, noted as $r_{S}$. For instance, the \textsl{total response time} is obtained by adding the \textsl{response time} of each \textsl{service} involved in the composition.

% ----------------------------------- DEFINITION 4 USER REQUIREMENTS DOMAIN --------------------------------- %
\bigskip
\noindent \textbf{Definition 6 (\textsl{Requirement domain})}. A \textsl{requirement domain} is a set of possible values which can be assumed by an \textsl{user requirement} $r$, represented by $Dom(r)$. For instance, a \textsl{requirement domain} ``\textit{response time}'' includes the possible values associated to the \textit{response time} \textsl{user requirement}. Each \textsl{user requirement} $r_{i}$ has its own \textsl{requirement domain} $D_{i}$. 

% ----------------------------------- DEFINITION 5 EVALUATION --------------------------------- %
\bigskip
\noindent \textbf{Definition 7 (\textsl{User requirement evaluation})}. The evaluation of an \textsl{user requirement} $r$, indicated by $eval(r)$, returns a set of values $\lbrace v_{1},..,v_{i} \rbrace$ that can be assigned to $r$ such that $\lbrace v_{1},..,v_{i} \rbrace \subset Dom(r)$.

% ----------------------------------- DEFINITION 6 COMPARABLE REQUIREMENTS --------------------------------- %
\bigskip
\noindent \textbf{Definition 8 (\textsl{Comparable requirements})}. Given two \textsl{user requirements} $r_{1}$ and $r_{2}$, both can be comparable, denoted by $r_{1} \perp r_{2}$, if and only if:  $Dom(r_{1}) = Dom(r_{2})$.

% ----------------------------------- DEFINITION 7 REQUIREMENTS EQUIVALENCE --------------------------------- %
\bigskip
\noindent \textbf{Definition 9 (\textsl{User requirements equivalence})}.
A set of \textsl{user requirements} $R_{1}$ is equivalent to a set of \textsl{user requirements} $R_{2}$, represented by $R_{1} \equiv R_{2}$, if and only if: $\forall r_{i} \in R_{1}, \ \exists r_{j} \in R_{2} \ \vert \ eval (r_{i}) = eval(r_{j}) \ and \ \vert R_{1} \vert = \vert R_{2} \vert$.

% ----------------------------------- DEFINITION 8 REQUIREMENTS MORE RESTRICT --------------------------------- %
\bigskip
\noindent \textbf{Definition 10 (\textsl{User requirements more restrict})}.
Given a set of \textsl{user requirements} $R_{1}$ and $R_{2}$, $R_{1}$ is more restrict than $R_{2}$, represented by $R_{1} \ \rhd \ R_{2}$, if and only if:
\begin{flushleft}
Case 1: $\forall r_{i} \in R_{1}, \ \exists r_{j} \in R_{2}, \ \nexists r_{k} \in R_{2} \ \vert \ eval (r_{i}) \subset eval(r_{j}) \ and \ eval (r_{k}) \subset eval(r_{i}) \ and \ \vert R_{1} \vert = \vert R_{2} \vert$.
\end{flushleft}
\begin{flushleft}
Case 2: $\forall r_{i} \in R_{1}, \ \exists r_{j} \in R_{1}, \ \exists r_{k} \in R_{2}, \ \nexists r_{l} \in R_{2} \ \vert \ \neg (r_{j} \ \bot \ r_{k}) \ and \ eval (r_{l}) \subset eval(r_{i})$.
\end{flushleft}
%\begin{flushleft}
%Case 3: $\forall r_{i} \in R_{1}, \ \exists r_{j} \in R_{2} \ \vert \ eval (r_{i}) \supset eval(r_{j}) \ and \ \vert R_{1} \vert > \vert R_{2} \vert$. 
%\end{flushleft}

% ----------------------------------- DEFINITION 9 REQUIREMENTS LESS RESTRICT --------------------------------- %
%\bigskip
%\noindent \textbf{Definition 9 (\textsl{User requirements less restrict})}.
%Given a set of \textsl{user requirements} $R_{1}$ and $R_{2}$, $R_{1}$ is less restrict than $R_{2}$, represented by $R_{1} \ \lhd \ R_{2}$, if and only if:
%\begin{flushleft}
%Case 1: $\forall r_{i} \in R_{1}, \ \exists r_{j} \in R_{2} \ \vert \ eval (r_{i}) \supset eval(r_{j}) \ and \ \vert R_{1} \vert = \vert R_{2} \vert$.
%\end{flushleft}
%\begin{flushleft}
%Case 2: $\forall r_{j} \in R_{2}, \ \exists r_{i} \in R_{1} \ \vert \ eval (r_{j}) \subset eval(r_{i}) \ and \ \vert R_{1} \vert < \vert R_{2} \vert$.
%\end{flushleft}

% ----------------------------------- DEFINITION 10 PART MORE / LESS RESTRICT --------------------------------- %
\bigskip
\noindent \textbf{Definition 11 (\textsl{Part of the user requirements more restrict and part less restrict})}.
Given a set of \textsl{user requirements} $R_{1}$ and $R_{2}$, part of the \textsl{user requirements} $R_{1}$ can be more restrict and part less restrict than the \textsl{user requirements} $R_{2}$, represented by $R_{1} \ \diamond \ R_{2}$, if and only if:
\begin{flushleft}
$\forall r_{i} \in R_{1}, \ \exists r_{j} \in R_{2} \ \vert \ (eval (r_{i}) \subset eval(r_{j}) \ or \ eval (r_{i}) \supset eval(r_{j})) \ and \ \vert R_{1} \vert = \vert R_{2} \vert$.
\end{flushleft}

% ----------------------------------- DEFINITION 11 REQUIREMENTS COMPLETELY DIFFERENT --------------------------------- %
\bigskip
\noindent \textbf{Definition 12 (\textsl{User requirements different})}.
Given a set of \textsl{user requirements} $R_{1}$ and $R_{2}$, $R_{1}$ is different of $R_{2}$, represented by $R_{1} \ \neq \ R_{2}$, if and only if:
\begin{flushleft}
$\forall r_{i} \in R_{1}, \ \nexists r_{j} \in R_{2} \ \vert \ eval (r_{i}) \subset eval(r_{j}) \ or \ eval (r_{i}) \supset eval(r_{j})$.
\end{flushleft}
%
%% ----------------------------------- DEFINITION 7 QUERY EQUIVALENCE  --------------------------------- %
%\bigskip
%\noindent \textbf{Definition 11 (\textsl{Query equivalence})}.
%Definition 2 (query equivalence): a query Q1 is equivalent to a query Q2 if: 
%Q1 and Q2 have the same number of abstract services
%For each abstract services in Q1 there is an equivalent in Q2
%
%% ----------------------------------- DEFINITION 8 QUERY SUBSET  --------------------------------- %
%\bigskip
%\noindent \textbf{Definition 12 (\textsl{Query subset})}.
%Definition 10 (query subset): a query Q1 is a subset of the query Q2 if: 
%Q1 has less abstract services than Q2
%For each abstract service in Q1 there is an equivalent in Q2.
%In other words, the query Q1 is contained in the query Q2.
%
%% ----------------------------------- DEFINITION 9 QUERY ??  --------------------------------- %
%\bigskip
%\noindent \textbf{Definition 13 (\textsl{different queries but with some abstract services in common})}.
%Definition 11 (different queries but with some abstract services in common): this case occurs when given a query Q1 and a query Q2: 
%Q1 and Q2 have a different number of abstract services
%There is at least one abstract service in Q1 that has an equivalent in Q2;
%There is at least one abstract service in Q1 that has no equivalent in Q2; and
%There is at least one abstract service in Q2 that has no equivalent in Q1;


\section{Query Variations}

For all the cases described bellow, we assume that there exists a previous processed integration request including \textit{(i)} a \textsl{query} $Q_{p}$; and \textit{(ii)} a set of \textsl{user requirements} $R_{p}$ specified over it.

\subsection{Equivalent \textsl{query} and equivalent \textsl{user requirements}}
Given a user request defining a \textsl{query} $Q_{n}$ and a set of \textsl{requirements} $R_{n}$, the case in which the \textsl{queries} and the \textsl{requirements} are equivalent occurs if and only if: $Q_{n} \equiv Q_{p}$ and $R_{n} \equiv R_{p}$.

\bigskip
\noindent \textbf{Solution:} This is the best case. The previous composition generated to the integration could be re-executed if all involved \textsl{concrete services} could be enforced. Thus, a new \textsl{integration SLA} would be produced and stored to the new request.

\subsection{Equivalent \textsl{query} and more restrict \textsl{user requirements}}
Given a user request defining a \textsl{query} $Q_{n}$ and a set of \textsl{requirements} $R_{n}$, the case in which the \textsl{queries} are equivalent and the \textsl{requirements} for the new request are more restrict occurs if and only if: $Q_{n} \equiv Q_{p}$ and $R_{n} \triangleright R_{p}$.

\bigskip
\noindent \textbf{Solution:} In this case,  the previous composition generated to the integration could be re-executed if all involved \textsl{concrete services} could be enforced to the new request according to the new requirements. Otherwise, the \textsl{concrete services} that do not satisfy the \textsl{requirements} are re-allocated by the ones which satisfy. Once a new composition that is in accordance with the \textsl{requirements} is produced, a new \textsl{integration SLA} is produced and stored to the new request.

\subsection{Equivalent \textsl{query} and less restrict \textsl{user requirements}}
Given a user request defining a \textsl{query} $Q_{n}$ and a set of \textsl{requirements} $R_{n}$, the case in which the \textsl{queries} are equivalent and the \textsl{requirements} for the new request are less restrict occurs if and only if: $Q_{n} \equiv Q_{p}$ and $R_{p} \triangleright R_{n}$.

\bigskip
\noindent \textbf{Solution:} In this case,  the previous composition generated to the integration could be re-executed if all involved \textsl{concrete services} could be enforced to the new request considering the their available cloud resources. Otherwise, the \textsl{concrete services} out of resources are re-allocated by the ones which have available resources and satisfy the \textsl{user requirements}. Once a new composition that is in accordance with the \textsl{requirements} is produced, a new \textsl{integration SLA} is produced and stored to the new request.

\subsection{\textsl{Query} subset and equivalent \textsl{user requirements}}
Given a user request defining a \textsl{query} $Q_{n}$ and a set of \textsl{requirements} $R_{n}$, the case in which the the incoming \textsl{query} is a subset of the previous \textsl{query} and the \textsl{requirements} are equivalent occurs if and only if: $Q_{n} \subset Q_{p}$ and $R_{n} \equiv R_{p}$.

\bigskip
\noindent \textbf{Solution:} In this case, the previous composition generated to the previous integration will be partially reused. The subset of the composition could be re-executed if all involved \textsl{concrete services} could be enforced considering that they have available resources. Otherwise, \textsl{services} out of resources are re-allocated with the ones that satisfy the \textsl{user requirement} and have available resources. Thus, the new composition is executed and a new \textsl{integration SLA} is produced and stored to the new request.

\subsection{\textsl{Query} subset and more restrict \textsl{user requirements}}
Given a user request defining a \textsl{query} $Q_{n}$ and a set of \textsl{requirements} $R_{n}$, the case in which the the incoming \textsl{query} is a subset of the previous \textsl{query} and the \textsl{requirements} for the new request are more restrict occurs if and only if: $Q_{n} \subset Q_{p}$ and $R_{n} \triangleright R_{p}$.

\bigskip
\noindent \textbf{Solution:} In this case, the previous composition generated to the previous integration will be partially reused. The subset of the previous composition which is interesting to the new request could be re-executed if all involved \textsl{concrete services} could be enforced considering that they have available resources and that the \textsl{user requirements} are being satisfied. Otherwise, \textsl{services} out of resources or which do not satisfy the \textsl{requirements} are re-allocated with the ones that satisfy the \textsl{user requirement} and have available resources. Thus, the new composition is executed and a new \textsl{integration SLA} is produced and stored to the new request.

\subsection{\textsl{Query} subset and less restrict \textsl{user requirements}}
Given a user request defining a \textsl{query} $Q_{n}$ and a set of \textsl{requirements} $R_{n}$, the case in which the the incoming \textsl{query} is a subset of the previous \textsl{query} and the \textsl{requirements} for the new request are less restrict occurs if and only if: $Q_{n} \subset Q_{p}$ and $R_{p} \triangleright R_{n}$.

\bigskip
\noindent \textbf{Solution:} In this case, the previous composition generated to the previous integration will be partially reused. The subset of the previous composition which is interesting to the new request could be re-executed if all involved \textsl{concrete services} could be enforced considering that they have available resources. Otherwise, \textsl{services} out of resources are re-allocated with the ones that satisfy the \textsl{user requirement} and have available resources. Thus, the new composition is executed and a new \textsl{integration SLA} is produced and stored to the new request.

\subsection{\textsl{Query} superset and equivalent \textsl{user requirements}}
Given a user request defining a \textsl{query} $Q_{n}$ and a set of \textsl{requirements} $R_{n}$, the case in which the the incoming \textsl{query} is a superset of the previous \textsl{query} and the \textsl{requirements} are equivalent occurs if and only if: $Q_{p} \subset Q_{n}$ and $R_{n} \equiv R_{p}$.

\bigskip
\noindent \textbf{Solution:} In this case, the composition generated to the previous integration will be reused if all involved \textsl{concrete services} could be enforced considering that they have available resources. The previous composition will be extended to include the \textsl{services} remaining to achieve the user needs. Once the composition is produced meeting the \textsl{requirements}, it is executed. Otherwise, \textsl{services} out of resources are re-allocated with the ones that satisfy the \textsl{user requirement} and have available resources. The composition is now extended to include the \textsl{services} remaining to achieve the user needs. Once the composition is produced meeting the \textsl{requirements}, it is executed. Thus, the new composition is executed and a new \textsl{integration SLA} is produced and stored to the new request.

\subsection{\textsl{Query} subset and more restrict \textsl{user requirements}}
Given a user request defining a \textsl{query} $Q_{n}$ and a set of \textsl{requirements} $R_{n}$, the case in which the the incoming \textsl{query} is a superset of the previous \textsl{query} and the \textsl{requirements} for the new request are more restrict occurs if and only if: $Q_{p} \subset Q_{n}$ and $R_{n} \triangleright R_{p}$.

\bigskip
\noindent \textbf{Solution:} In this case, the composition generated to the previous integration will be reused if all involved \textsl{concrete services} could be enforced considering that they have available resources and the \textsl{user requirements} are met. The composition will be extended to include the \textsl{services} remaining to achieve the user needs. Once the composition is produced meeting the \textsl{requirements}, it is executed. Otherwise, \textsl{services} out of resources or violating the \textsl{requirements} are re-allocated with the ones that satisfy the \textsl{user requirement} and have available resources. The composition is now extended to include the \textsl{services} remaining to achieve the user needs. Once the composition is produced meeting the \textsl{requirements}, it is executed. Thus, the new composition is executed and a new \textsl{integration SLA} is produced and stored to the new request.

\subsection{\textsl{Query} subset and less restrict \textsl{user requirements}}
Given a user request defining a \textsl{query} $Q_{n}$ and a set of \textsl{requirements} $R_{n}$, the case in which the the incoming \textsl{query} is a superset of the previous \textsl{query} and the \textsl{requirements} for the new request are more restrict occurs if and only if: $Q_{p} \subset Q_{n}$ and $R_{p} \triangleright R_{n}$.

\bigskip
\noindent \textbf{Solution:} In this case, the composition generated to the previous integration will be reused if all involved \textsl{concrete services} could be enforced considering that they have available resources. The composition will be extended to include the \textsl{services} remaining to achieve the user needs. Once the composition is produced meeting the \textsl{requirements}, it is executed. Otherwise, \textsl{services} out of resources are re-allocated with the ones that satisfy the \textsl{user requirement} and have available resources. The composition is now extended to include the \textsl{services} remaining to achieve the user needs. Once the composition is produced meeting the \textsl{requirements}, it is executed. Thus, the new composition is executed and a new \textsl{integration SLA} is produced and stored to the new request.