\documentclass[12pt,a4paper,oneside]{book}
\usepackage[utf8]{inputenc}
\usepackage[english]{babel}
\usepackage{amsmath}
\usepackage{amsfonts}
\usepackage{amssymb}
\usepackage{graphicx}
\usepackage[left=2cm,right=2cm,top=2cm,bottom=2cm]{geometry}
\author{Daniel Aguiar da Silva Carvalho}
\title{State of The Art}
\begin{document}

%-------------------------------------------------------------------------------------------------------%
\chapter{State of The Art}

\section{Service Level Agreements}

\subsection{A Framework for SLA-Based Cloud Services Verification and Composition}

\begin{description}
\item Authors: Asma Al Falasi, Mohamed Adel Serhami
\item Reference: \cite{001}
\item Year: 2011 
\item Resume: They propose a framework for dynamic specification of SLAs. The focus of their approach is on a SLA-based model for the verification and composition of the services. Their approach starts from the dynamic SLAs negotiation, then the verification and composition process, until the agreement. The framework is composed by three components: (i) \textit{A Third-Party Cloud Directory} is the intermediary between the costumers and providers. Providers should sign up with the directory and customers can search and initiate a negotiation with a selected provider. Customers define their SLOs using WSOL; (ii) \textit{The Cloud providers} expose their infrastructure as web services. During a SLA negotiation, the provider search for candidate concrete services that realizes the customers' requirements. After that the provider asks for the composition broker to come up with the optimal service SLA with the requirements; and (iii) \textit{A Trusted Composition Broker} uses the E$^{3}$-MOGA genetic algorithm to find the optimal cloud services composition.
\item Advantages: the framework enables the client to change his SLOs at runtime. I really do not understand how does it works just reading the paper.
\item Disadvantages: 
\begin{itemize}
\item There is no example to illustrate the use of the approach;
\item There is no simulation/tests of the framework in order to know if it is efficient or not;
\item There is no way to know if the algorithm used in the verification process is the best choice;
\item Perhaps concentrate all the mediation between the providers and the customers in a single third-party cloud directory could create a ``point of fail'' to the model ;
\item Maybe other QoS parameters should be considered in the SLA. Not only throughput, response and cost.
\end{itemize}
\end{description}

\subsection{}

\begin{description}
\item Authors: 
\item Reference: 
\item Year: 
\item Resume: 
\item Advantages: 
\item Disadvantages: 
\begin{itemize}
\item
\end{itemize}
\end{description}

\subsection{}

\begin{description}
\item Authors: 
\item Reference: 
\item Year: 
\item Resume: 
\item Advantages: 
\item Disadvantages: 
\begin{itemize}
\item
\end{itemize}
\end{description}

%∞∞∞∞∞∞∞∞∞∞∞∞∞∞∞∞∞∞∞∞∞∞∞∞∞∞∞∞∞∞∞∞∞∞∞∞∞∞∞∞∞∞∞∞∞∞∞∞∞∞∞∞∞∞∞∞∞∞∞∞∞∞∞∞∞∞∞∞∞∞∞∞∞∞∞∞∞∞∞∞∞∞∞∞∞∞∞∞∞∞∞∞∞∞∞∞∞∞∞∞∞∞∞%

%-------------------------------------------------------------------------------------------------------%
\bibliographystyle{plain}
\bibliography{refs}
%∞∞∞∞∞∞∞∞∞∞∞∞∞∞∞∞∞∞∞∞∞∞∞∞∞∞∞∞∞∞∞∞∞∞∞∞∞∞∞∞∞∞∞∞∞∞∞∞∞∞∞∞∞∞∞∞∞∞∞∞∞∞∞∞∞∞∞∞∞∞∞∞∞∞∞∞∞∞∞∞∞∞∞∞∞∞∞∞∞∞∞∞∞∞∞∞∞∞∞∞∞∞∞%


\end{document}