\documentclass[12pt,a4paper,oneside]{book}
\usepackage[utf8]{inputenc}
\usepackage[english]{babel}
\usepackage{amsmath}
\usepackage{amsfonts}
\usepackage{amssymb}
\usepackage{graphicx}
\usepackage[left=2cm,right=2cm,top=2cm,bottom=2cm]{geometry}
\usepackage{titlesec}
\author{Daniel Aguiar da Silva Carvalho}
\title{State of The Art}

\begin{document}
%\renewcommand{\chaptername}{}
\titleformat{\chapter}[hang]{\bf\huge}{\thechapter}{2pc}{}
%-------------------------------------------------------------------------------------------------------%
\maketitle
\chapter{Report april - 2015}

\section{Service Level Agreements Approaches}

In~\cite{001}, the authors propose a framework for dynamic specification of SLAs. The focus of their approach is on a SLA-based model for the verification and composition of the services. Their approach starts from the dynamic SLAs negotiation, then the verification and composition process, until the agreement. The framework is composed by three components: (i) \textit{A Third-Party Cloud Directory} is the intermediary between the costumers and providers. Providers should sign up with the directory and customers can search and initiate a negotiation with a selected provider. Customers define their SLOs using WSOL; (ii) \textit{The Cloud providers} expose their infrastructure as web services. During a SLA negotiation, the provider search for candidate concrete services that realizes the customers' requirements. After that the provider asks for the composition broker to come up with the optimal service SLA with the requirements; and (iii) \textit{A Trusted Composition Broker} uses the E$^{3}$-MOGA genetic algorithm to find the optimal cloud services composition.

\begin{description}
\item Advantages: the framework enables the client to change his SLOs at runtime. I really do not understand how does it works just reading the paper.
\item Disadvantages: 
\begin{itemize}
\item There is no example to illustrate the use of the approach;
\item There is no simulation/tests of the framework in order to know if it is efficient or not;
\item There is no way to know if the algorithm used in the verification process is the best choice;
\item Perhaps concentrate all the mediation between the providers and the customers in a single third-party cloud directory could create a ``point of fail'' to the model/system;
\item Maybe other QoS parameters should be considered in the SLA. Not only throughput, response and cost.
\end{itemize}
\end{description}

\bigskip
In~\cite{002}, a review about SLAs is presented. The authors are interested in challenges associated to trust, SLA management and cloud computing \textit{(in my opinion they focus on frameworks for SLA and cloud computing definition)}. To examine/analize this challenges, the definition of cloud computing is discussed and an analisys of the use/approaches of SLAs for different domains web services, grid computing and cloud computing is performed. They finish their work discussing challenges for SLAs in cloud computing.

\bigskip
\cite{004} is an extended version of~\cite{002}. The authors surveyed performance measurement models in different domains (such as SOA, distributed systems, grid computing and cloud services) in order to develop a general trust model for cloud community. They presented.. their work discussing challenges for SLAs in cloud computing and performance models cloud computing...

\bigskip


%∞∞∞∞∞∞∞∞∞∞∞∞∞∞∞∞∞∞∞∞∞∞∞∞∞∞∞∞∞∞∞∞∞∞∞∞∞∞∞∞∞∞∞∞∞∞∞∞∞∞∞∞∞∞∞∞∞∞∞∞∞∞∞∞∞∞∞∞∞∞∞∞∞∞∞∞∞∞∞∞∞∞∞∞∞∞∞∞∞∞∞∞∞∞∞∞∞∞∞∞∞∞∞%

%-------------------------------------------------------------------------------------------------------%
\bibliographystyle{plain}
\bibliography{bibliography}
%∞∞∞∞∞∞∞∞∞∞∞∞∞∞∞∞∞∞∞∞∞∞∞∞∞∞∞∞∞∞∞∞∞∞∞∞∞∞∞∞∞∞∞∞∞∞∞∞∞∞∞∞∞∞∞∞∞∞∞∞∞∞∞∞∞∞∞∞∞∞∞∞∞∞∞∞∞∞∞∞∞∞∞∞∞∞∞∞∞∞∞∞∞∞∞∞∞∞∞∞∞∞∞%


\end{document}