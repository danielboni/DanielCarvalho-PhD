%Integrating data across different databases, and provide an unique view of it to the user, is a problem in the database domain (called data integration).
Data integration problem has been studied by many researchers in the database domain.
This can be seen in the service-oriented architecture as a service composition issue, such as given a query, the objective is to lookup for data services and compose them in order to achieve a desired result. 
%
%Finding the best service composition to answer a query can be computationally costly. 	
However, it is computationally costly to find the best service composition to answer a query. 
Furthermore, executing the composition can lead to retrieve and process data collections that can require important memory, storage and computing resources.
The on-demand resource provision imposed by the cloud computing architecture opens challenges to data processing and management.
%The cloud architecture allows and opens challenges to the data integration once such architecture give us an unlimited access to resources. 
%The possibility of having an unlimited access to resources, the resource management, the geographically distributed location of services, and the economic model imposed by the cloud architecture open challenges to data integration solutions.
Data integration has evolved with the emergence of data services that
deliver data under different quality conditions related to data freshness, cost, reliability,
availability, among others. Data are produced continuously and on demand in huge
quantities and sometimes with few associated meta-data, which makes the
integration process more challenging. Some approaches express data integration
as a service composition problem where given a query the objective is to lookup
and compose data services that can contribute to produce a result. Finding the
best service composition that can answer a query can be computationally costly.
Furthermore,  executing the composition can lead to retrieve and process data
collections that can require important memory, storage and computing resources.     

This problem has been addressed in the service-oriented
domain~\cite{Barhamgi2010,Benouaret2011,ba2014}.
Generally, these solutions deal with query rewriting problems.
\cite{Barhamgi2010} proposed a query rewriting approach which processes queries
on data provider services. \cite{Benouaret2011} introduced a service composition
framework to answer preference queries. In that approach, two algorithms based
on~\cite{Barhamgi2010} are presented to rank the best rewritings based on previously computed scores.
\cite{ba2014} presented an algorithm that produces and order rewritings
according to user preferences. Yet, to our knowledge few works consider quality
measures associated both to data services and to user preferences in order to
guide the rewriting process. 

Computing resources are delivered as services by the cloud. Services are billed and agreed between service providers and service customers under service level agreement (SLA) contracts.
Both sides must agree together on quality conditions and penalties under which the service is delivered. 
%Generally, those conditions and penalties associated to its violation are defined in service level agreement (SLA). 
Several researches have reported their studies on SLA in different domains~\cite{AlhamadDC11}.
SLA proposals for cloud computing could be divided in two groups: (i) works developing tools and methods to help on SLA negotiation and enforcement phase~\cite{rak2013,Mavrogeorgi2013}; and (ii) approaches that monitor contracts and cloud resources in order to detect and avoid SLA violations~\cite{Leitner2010,Maarouf2015}. To our knowledge, SLA publications have not been yet integrated to data integration in a multi-cloud environment.

Based on these concepts and open challenges, the goal of this work is to present
a data integration solution concerning a service-based query rewriting algorithm guided by SLA's.
Our work addresses these issues and proposes the algorithm (we
called \textit{Rhone}) with two original aspects: (i) the user can express her
quality preferences and associate them to her queries; and (ii)  service's quality aspects defined on Service Level Agreements (SLA) guide service selection and the whole rewriting process.
Yet, to the best of our knowledge, we have not identified any other work that uses SLA to guide the entire data integration solution.

This paper is organized as follows. Section ~\ref{sec:disla} discusses
about data integration and service level agreements (SLA).
Section~\ref{sec:scenario} contains the running scenario and challenges.
Section~\ref{sec:relatedwork} presents some related works.
Section~\ref{sec:rhone} describes the \textit{Rhone} algorithm and its
formalization.
Experiments and results are described in the section~\ref{sec:experiments}. 
Finally, section~\ref{sec:conclusion} concludes the paper and discusses future works.
