Integrating data across different databases and providing a unique view of it to the user is a problem in the database domain (called data integration).
%Data integration is a well-known problem in the database domain.
This problem can be seen on the service-oriented architecture as a service composition issue in which given a query, the objective is to lookup and compose data services to produce a result. 
% 	
Finding the best service composition to answer a query can be computationally costly. 
Furthermore, executing the composition can lead to retrieve and process data collections that can require important memory, storage and computing resources.
%The cloud architecture allows and opens challenges to the data integration once such architecture give us an unlimited access to resources. 
The possibility of having an unlimited access to resources, the resource management, the geographically distributed location of services, and the economic model imposed by the cloud architecture open challenges to data integration solutions.

Service provider and service customer are first class citizens on cloud architecture.
Both must agree together on quality conditions expected from the other side. 
Generally, those condition and penalties associated to its violation are defined in service level 
agreement (SLA). 
Proposals concerning SLAs on cloud computing are divided in two groups: (i)
approaches focusing on the negotiation phase between providers and customers
(REFs??); and, (ii) approaches that manage SLA to avoid SLA violation (REFs??).
To our knowledge, SLA approaches have not been integrated to data integration solutions.

The goal of this work is to present a data integration solution
concerning the \textit{Rhone} service-based query rewriting algorithm guided by SLA's.
Our work addresses this issue and proposes the algorithm (we
called \textit{Rhone}) with two original aspects: (i) the user can express her
quality preferences and associate them to her queries; and (ii)  service's quality aspects defined on Service Level Agreements (SLA) guide service selection and the whole rewriting process.
Yet, to the best of our knowledge, we have not identified any other work that uses SLA to guide the entire data integration solution.

The rest of this paper is organized as follows. 
Section~\ref{sec:relatedwork} describes our related works. 
Section~\ref{sec:scenario} contains the running scenario and challenges.
Section~\ref{sec:rhone} describes the Rhone and its formalization. 
Experiments and results are described in the section~\ref{sec:experiments}. 
Finally, section~\ref{sec:conclusion} concludes the paper and discusses future works.
