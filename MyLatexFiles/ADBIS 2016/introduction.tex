Current data integration implies consuming data from different data services and integrating the results according to different quality requirements related to data cost, provenance, privacy, reliability, availability, among others. Data services and data processing services can take advantage from the on-demand and \textit{pay-as-you-go} model imposed by the cloud architecture. The quality conditions and penalties in case of its violation under which these services are delivered are defined in contracts called Service Level Agreements (SLA).

Cloud services (data services, data processing services, for instance) and the cloud provider export their SLA specifying the level of services the user can expect from them. A user willing to integrate data establishes a contract with the cloud provider guided by an economic model that defines the services she can access, the conditions in which they can be accessed (duplication, geographical location) and their associated cost. Thus, for a given requirement, a user could decide which cloud services (from one or several cloud providers) to use for retrieving, processing and integrating data according to the type of contracts she can establish with them.

In this context, data integration deals with a matching problem of the user's integration preferences which includes quality constraints and data requirements, and her specific cloud subscription with the SLA's provided by cloud services. Matching SLAs can imply dealing with heterogeneous SLA specifications and SLA-preferences incompatibilities. Moreover, even with possibility of having an unlimited access to cloud resources, the user is limited to the resources and the budget agreed by her cloud subscription. The aim of this paper is tackle these problems and design a data integration solution guided by SLAs in a multi-cloud context. In addition, we present our service-based query rewriting algorithm guided by user preferences and  SLAs as proof of concept.

This paper is organized as follows. 
Section~\ref{sec:relatedwork} discusses related works.
Section~\ref{sec:scenario} describes the scenario and some challenges.
Section~\ref{sec:disla} introduces our data integration approach. 
Section~\ref{sec:rhone} describes the \textit{Rhone} algorithm and its formalization. Experiments and results are described in the section~\ref{sec:experiments}. 
Finally, section~\ref{sec:conclusion} concludes the paper and discusses future works.

%\textit{second challenge} is to guide data integration taking into
%consideration the integrated SLA. Here, the data integration process includes (i) looking up services that can be used as data providers, and for services required to process retrieved data and build an integrated result; (ii) performing data retrieval, processing and integration and (iii) deliver results to the user considering her preferences (quality requirements, context and resources consumption). The integrated SLA can guide services filtering in the look up phase; it can help to control the amounts of data to retrieve and process according to consumption rights depending on the
%user subscription to the participating cloud providers and how to deliver data considering the user's context.

%Current data integration implies consuming data from different data services and integrating the results according to users' quality requirements. These requirements are associated to the data that is retrieved and integrated, but also to the properties of the data, its producers and the conditions in which such data is produced and processed. For example, whether the user accepts to pay for data, its provenance, and how much is the user ready to pay for the resources necessary for integrating her expected result. 

%Data provision and data processing services may need a considerable amount of storage, memory and computing capacity. To avoid the lack of resources and the high price paid to use them, these services can be deployed and delivered by cloud architectures. Some approaches express the data integration as a service composition problem in which given a query the objective is to lookup and compose data services that can contribute to produce a result.
%Furthermore, users could need to integrate the data provisioned by services in a homogeneous and general result. The integration process can also require important computing resources that can be obtained from the cloud too. Data provision and processing services can be deployed in the cloud. 
%Finding the best service composition that can answer a query can be computationally costly. Furthermore, executing the composition can lead to retrieve and process data collections that can require important memory, storage and computing resources.

%The service level agreement (SLA) specifies the quality conditions under which a service is delivered. It also includes information about the cloud services that it requires to execute requests. The cloud, itself exports a general SLA that specifies the conditions in which users can access the services (infrastructure, platform and software) deployed in it. A user willing to use the cloud services establishes a contract with the cloud provider guided by an economic model that defines the services she can access, the conditions in which they can be accessed (duplication, geographical location) and their associated cost. Thus, for a given requirement, a user could decide which cloud services (from one or several cloud providers) to use for retrieving, processing and integrating data according to the type of contracts she can establish with them.

%In this context, the \textit{first challenge} is to compute what we call an integrated SLA that matches the user's integration preferences (including quality constraints and data requirements) with the SLA's provided by cloud services, given a specific user cloud subscription. Computing an integrated SLA can imply dealing with heterogeneous SLA specifications and SLA-preferences incompatibilities. The \textit{second challenge} is to guide data integration taking into
%consideration the integrated SLA. Here, the data integration process includes (i) looking up services that can be used as data providers, and for services required to process retrieved data and build an integrated result; (ii) performing data retrieval, processing and integration and (iii) deliver results to the user considering her preferences (quality requirements, context and resources consumption). The integrated SLA can guide services filtering in the look up phase; it can help to control the amounts of data to retrieve and process according to consumption rights depending on the
%user subscription to the participating cloud providers and how to deliver data considering the user's context.          
%
%The objective of this research is design a data integration solution guided by SLAs in a multi-cloud context. In this paper, we focus on presenting our service-based query rewriting algorithm guided by user preferences and services' quality aspects extracted from SLAs. Our work addresses these issues and proposes the algorithm (we called \textit{Rhone}) with two original aspects: (i) the user can express her quality preferences and associate them to her queries; and (ii) service's quality aspects defined on SLAs guide service selection and the whole rewriting process.
%
%This paper is organized as follows. Section ~\ref{sec:disla} discusses about data integration and service level agreements (SLA). Section~\ref{sec:scenario} contains the running scenario and challenges. Section~\ref{sec:relatedwork} presents some related works. Section~\ref{sec:rhone} describes the \textit{Rhone} algorithm and its formalization. Experiments and results are described in the section~\ref{sec:experiments}. Finally, section~\ref{sec:conclusion} concludes the paper and discusses future works.

%The service level agreement (SLA) specifies the quality conditions under which a service is delivered. It also includes information about the cloud services that it requires to execute requests. The cloud, itself exports a general SLA that specifies the conditions in which users can access the services (infrastructure, platform and software) deployed in it. A user willing to use the cloud services establishes a contract with the cloud provider guided by an economic model that defines the services she can access, the conditions in which they can be accessed (duplication, geographical location) and their associated cost. %Different cloud providers have different possible contracts to establish with users (i.e., platinum, silver, gold, ivory users). 
%Thus, for a given requirement, a user could decide which cloud services (from one or several cloud providers) to use for retrieving, processing and integrating data according to the type of contracts she can establish with them.

%Integrating data across different databases, and provide an unique view of it to the user, is a problem in the database domain (called data integration).
%Data integration problem has been studied by many researchers in the database domain. It consists in merging data which is distributed across different data sources, and providing a unified view of it to the user.
%This can be seen in the service-oriented architecture as a service composition issue, such as given a query, the objective is to lookup for data services and compose them in order to achieve a desired result. 
% 
%Finding the best service composition to answer a query can be computationally costly. 	
%However, it is computationally costly to find the best service composition to answer a query. 
%Furthermore, executing the composition can lead to retrieve and process data collections that can require important memory, storage and computing resources.
%The on-demand resource provision imposed by the cloud computing architecture opens challenges to data processing and management.
%The cloud architecture allows and opens challenges to the data integration once such architecture give us an unlimited access to resources. 
%The possibility of having an unlimited access to resources, the resource management, the geographically distributed location of services, and the economic model imposed by the cloud architecture open challenges to data integration solutions.
%Data integration has evolved with the emergence of data services that deliver data under different quality conditions related to data freshness, cost, reliability, availability, among others.  
%Current data integration implies consuming data from different data services and integrating the results according to users' quality requirements. These requirements are associated to the data that is retrieved and integrated, but also to the properties of the data, its producers and the conditions in which such data is produced and processed. For example, whether the user accepts to pay for data, its provenance, and how much is the user ready to pay for the resources necessary for integrating her expected result. 
%(for example, whether the user accepts to pay for data, its provenance, veracity and freshness and how much is the user ready to pay for the resources necessary for integrating her expected result). 

%Data services provide data according to specific APIs that specify method headers with input parameters describing the data to be retrieved and the type of results they can produce. Moreover data provision can be done by services according to different data quality measures. Such measures describe the conditions in which a service can provide or process data. These measures can be expressed in a service level agreement (SLA). An SLA states, what the user can expect from a service or system behavior. For example, whether it implements an authentication process, if it respects data consumer's privacy and the quality of the data the service can deliver, like freshness, veracity, reputation and other non-functional conditions like the business model that controls data delivery.   

% Data are produced continuously and on demand in huge
% quantities and sometimes with few associated meta-data, which makes the
% integration process more challenging. Some approaches express data integration
% as a service composition problem where given a query the objective is to lookup
% and compose data services that can contribute to produce a result. Finding the
% best service composition that can answer a query can be computationally costly.
% Furthermore,  executing the composition can lead to retrieve and process data
% collections that can require important memory, storage and computing resources.     

%This problem has been addressed in the service-oriented
%domain~\cite{Barhamgi2010,Benouaret2011,ba2014}.
%Generally, these solutions deal with query rewriting problems.
%\cite{Barhamgi2010} proposed a query rewriting approach which processes queries
%on data provider services. \cite{Benouaret2011} introduced a service composition
%framework to answer preference queries. In that approach, two algorithms based
%on~\cite{Barhamgi2010} are presented to rank the best rewritings based on previously computed scores.
%\cite{ba2014} presented an algorithm that produces and order rewritings
%according to user preferences. Yet, to our knowledge few works consider quality
%measures associated both to data services and to user preferences in order to
%guide the rewriting process. 

%Data provision and data processing services may need a considerable amount of storage, memory and computing capacity that can be provided by cloud architectures. Furthermore, users could need to integrate the data provisioned by services in a homogeneous and general result. The integration process can also require important computing resources that can be obtained from the cloud too. Data provision and processing services can be deployed in the cloud. Their SLA includes the measures about the cloud services that they require to execute their requests. The cloud, itself exports a general SLA that specifies the conditions in which users can access the services (infrastructure, platform and software) deployed in it. A user willing to use the cloud services establishes a contract with the cloud provider guided by an economic model that defines the services she can access, the conditions in which they can be accessed (duplication, geographical location) and their associated cost. Different cloud providers have different possible contracts to establish with users (i.e., platinum, silver, gold, ivory users). Thus, for a given requirement, a user could decide which cloud services (from one or several cloud providers) to use for retrieving, processing and integrating data according to the type of contracts she can establish with them.

%Computing resources are delivered as services by the cloud. Services are billed and agreed between service providers and service customers under service level agreement (SLA) contracts.
%Both sides must agree together on quality conditions and penalties under which the service is delivered. 
%%Generally, those conditions and penalties associated to its violation are defined in service level agreement (SLA). 
%Several researches have reported their studies on SLA in different domains~\cite{AlhamadDC11}.
%SLA proposals for cloud computing could be divided in two groups: (i) works
%developing tools and methods to help on SLA negotiation and enforcement
%phase~\cite{rak2013,Mavrogeorgi2013}; and (ii) approaches that monitor contracts
%and cloud resources in order to detect and avoid SLA
%violations~\cite{Leitner2010,Maarouf2015}. To our knowledge, SLA publications
%have not been yet integrated to data integration in a multi-cloud environment.

%Thus, the \textit{first challenge} is to compute what we call an integrated SLA
%that matches the user's integration preferences (including quality constraints
%and data requirements) with the SLA's provided by cloud services, given a
%specific user cloud subscription. Computing an integrated SLA can imply dealing
%with heterogeneous SLA specifications and SLA-preferences incompatibilities.
%The \textit{second challenge} is to guide data integration taking into
%consideration the integrated SLA. Here, the data integration process includes (i) looking up services that can be used as data providers, and for services required to process retrieved data and build an
%integrated result; (ii) performing data retrieval, processing and integration
%and (iii) deliver results to the user considering her preferences (quality
%requirements, context and resources consumption). The integrated SLA can guide
%services filtering in the look up phase; it can help to control the amounts of
%data to retrieve and process according to consumption rights depending on the
%user subscription to the participating cloud providers and how to deliver data
%considering the user's context.          


%Based on the open challenges, the long-term objective of this research is design a data integration solution guided by SLAs in a multi-cloud context. In this paper, we focus on presenting our service-based query rewriting algorithm guided by user preferences and services' quality aspects extracted from SLAs. Our work addresses these issues and proposes the algorithm (we called \textit{Rhone}) with two original aspects: (i) the user can express her
%quality preferences and associate them to her queries; and (ii)  service's quality aspects defined on SLAs guide service selection and the whole rewriting process.
%Yet, to the best of our knowledge, we have not identified any other work that combines SLA and data integration solution in the (multi-)cloud context.

%Based on these concepts and open challenges, the goal of this work is to present
%a data integration solution concerning a service-based query rewriting algorithm guided by SLA's.
%Our work addresses these issues and proposes the algorithm (we
%called \textit{Rhone}) with two original aspects: (i) the user can express her
%quality preferences and associate them to her queries; and (ii)  service's quality aspects defined on Service Level Agreements (SLA) guide service selection and the whole rewriting process.
%Yet, to the best of our knowledge, we have not identified any other work that uses SLA to guide the entire data integration solution.

%This paper is organized as follows. Section ~\ref{sec:disla} discusses
%about data integration and service level agreements (SLA).
%Section~\ref{sec:scenario} contains the running scenario and challenges.
%Section~\ref{sec:relatedwork} presents some related works.
%Section~\ref{sec:rhone} describes the \textit{Rhone} algorithm and its
%formalization.
%Experiments and results are described in the section~\ref{sec:experiments}. 
%Finally, section~\ref{sec:conclusion} concludes the paper and discusses future works.
