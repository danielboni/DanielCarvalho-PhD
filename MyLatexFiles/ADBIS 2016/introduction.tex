Integrating data across different databases and providing a unique view of it to the user is a well-known problem in the database domain.
%Data integration is a well-known problem in the database domain.
This problem can be seen in the cloud context as integration of data services.
%The cloud architecture allows and opens challenges to the data integration once such architecture give us an unlimited access to resources. 
The possibility of having an unlimited access to cloud resources, the resource management, the geographically distributed location of services, and the economic model imposed by the cloud architecture open challenges to data integration solutions.
In this environment, there are two important roles: the service provider
and the service customer. Both agreed together on quality aspects expected
from the other side. Generally, those aspects are defined in service level 
agreement contracts. Works on this context goes in two ways: (1) controls/orchestrate
the negotiation phase (REFs??); and (2) approaches that manage agreements to avoid SLA violation (REFs??).
Taking into account the aforementioned, our research interest is on a data integration approach in a 
multi-cloud context completely guided by SLAs. To achieve this, we have developed a query rewriting 
algorithm for integrating data services which takes into account user preferences and SLAs to guide the
service selection and rewriting process. Yet, to the best of our knowledge, we have not identified any other
work that uses SLA to guide the entire data integration solution. So, the aim of this paper is to present part of work considering the description and formalization of the Rhone service-based query rewriting algorithm which considers user preferences and services' quality aspects defined on SLAs. Additionally, experiments and results achieved are detailed as a proof of concept.

The rest of this paper is organized as follows. 
Section ref? describes our related works. 
Section ref? contains the running scenario and challenges.
Section ref? describes the Rhone and its formalization. 
Experiments and results are described in the section ref?. 
Conclusion and future works come in the section ref?.