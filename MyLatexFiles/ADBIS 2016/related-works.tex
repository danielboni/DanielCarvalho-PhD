The main aspect in a data integration solution is 
the query rewriting process executed by a mediator
in accordance with the different databases.
%
In this way, algorithms for rewriting queries have
been proposed in two domains: (i) on the database 
domain; and (ii) on the service-oriented domain.

On the database domain, query rewriting approaches using
views have been widely discussed~\cite{Halevy:2001}.
%
For instance, the \textit{bucket algorithm}~\cite{Levy:1996}, 
\textit{inverse-rules algorithm}~\cite{Duschka:1997} and 
\textit{MiniCon algorithm}~\cite{Pottinger:2001} have
tackled the rewriting problem on the database domain.
%
In addition, \cite{Pottinger:2001} has inspired rewriting methods in service-oriented domain~\cite{costa2013,ba2014}.
%In addition, these algorithms have also inspired other algorithms in the database and service-oriented domains♣~\cite{costa2013} (REF THE WORKS).
%These approaches will not be detailed here once the focus
%of this paper is on algorithms in the service-oriented domain.

%Generally, data integration solutions on the service-oriented domain deal with query rewriting problems.
Data integration solutions on the 
service-oriented domain deal with query rewriting problems. \cite{Carvalho2015} identifies
trends and open issues regarding the use of SLA in data integration solutions on
multi-cloud environments.
%
\cite{Barhamgi2010} proposes a query rewriting approach 
which processes queries on data provider services.
The query and data services are modeled as RDF views.
A rewriting answer is a service composition in which 
the set of data service graphs fully satisfy the query graph.  
%
\cite{Benouaret2011} introduces a service composition
framework to answer preference queries. In that approach, two algorithms based
on~\cite{Barhamgi2010} are presented to rank the best rewritings based on previously computed scores.
%
\cite{ba2014} presents an algorithm based on \textit{MiniCon} 
that produces and order rewritings according to user preferences. 
The user preferences on this approach are scores used to rank 
services that should be previously define by the user. 
%
Considering the scenario presented in the previous section and the related works
discussed in this one, we identify a gap between the application of quality
measures in the context of data integration domain. Thus, the main
and original proposal of our work is to use SLA to guide the entire data integration process.

Our approach differs from these works in three aspects:
(i) the user can express quality measures and associate them
to his queries, such as: \textit{I want to use services with response
time less than 2 seconds, price per request less than 1 dollar
and location close to my city}; 
(ii) the user preferences guides the service selection. 
These preferences are matched with the services' quality aspects
that are extracted from service level agreement contracts.
Here, it is important to highlight that there
is a previous phase in which the services' quality aspects are 
processed and extracted from SLAs. 
In our proposal we are assuming that these information are accessible and
well-formatted to the algorithm; and
(iii) the user preferences are also used to guide the rewriting process.
The rewriting answers (services compositions) produced must be in 
accordance with the user preferences.
%

% Yet, to the best of our knowledge, few works consider quality
% measures associated both to data services and to user preferences in order to
% guide the rewriting process, and there is no approach using SLAs to guide the
% data integration solution. 

Here, it is important to highlight that this paper focus on the description and
evaluation of the algorithm that rewrites queries in terms of services
composition taking into account user preferences and service quality aspects
expressed in SLA contracts. We are assuming that the extraction of quality
aspects from SLAs is performed in a previous phase of our global data
integration solution. 
In the next section, the Rhone service-based algorithm is described and
formalized.
