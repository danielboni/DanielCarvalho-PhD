In recent years, the cloud have been the most popular deployment environment for data integration~\cite{Carvalho2015}. Researches have proposed their works addressing this issue~\cite{ElSheikh2013,Tian2010}.
Moreover, once query rewriting is strictly related to data integration, rewriting algorithms have been presented~\cite{ba2014,Barhamgi2010,Benouaret2011}.

In~\cite{ElSheikh2013}, the authors introduced a system (called SODIM) which combines data
integration, service-oriented architecture and distributed processing.
The novelty of their approach is that they perform data integration 
in service oriented contexts, particularly considering data services. 
A major concern when integrating data from different sources (services) is privacy that can 
be associated to the conditions in which integrated data collections are built and shared. 
\cite{Tian2010} proposed an inter-cloud data integration system that considers a trade-off
between users' privacy requirements and the cost for protecting and processing data. 
According to the users' privacy requirements, the query plan in the cloud repository 
creates the users' query. 
Thus, the query plan executor decides the best location to execute the sub-query to 
meet privacy and cost constraints.
This work is mostly interested in privacy and performance issues forgetting other users' 
integration requirements.

The main aspect in a data integration solution is the query rewriting. 
In the database domain, the query rewriting problem using views have been widely discussed~\cite{Halevy:2001}.
Similarly, data integration can be seen in the service-oriented domain as a service composition problem in which given a query the objective is to lookup and compose data services that can contribute to produce a result.
Generally, data integration solutions on the service-oriented domain deal with
query rewriting problems. 
\cite{Barhamgi2010} proposed a query rewriting approach which processes queries on data provider services.
\cite{Benouaret2011} introduced a service composition framework to answer
preference queries. Two algorithms are presented to rank the best rewritings based on previously computed scores.
\cite{ba2014} presented an refinement algorithm based
on \textit{MiniCon} that produces and order rewritings according to user preferences and scores used to rank services that should be previously define by the user.
Furthermore, they do not take into consideration user's integration requirements which can lead to produce rewritings that are not satisfactory in terms of quality requirements and constraints imposed by the user and the cloud environment. 
We assume that these requirements and constraints be expressed on SLAs. 
In the next section, we introduce our query rewriting algorithm that deals with SLAs while selecting, filtering and producing results.