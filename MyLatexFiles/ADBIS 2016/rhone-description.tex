This section describes our quality-based query rewriting algorithm called \textit{Rhone}. It is guided by user preferences and services' quality aspects extracted from service level agreements (SLA). Our algorithm has two original aspects: \textit{first}, the user can express his/her quality preferences and associate them to his/her queries; and, \textit{second},  service's quality aspects defined on Service Level Agreements (SLA) guide service selection and  the whole rewriting process. %The \textit{Rhone} assumes a previous step in our data integration approach in which the services' definition and their associated quality measures are extracted from their respective SLAs.
%

\subsection{Preliminaries}

The idea behind our algorithm consists in deriving a set of service compositions that fulfill the users' integration preferences concerning the context of data service deployment given a set of \textit{abstract services}, a set of \textit{concrete services}, a user' \textit{query} defined hereafter and a set of user' \textit{integration preferences}.
%
%
%The \textit{Rhone} includes four macro-steps: 
%(\textit{i}) select services; 
%(\textit{ii}) create variable mappings from services to the query; and 
%(\textit{iii}) combine the services and 
%(\textit{iv}) produce rewriting that matches with the query. 
%% In the following lines, basic definitions are presented to explain and describe each step of the 
%% algorithm in detail. 
%\bigskip