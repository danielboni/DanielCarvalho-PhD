
%
The input for the \textit{Rhone} algorithm is: (1) a query; (2) a list of concrete services.

\noindent \textbf{Definition 1 (Query):}
A query $Q$ is defined as a set of \textit{abstract services}, a set of \textit{constraints}, and a set of \textit{user preferences} in accordance with the grammar: 

\begin{center}
\small
\begin{math}
Q (\overline{I}_{h}; \overline{O}_{h}) := A_{1}(\overline{I}_{1l}; \overline{O}_{1l}), A_{2}(\overline{I}_{2l}; \overline{O}_{2l}), ..,  A_{n}(\overline{I}_{nl}; \overline{O}_{nl}),C_{1},C_{2}, .., C_{m}[P_{1},P_{2}, .., P_{k}]
\end{math}
\end{center}
%
The left-hand of the definition is called the \textit{head} of the query; and the right-hand is called the \textit{body}. 
%
$\overline{I}$ and $\overline{O}$ are a set of comma-separated \textit{input} and \textit{output} parameters, respectively.
%
There are two types of parameters: the ones who appears in the \textit{head} definition called \textit{head} variables, and the ones who appears only in the \textit{body} definition called \textit{local} variables.
%
The sets $\overline{I}_{h}$ and $\overline{O}_{h}$ refer to \textit{head} input and output variables, 
and the sets $\overline{I}_{l}$ and $\overline{O}_{l}$ refer to \textit{local} input and output variables.
Intuitively, $\overline{I}$ is the union of $\overline{I}_{h}$ and all $\overline{I}_{l}$ such as  
$\overline{I}$ =  $\overline{I}_{h} \cup \lbrace\overline{I}_{1l},..,\overline{I}_{nl}\rbrace$. 
Also, the intersection $\overline{I}_{h} \cap \overline{I}_{1l} \cap \overline{I}_{2l},.., \cap \overline{I}_{nl}$ is not empty which means that: (1) there are \textit{head} variable that are used on the \textit{abstract services} in the body definition; and (2) there are \textit{local variables} that are shared among the \textit{abstract services}. 
The same rule can be applied to output variables: $\overline{O}$ =  $\overline{O}_{h} \cup \lbrace\overline{O}_{1l},..,\overline{O}_{nl}\rbrace$, and the intersection among $\overline{O}_{h}$ and all $\overline{O}_{l}$ is not empty.
% 
\textit{Abstract services} ($A_{1}, A_{2}, .., A_{n}$) describes a set of basic service capabilities.
%
$C_{1}, C_{2}, .., C_{m}$ are \textit{constraints} over the \textit{input} and/or \textit{output} parameters. These constraints are used while querying the databases. 
The \textit{user preferences} (over the services) are specified in $P_{1}, P_{2}, .., P_{k}$.  
%
$C_{i}$ and $P_{j}$ are in the form $x \otimes c$, where $x$ is a identifier; $c$ is a constant; and
$\otimes \in\lbrace \geq, \leq, =, \neq, <, >\rbrace$.

\noindent \textbf{Definition 2 (Concrete service):} A concrete service ($S$) is defined as a set of 
\textit{abstract services}, and by its \textit{quality measures} according to the grammar:

\begin{center}

\begin{math}
S (\overline{I}_{h}; \overline{O}_{h}) := A_{1}(\overline{I}_{1l}; \overline{O}_{1l}), A_{2}(\overline{I}_{2l}; \overline{O}_{2l}), ..,  A_{f}(\overline{I}_{fl}; \overline{O}_{fl})[M_{1},M_{2}, ..,M_{g}]
\end{math}

\end{center} 

A \textit{concrete service} definition is similar to the \textit{query} definition, excepting the fact that a \textit{concrete service} does not have constraints over input and output variables.
The input variables $\overline{I}$ is the union of $\overline{I}_{h}$ (\textit{head} variables) and all $\overline{I}_{l}$ (\textit{local} variables) such as  
$\overline{I}$ =  $\overline{I}_{h} \cup \lbrace\overline{I}_{1l},..,\overline{I}_{nl}\rbrace$, and
the intersection $\overline{I}_{h} \cap \overline{I}_{1l} \cap \overline{I}_{2l},.., \cap \overline{I}_{fl}$ is not empty, meaning that there are \textit{head} and \textit{local }variables being shared among the \textit{abstract services}. 
The same rules are applied to output variables $\overline{O}$ = $\overline{O}_{h} \cup \lbrace\overline{O}_{1l},..,\overline{O}_{nl}\rbrace$, and $\overline{O}_{h} \cap \overline{O}_{1l} \cap \overline{O}_{2l},.., \cap  \overline{O}_{fl}$ is not empty.
%
$M_{1},M_{2}, .., M_{g}$ are \textit{quality measures} associated to the concrete service. 
These \textit{measures} reflect the quality aspects present in the service level agreement exported by the concrete service. 
%
In this algorithm, we are assuming that there is a previous phase in our approach which allow us to extract the \textit{quality measures} associated to concrete services from SLAs, and generate the expected input data according to the grammar.
%
$M_{i}$ is in the form $x \otimes c$, where $x$ is a special class of identifiers associated to the services; $c$ is a constant; and $\otimes \in\lbrace \geq, \leq, =, \neq, <, >\rbrace$.
 
\bigskip
\noindent \textit{Example 1 (query and concrete service):} To illustrate the definition above let us suppose an example based on the scenario. 
\textit{A user wants to retrieve patients DNA information that were infected by the disease ``flu'', using services with availability higher than 98\%, price per call less than 0.2\$ and total cost less than 2\$.}
The query can be expressed using the grammar as follows. The decorations $?$ and $!$ differentiate input from output parameters, respectively. 
\begin{center}
$Q(d?; dna!) := diseasePatients(d?; p!), GetDNA(p?; dna!),$
\\
$d = ``flu'' [availability > 98\%, \ price \ per \ call < 0.2\$, \ total \ cost < 2\$]$
\end{center}
The query $Q$ has an input parameter $d$ and an output parameter $dna$. 
These \textit{head} variable are used in the \textit{abstract services} $diseasePatients$ and $DNAinformation$. 
The local variable $p$ is an output in $diseasePatients$ and it is used as input in $GetDNA$.
The constraint $d = ``flu''$ will be used while querying the databases in the \textit{where clause}.
Between brackets there are \textit{user preferences}: $availability$, $price \ per \ call$ and $total \ cost$.
Five concrete services are exemplified below. 
\begin{flushleft}
\small
$S1(d?; p!) := diseasePatients(d?; p!)[availability > 99\%, \ price \ per \ call = 0.1\$]$ 
\\
$S2(d?; p!) := diseasePatients(d?; p!)[availability > 97\%, \ price \ per \ call = 0.2\$]$
\\
$S3(p?; dna!) := GetDNA(d?; dna!)[availability > 98\%, \ price \ per \ call = 0.1\$]$
\\
$S4(p?; dna!) := DNA(d?; dna!)[availability > 98\%, \ price \ per \ call = 0.1\$]$
\\
$S5(d?; dna!) := diseasePatients(d?; p!), GetDNA(p?; dna!)[availability > 98\%, \ price \ per \ call = 0.1\$]$
\end{flushleft}
$S1$, $S2$, $S3$, $S4$ and $S5$ are five different concrete services defined in terms of the abstract services described before in our scenario (see section \ref{scenario}). 
Each concrete service is tagged with its \textit{quality measures} between the brackets. 
Here, it is important to highlight that the \textit{quality measures} are extract from service level agreement in a previous phase of our approach that is not the focus in this paper.

\bigskip
The main function of the Rhone is described in the algorithm \ref{algo-rhone}. 
The input data for this function is a query, which includes a set of user preferences, and a set of concrete services. The result is a set of rewriting of the query in terms of concrete services, fulfilling the user preferences.

\begin{algorithm}
\small
\caption{ - RHONE}
\label{algo-rhone}
\begin{algorithmic}[1]
\REQUIRE A query $Q$, a set of user preferences, and a set of concrete services $\bigS$.
\ENSURE A set of rewritings $R$ that matches with the query and fulfill the user preferences.
\STATE \textbf{function} $\mathit{rhone} (Q, \bigS)$
 \STATE  $\bigLS \leftarrow \mathit{SelectCandidateServices}(Q, \bigS)$ \label{rhone:buildPCD}
 \STATE  $\bigLCSD \leftarrow CreateCSDs(Q, \bigLS)$
 \STATE  $I \leftarrow CombineCSDs(\bigLCSD)$
 \STATE $R\leftarrow \emptyset$
% \STATE ~\!\tqI{\agg{Q}} 
    \STATE $p \leftarrow I.next()$
    \WHILE {$p\ \neq\ \emptyset$ \AND ~\!\tqI{\agg{Q}}} 
      \IF {\textit{isRewriting}$(Q, p)$}
  \STATE $R\leftarrow R\,\cup \mathit{Rewriting}(p)$
  \STATE ~\!\tqS{\agg{Q}}
   \ENDIF
      \STATE $p \leftarrow I.\mathit{Next}()$
 \ENDWHILE
    \STATE \textbf{return} $R$
\STATE \textbf{end function}
\end{algorithmic}
\end{algorithm}

In the first step, the algorithm looks for concrete services that 
can be matched with the query (line 2), resulting in a set of candidate concrete
services. For this set of services, the Rhone tries to create 
\textit{concrete services description} (CSD) for each service (line 3). 
A CSD is a structure that maps a concrete service to the query. 
The  result of this step is a list of CSDs.
Given all produced CSDs  (line 4), they are combined among each other to generate a list of lists of CSDs, each element representing a possible rewriting.
The final step (lines 5-13) identifies which lists of CSDs are a valid
rewriting of the user query given the list of lists of CSDs.
A combination of CSDs is a valid rewriting if: (i) they cover all abstract services in the query; and 
(ii) there is mapping to all head variables in the query (implemented by the function \textit{isRewriting}$(Q, p)$ - line 8).
The originality of our algorithm concerns the aggregation function (\agg{Q}).
It is responsible to check and increment \textit{composed measures} (if present in the query). 
This means for each element in the CSD list the value of \textit{composed measure} is incremented (line 10), and rewritings are produced while the values of these measures are respected (line 7). 
The result of this step is the list of valid rewriting of the query (line 14). 
In the next sections, each phase of the algorithm is described in detail. 

\subsection{Selecting services}

While selecting services, the algorithm deals with three matching problems: \textit{measures} matching, \textit{abstract service} matching and \textit{concrete service} matching.

\noindent \textbf{Definition 3 (\textit{measures} matching):} 
Given a \textit{user preference} $P_{i}$ and a quality measure $Q_{j}$, a matching between them can be made if:
(\textit{i}) the identifier $c_{i}$ in $P_{i}$ has the same name of $c_{j}$ in $Q_{j}$; and
(\textit{ii}) the evaluation of $Q_{j}$, denoted $eval(Q_{j})$, must satisfy the evaluation of $P_{i}$ ($eval(P_{i})$). In other words, $eval(Q_{j}) \subset eval(P_{i})$.

\noindent \textbf{Definition 4 (\textit{abstract service} matching):} 
Given two abstract services $A_{i}$ and $A_{j}$, a match between \textit{abstract services} occurs when an \textit{abstract service} $A_{i}$ can be matched to $A_{j}$, denoted $A_{i} \equiv A_{j}$, according to the following conditions: 
(\textit{i}) $A_{i}$ and $A_{j}$ must have the same abstract function name; 
(\textit{ii}) the number of input variables of $A_{i}$, denoted $vars_{input}(A_{i})$, is equal or higher than the number of input variables of $A_{j}$ ($vars_{input}(A_{j})$); and 
(\textit{iii}) the number of output variables of $A_{i}$, denoted $vars_{output}(A_{i})$, is equal or higher than the number of output variables of $A_{j}$ ($vars_{output}(A_{j})$).

\noindent \textbf{Definition 5 (\textit{Concrete service} matching):} 
A \textit{concrete service} $S$ can be matched with the \textit{query} $Q$ according to the following conditions:
(\textit{i}) $\forall A_{i}  \ s. \ t. \lbrace\ A_{i} \in \ S\rbrace, \ \exists \ A_{j} \ $ $s. \ t. \lbrace\ A_{j} \in \ Q\rbrace, \ where \ A_{i} \equiv A_{j}.$ For all \textit{abstract services} $A_{i}$ in $S$, there is one \textit{abstract service} $A_{j}$ in $Q$ that satisfies the \textit{abstract service} matching problem (Definition 4); and
(\textit{ii}) . For all \textit{single measure} $P_{i}$ in $Q$, there is one \textit{single measure} $Q_{i}$ in $S$ that satisfies the \textit{measures} matching problem (Definition 3).

The process of selecting candidate concrete services
is described in the algorithm~\ref{selectingservices}.
Given the query and a set of concrete services, the algorithm
looks for concrete services that can be used in the rewriting process.
While iterating all concrete services in the list $\bigS$ (line 3), firstly,
each service is checked to analyze if all its quality measures satisfies the user preferences
in $Q$ (line 4). If it satisfies, each abstract service in $S_{i}$ is checked to confirm if 
it matches or not with the query (lines 6-11). Once the service satisfies all the matching 
problems, a set of candidate concrete services is produced (line 12-13). The result
is a list of \textit{candidate concrete services} $\bigLS$ which
probably can be used in the rewriting process (line 17).

\begin{algorithm}
%\small
\caption{ - Select candidate services}
\label{selectingservices}
\begin{algorithmic}[1]
\REQUIRE A query $Q$ and a set of concrete services $\bigS$.
\ENSURE A set of candidate concrete services $\bigLS$ that can be used in the rewriting process and fulfill the user preferences.
\STATE \textbf{function} $\mathit{SelectCandidateServices} (Q, \bigS)$
\STATE $\bigLS \leftarrow \emptyset$
\FORALL  {$S_{i}$ in $\bigS$}
	\IF {$\mathit{SatisfyQualityMeasures(Q, S_{i})}$}
		\STATE $b \leftarrow \mathit{true}$		
		\FORALL  {$A_{j}$ in $S_{i}$}
			\IF {$Q.\mathit{notContains(A_{i})}$}
				\STATE $b \leftarrow \mathit{false}$	
				\STATE $\mathit{break}$
			\ENDIF
		\ENDFOR
		\IF {$b = true$}
			\STATE $\bigLS \leftarrow \bigLS \cup \lbrace S_{i} \rbrace$	
		\ENDIF
	\ENDIF
\ENDFOR
\STATE \textbf{return} $\bigLS$
\STATE \textbf{end function}
\end{algorithmic}
\end{algorithm}

\bigskip
\noindent \textit{Example 2 (selecting candidate concrete services):} let us use as example, the query and concrete services presented in the example 1. The Rhone algorithm will iterate in the concrete service list,
looking for the ones which satisfy the matching problems. 
$S1$ is a candidate concrete service once it satisfies all measures problems.
$S2$ is not select once it measures violate the user preferences.
$S3$ is selected once it satisfies all measures problems.
$S4$ is not select once it contains an abstract service that cannot be matched with the query.
$S5$ is a candidate concrete service once it satisfies all measures problems. 
Summarizing, $S1$, $S3$ and $S5$ are the services selected while applying the matching rules.  
They will be included to the set of candidate concrete services $\bigLS$. 

%\begin{algorithm}
%%\small
%\caption{ - Satisfy quality measures}
%\label{satisfymeasures}
%\begin{algorithmic}[1]
%\REQUIRE A query $Q$ and a concrete services $S_{i}$.
%\ENSURE A boolean value. \textit{True}, if the service satisfies the user preferences. \textit{False}, otherwise.
%\STATE \textbf{function} $\mathit{SatisfyQualityMeasures (Q, S_{i})}$
%\STATE to do
%%\STATE $\bigLS \leftarrow \emptyset$
%%\FORALL  {$S_{i}$ in $\bigS$}
%%	\IF {$\mathit{SatisfyQualityMeasures(Q, S_{i})}$}
%%		\STATE $b \leftarrow \mathit{true}$		
%%		\FORALL  {$A_{j}$ in $S_{i}$}
%%			\IF {$Q.\mathit{notContains(A_{i})}$}
%%				\STATE $b \leftarrow \mathit{false}$	
%%				\STATE $\mathit{break}$
%%			\ENDIF
%%		\ENDFOR
%%		\IF {$b = true$}
%%			\STATE $\bigLS \leftarrow \bigLS \cup \lbrace S_{i} \rbrace$	
%%		\ENDIF
%%	\ENDIF
%%\ENDFOR
%\STATE \textbf{return} $true$
%\STATE \textbf{end function}
%\end{algorithmic}
%\end{algorithm}

\subsection{Candidate service description}

After producing the set of candidate concrete services, the next step tries 
to create candidate service descriptions (CSDs). 
A CSD maps abstract services and variables of a concrete service into abstract 
services and variables of the query. 

\noindent \textbf{Definition 6 (candidate service description):} A CSD is represented by an n-tuple:
\begin{center}
$\langle S, h, \varphi, G, P\rangle$
\end{center}
where $S$ is a \textit{concrete service}. 
\textit{h} are mappings between variables in the \textit{head} of $S$ to variables in the \textit{body} of $S$. 
$\varphi$ are mapping between variables in the \textit{concrete service} to variables in the \textit{query}.
$G$ is a set of \textit{abstract services} covered by $S$. 
$P$ is a set \textit{quality measures} associated to the service $S$. 
 
A CSD is created according to 4 rules: (1) for all head variables in a concrete service, the mapping $h$ from the head to the body definition must exist; (2) Head variables in concrete services can be mapped to head or local variables in the query; (3) Local variables in concrete services can be mapped to head variables in the query;
and (4) Local variables in concrete services can be mapped to local
variables in the query if and only if the concrete service covers all abstract services in the query that depend on this variable. The relation ``depends''  means that this an output local variable is used as input in another abstract service. 

The algorithm~\ref{creatingcsds} describes the creation of CSDs. Given the query $Q$ and a list of candidate concrete services $\bigLS$, a list of CSDs $\bigLCSD$ is produced. 
The algorithm iterates on each service in $\bigLS$ (line 3), verifying if the mappings rules are being satisfied (line 4). For the ones which satisfies the mapping rules, a fresh copy of the abstract services in the concrete service is done in $G$(lines 7-9) and a copy of the service quality measures in done in $P$ (lines 10-12). Then,
a CSD is created (line 13), and added to the final list os CSDs $\bigLCSD$ (line 14).
The result of this phase is a list of CSDs that can be used to build rewriting of the query (line 17).

\begin{algorithm}
%\small
\caption{ - Create candidate service descriptions (CSDs)}
\label{creatingcsds}
\begin{algorithmic}[1]
\REQUIRE A query $Q$ and a set of candidate concrete services $\bigLS$.
\ENSURE A set of candidate service descriptions (CSDs) $\bigLCSD$ that contains mappings from candidate concrete service to the query.
\STATE \textbf{function} $\mathit{CreateCSDs} (Q, \bigLS)$
\STATE $\bigLCSD \leftarrow \emptyset$
\FORALL  {$S_{i}$ in $\bigLS$}
	\IF {There are mappings $\mathit{h}$ and $\varphi$ from $S_{i}$ to $Q$}	
		\STATE $G \leftarrow \emptyset$	
		\STATE $P \leftarrow \emptyset$		
		\FORALL  {$A_{j}$ in $S_{i}$}
			\STATE $G \leftarrow G \cup \lbrace A_{j} \rbrace$ 
		\ENDFOR
		\FORALL  {$M_{k}$ in $S_{i}$}
			\STATE $P \leftarrow P \cup \lbrace M_{k} \rbrace$ 
		\ENDFOR
		\STATE $CSD := \langle S_{i}, h, \varphi, G, P \rangle$	
		\STATE $\bigLCSD \leftarrow \bigLCSD \cup \lbrace CSD \rbrace$	
	\ENDIF
\ENDFOR
\STATE \textbf{return} $\bigLCSD$
\STATE \textbf{end function}
\end{algorithmic}
\end{algorithm}

\subsection{Combining candidate service descriptions}

\subsection{Producing rewritings}