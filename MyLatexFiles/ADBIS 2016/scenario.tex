This section is devoted to describe the motivation scenario %and challenges
concerning our data integration approach.
%

Let us assume the following scenario in the medical domain. 
Users are able to retrieve information about (i) patients that were infected by a disease; 
(ii) regions most affected by a disease in Europe; 
(iii) patients' personal information; and 
(iv) patients' dna information.
To perform these actions, four family of services are necessary: family
\textbf{A} has services which given a disease name, it retrieves the list
of infected patients; family \textbf{B} has services which given a disease name,
it retrieves the list of cities most affected by that disease; family \textbf{C}
has services which given a patient id, it retrieves patients' personal
information; and family \textbf{D} has services which given a patient id, it
retrieves patients' dna information. 

Doctor Marcel would like to study the type of people suffering of a
particular disease. For instance, he needs to query the patients' personal
information and patients' dna information from the set of patients that were
infected by flu. Presuming that Marcel has at his disposal a cloud
including a set of services from the families \textbf{A}, \textbf{B}, \textbf{C} and \textbf{D}. To achieve
his needs, Marcel can use the data services as follows: (i) he invokes service
\textbf{S1} (family \textbf{A}) with the disease information then he gets the
set of people infected by flu; (ii) then he invokes service \textbf{S2} (family
\textbf{C}) with the obtained patients in order to retrieve their personal
information; just after (iii) Doctor Marcel invokes service \textbf{S3} (family
\textbf{D}) with the obtained patients to retrieve their dna information.
Finally, the query results is integrated and returned.

Depending on the amount of services in each family type, a lot of other service 
compositions could be done to answer Marcel's query. A large quantity of
algorithms for this purpose have been developed, and all of them share the same
problems: (1) producing rewritings when a big amount of services are available
is extremely expensive; and (2) not always the quality of the rewriting (composition) produced is enough for meeting your needs.
Motivated by these problems, our approach proposes a new vision of data
integration as follows.

Assuming the same medical scenario and the same families of data services.
Let us suppose Marcel would like to study the type of people suffering of a
particular disease as before. However, in this new scenario, he is also capable
to express his preferences while integrating services. For instance, he needs to
query the patients' personal information and patients' dna information from
patients that were infected by flu, using services with availability higher than
98\%, price per call less than 0.2\$ and total cost less than 2\$. Marcel has at his disposal a set of services from the families \textbf{A}, \textbf{B}, \textbf{C} and \textbf{D} geographically disposed on different cloud provides (configuring a multi-cloud environment).
To achieve his needs, Marcel can use the data services as before invoking one
service from the families \textbf{A}, \textbf{C} and \textbf{D} in sequence.
However, in this new configuration all the services involved must satisfy the
user preferences expressed in the query. The selection and rewriting process is
guided by the service level agreements (SLA) exported from different services.
The user preferences are matched with the service quality aspects that are
defined on its SLAs.     

This vision of data integration brings some reflections and questions
considering the challenges presented previously, such as:
(i) The amount of services involved in a multi-cloud context is bigger than in a
single cloud. Consequently, the number of services that can be used in the
rewriting process and the number of rewriting produced is higher. Such
environment calls for a better services selection process, \textit{i.e.}, guided
by the SLAs and user preferences; (ii) How can the different SLAs associated to
services and cloud provider can be integrated with the user preferences? There
are different levels of SLAs: the one agreed between services and cloud
providers; and the ones agreed between services and users. These SLAs should be
integrated; and (iii) How can a previous processed query be reused for a next query?

% Here, it is important to highlight that this paper focus on the description and
% evaluation of the algorithm that rewrites queries in terms of services
% composition taking into account user preferences and service quality aspects
% expressed in SLA contracts. We are assuming that the extraction of quality
% aspects from SLAs is performed in a previous phase of our global data integration solution.
In the next section we are going to discuss the related works concernig data
integration and service level agreements, in order to identify the gap between
the presented challenges and how the works address these problems.
