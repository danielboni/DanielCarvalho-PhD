This section is devoted to describe the motivation scenario and challenges
concerning our data integration approach.
%

Let us assume the following scenario in the medical domain. 
Users are able to retrieve information about (1) patients that were infected by a disease; 
(2) regions in the Europe most infected by a disease; 
(3) patients' personal information; and 
(4) patients' dna information.
To perform these actions, four family of services are necessary: family \textbf{A} services that given a disease name, it retrieves the list of infected patients by the disease; family \textbf{B} services that given a disease name, it retrieves the list of cities most affected by the disease; family \textbf{C} services that given a patient id, it retrieves patients' personal information; and family \textbf{D} services that given a patient id, it retrieves patients' dna information. Assume that the doctor Marcel would like to study the type of people suffering of a particular disease. For instance, he needs to query the patients' personal information and patients' dna information from patients that were infected by flu. Presuming Marcel has at his disposal a cloud including a set of services from the families \textbf{A}, \textbf{B}, \textbf{C} and \textbf{D}. To achieve his needs Marcel can use the data services  as follows: He invokes service \textbf{S1} (family \textbf{A}) with the disease information then he gets the set of people infected by flu; then he invokes service \textbf{S2} (family \textbf{C}) with the obtained patients to retrieve their personal information; Then he invokes service \textbf{S3} (family \textbf{D}) with the obtained patients to retrieve their dna information. Then the results for the query is integrated and returned.

Depending on the amount of services in each family type, a lot of other service compositions could be done to answer Marcel's query. A large quantity of algorithms for this purpose have been developed, and all of them share the same problems: (1) producing rewritings when a big amount of services are available is extremely expensive; and (2) not always the quality of the rewriting (composition) produced is enough for meeting your needs.
Motivated by these problems, our approach proposes a new vision of data integration as follows.

Assuming the same medical scenario and the same family of data services.
Let us suppose Marcel would like to study the type of people suffering of a particular disease as before. However, in this new scenario, he is also capable to express his preferences while integrating services. For instance, he needs to query the patients' personal information and patients' dna information from patients that were infected by flu, using services with availability higher than 98\%, price per call less than 0.2\$ and total cost less than 2\$. Marcel has at his disposal a set of services from the families \textbf{A}, \textbf{B}, \textbf{C} and \textbf{D} geographically disposed on different cloud provides (configuring a multi-cloud environment). 
To achieve his needs Marcel can use the data services  as follows.....
