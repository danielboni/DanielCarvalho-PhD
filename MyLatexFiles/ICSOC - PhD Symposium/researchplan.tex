Recently, we have been working on a query rewriting algorithm called \textit{Rhone}. It serves as proof for the feasibility of our data integration process guided by cloud constraints and user preferences~\cite{carvalho2016}. The first results are promisingly: \textit{Rhone} reduces the rewriting number and processing time while considering user preferences and services' quality aspects extracted from SLAs to guide the service selection and rewriting. Furthermore, the integration quality is enhanced, and the integration total cost is reduced.

SLA incompatibilities have not been treated yet. Currently, we are working on this issue improving the SLA model and schema for data integration (i) to be adapted to the requirements and constraints imposed by the multi-cloud context; and (ii) to able to solve the different incompatibilities (schema, semantics and units, for instance) that can be found on the contracts in this context.  Other important research aspects are how to make efficient the rewriting process by reducing the composition search space, and how should be a parallel execution of the query plan to let the execution efficient in the multi-cloud. In addition, we have been working on evaluating and validating the entire quality-based data integration approach on a multi-cloud environment. 