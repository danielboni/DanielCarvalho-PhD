Current data integration systems imply consuming data from data services deployed in cloud contexts and integrating the results. The data is delivered under different quality conditions related to data freshness, cost, reliability, availability, among others. However, the lack of an associated meta-data describing the data, currently produced in huge quantities, makes the integration process more challenging.  

In cloud computing, the quality conditions under which services are delivered to users are agreed in contracts called service level agreements (SLA). We strongly believe that SLAs can be re-modeled in order (i) to cover the limitations regarding the users' integration requirements (including quality constraints and data requirements); and (ii) to enhance the quality and performance in the current data integration solutions. 
%
Works on SLA in cloud computing mainly concern (i) the \textit{negotiation} of contracts between customers and providers; and (ii) \textit{monitoring} of cloud resources to detect SLA violations.
%Yet, to the best of our knowledge, we have not find works proposing SLA-based approach for data integration in a multi-cloud environment. 
Yet, to the best of our knowledge, few works considers SLAs in order to guide and enhance the entire data integration in multi-cloud environments.
Similarly to our idea, \cite{Nie07} proposed a SLA-based data integration model for grid environments. The approach uses SLAs to define database resources and to evaluate them in terms of processing cost, amount of data and price for using the grid. However, we believe that other requirements and quality constraints should be considered in the SLA while processing integration tasks.
%In addition, a matching algorithm is proposed to produce query plans using the selected resources. 

%The most appropriated solutions based on these QoS are selected as final results. Our work differs from \cite{Nie07} in some aspects: 
%\begin{itemize}
%\item Data is delivered as \textit{data services} in a multi-cloud context. \textit{Data services} and \textit{cloud providers} export their SLA defining the quality conditions under which the service is delivered.
%\item SLAs are not limited only to describe the cost and amount of data, but also data quality aspects such its provenance, privacy, confidentiality, freshness, and service's delivery aspects such as response time, availability, reliability, among others.
%\item Users are able to express queries associating quality integration requirements to them. Then, the service selection and rewriting process in terms of service compositions are guided by the user's requirements and the SLAs exported by \textit{data services} and \textit{cloud providers}.
%\end{itemize}

In multi-cloud context, current SLA models are not sufficient to cover the data integration requirements. Usually, SLAs are related to cloud resources, and not to the services. Thus, new models should be defined to include users' integration preferences concerning data requirements and quality constraints. In summary, the objectives of this PhD project contribute as following: (1) design of SLA model for data integration; (2) proposal of a data integration approach adapted to the vision of the economic model of the cloud. The originality of our approach consists in guiding the entire data integration solution - while selecting and composing cloud services, and delivering the results - taking into account (i) user preferences statements; (ii) SLA contracts exported by different cloud providers; and (iii) several QoS measures associated to data collections properties (for instance, trust, privacy, economic cost); and (3) validation of our approach in a multi-cloud scenario.
