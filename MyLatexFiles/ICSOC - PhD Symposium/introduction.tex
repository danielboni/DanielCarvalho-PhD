Current data integration systems imply consuming and integrating data from data services which deliver data under different quality conditions related to freshness, trust, cost, reliability, availability, among others. Selecting data services, producing query rewritings and executing query plans are computationally expensive. These data integration tasks can take advantages from multi-cloud architectures considering their elasticity, \textit{pay-per-use} model and parallel processing.

Multi-cloud environments bring new challenges to data integration due to different entities (data provider, data consumer, infrastructure, and the data) that should be taken into account, and their associated constraints and characteristics. 
To better understand our problem, let us suppose the following scenario. During Brazilian Olympic games 2016, Lucas is an spectator willing to collect information about good restaurants near your actual location, and the best route and transport to go in the next volleyball match. To achieve his needs, there are several data provider services distributed on different clouds that can be integrated to produce an answer for his query. Thus, given a user query, the integration process deals with different matching problems: (i) matching the \textit{query} and \textit{data provider services} - the data provider services have to produce a result for the query; (ii) matching the \textit{user preferences} and the \textit{quality guarantees} provided by the data provider - the user preferences concern the data itself, the data services and the type of subscription the user has with the clouds; (iii) matching the \textit{user preferences} and \textit{user' type of subscriptions} - the user may have several subscriptions with the clouds, and perhaps some of them can be running out of budget for consuming cloud resources; and (iv) the \textit{data provider services} and \textit{their type of subscriptions} - the data provider services also have  subscription with the clouds, and they can also be out of budget and resources according to their subscription.
%Thus, given a user query, the integration process deals with different matching problems. They are: (i) matching the query and data services - the data services have to produce a result for the query; (ii) matching the user preferences and the quality guarantees provided by the data provider - The preferences concern the data itself, the data services and the type of subscription the user has with the clouds; (iii) matching the user preferences and type of subscriptions - the subscriptions the user has with the clouds; and (iv) the data providers and the type of subscriptions - These subscription concern the ones data providers have with the clouds.

In cloud computing, the quality conditions that the user can expect from a service are defined in contracts called service level agreements (SLA). Current SLA models are not sufficient to cover the data integration requirements. Usually, SLAs are related to cloud resources, and not to the services. Thus, we strongly believe that SLAs can be re-modeled in order (\textit{i}) to cover the limitations regarding the users' integration requirements (including quality constraints and data requirements); and (\textit{ii}) to enhance the quality and performance in the current data integration solutions. 
Considering the problems and challenges aforementioned, this PhD project contributes as following: (1) design of SLA model for data integration; (2) proposal of a data integration approach adapted to the vision of the economic model of the cloud. The originality of our approach consists in guiding the entire data integration solution - while selecting, filtering and composing cloud services, and delivering the results - taking into account (\textit{a}) user preferences statements; (\textit{b}) SLA contracts exported by different cloud providers; and (\textit{c}) several QoS measures associated to data collections properties (for instance, trust, privacy, economic cost); and (3) validation of our approach in a multi-cloud scenario.
