Current data integration systems imply consuming and integrating data from data services which deliver data under different quality conditions related to freshness, trust, cost, reliability, availability, among others. Selecting data services, producing query rewritings and executing query plans are computationally expensive requiring a huge amount of resources that can be probably unavailable or insufficient to perform them. These data integration tasks can take advantages from multi-cloud architectures considering their elasticity, \textit{pay-per-use} model and parallel processing.

Multi-cloud environments bring new challenges to data integration due to 4 principal dimensions, they are: \textit{data provider, data consumer, infrastructure}, and the \textit{data}. On the other hand, multi-cloud imposes at
real-time their associated constraints and characteristics such as access policies, access constraints,
processing capacity, memory limits, among others. 
%
To better understand the implied issues, let us suppose the following scenario. 
During Brazilian Olympic games 2016, Lucas is a spectator willing to collect information about the weather forecast near his  location with a time interval of two days in advance. According to the weather different matches could be proposed, in different stadiums with different options of seats to buy (such as sunny seats or not, in the middle or in the sides, and in the side of a specific team).
%
Lucas may also have several preferences such privacy issues (i.e. he wants his personal data to be anonymous), time interval, budget, using free services or not, for instance. 
%
To achieve his needs, there are several data provider services distributed on
different clouds that can be integrated to produce an answer for his query or
part of it. Lucas must be subscribed in the cloud to have access to its
information and the cloud service location is known. From this, he
is able to create his query.
%
Thus, given a user query, the integration process deals with different matching problems: 
(i) matching the \textit{query} and \textit{data provider services} - the data provider services should be able to produce a (complete or partial) result for the query; 
(ii) matching the \textit{user preferences} and the \textit{quality guarantees} provided by the data provider - the user preferences and requirements concern the data itself, the data services and the type of subscription the user has with the clouds; 
(iii) matching the \textit{user preferences} and \textit{user' type of subscriptions} - the user may have several subscriptions with different clouds that should influence the way to choose the services according to the cloud resources offered thank to user subscription; and 
(iv) the \textit{data provider services} and \textit{their type of subscriptions} - the data provider services also have  subscription with the clouds, and this imposes to adapt the way service is delivered according to the resources to which it has access.

The quality conditions that the user can expect from a service are defined in
contracts called service level agreements (SLA), in a multi-cloud environment.
%
Current SLA models are not sufficient to cover the data integration requirements. 
%
Usually, SLAs describe only cloud resources, and do not tackle the data integration aspects. 
%
Thus, we strongly believe that: 
(\textit{i}) a new kind of SLA is needed to express all information regarding the constraints and requirements as a meaning for the integration process; and 
(\textit{ii}) keeping the new SLA in integration history and its clever use while reusing could enhance the quality and performance in the current data integration solutions. 

Considering the problems and challenges aforementioned, this PhD project
contributes designing a SLA model and an approach for data integration adapted
to the cloud economic model. The originality of our approach consists in guiding
the entire data integration solution - while selecting, filtering and composing
cloud services, and delivering the results - taking into account (\textit{a})
user preferences statements; (\textit{b}) SLA contracts exported by different
cloud providers; and (\textit{c}) several QoS measures associated to data
collections properties (for instance, trust, privacy, economic cost).    

The reminder of this paper is organized as follows.
Section 2 discusses the related works.
Section 3 presents our SLA-based data integration approach
Section 4 describes the research plan, and finally, section 5 presents
our conclusions.
