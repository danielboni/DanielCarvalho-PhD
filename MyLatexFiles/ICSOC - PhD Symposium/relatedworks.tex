The related works concerning the problem stated in the previous section can be divided in three topics: (\textit{i}) data integration approaches in the cloud or in service-oriented contexts; (\textit{ii}) query rewriting approaches; and (\textit{iii}) service level agreements for cloud computing.

The authors in~\cite{Correndo2010,ElSheikh2013} perform data integration in service-oriented contexts, particularly considering data services. However, they only take into consideration the requirement of computing resources for integrating data focusing on performance aspects. \cite{YauY08} focused on data privacy in order to integrate data obtained from different data services. \cite{Tian2010} proposed an inter-cloud data integration system considering privacy requirements and the cost for protecting and processing data. Although \cite{Tian2010,YauY08} tackled in their approaches quality aspects of the integration, we believe there are other crucial elements that should be studied regarding the requirements and constraints of data consumers, data providers, the associated infrastructures and the data itself, and how to filter services and produce the best query plan considering these requirements and constraints.

As traditional databases theory, data integration on cloud and service-oriented context deals with query rewriting issues. Researches~\cite{ba2014,Barhamgi2010,Benouaret2011,Umberto} have refereed it as a service composition problem in which given a query the objective is to lookup and compose data services that can contribute to produce a result. In general, these works share the same performance problem depending on the size of the query and on the number of available services. Although \cite{ba2014,Benouaret2011} have considered preferences and scores to produce rewritings, the multi-cloud context brings new challenges once new requirements and constraints are introduced. Thus, new heuristics should be considered in the rewriting process in order to make it efficient. 

Service level agreements (SLA) have been widely adopted in different domains in order to specify what service consumer can expect from the service delivered by a service provider. Research contributions in cloud computing mainly concern (i) SLA negotiation phase (step in which the contracts are established between customers and providers) and (ii) monitoring and allocation of cloud resources to detect and avoid SLA violations. We strongly believe that SLAs could be used in order to cover the limitations discussed in the previous paragraphs and to enhance the quality in the current data integration solutions. In this sense, to the best of our knowledge, we have not identified other works that uses SLA to guide the entire data integration on a multi-cloud context.
