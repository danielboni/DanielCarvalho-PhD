Related works can be divided into three topics: (\textit{i}) data integration approaches in the cloud and in service-oriented contexts; (\textit{ii}) query rewriting approaches; and (\textit{iii}) service level agreements for cloud computing.

Correndo \textit{et al.} and ElSheikh \textit{et al.}~\cite{Correndo2010,ElSheikh2013} performed data integration in service-oriented contexts, particularly considering data services. However, they  consider computing resources consumption versus performance for guiding the data integration process. \cite{YauY08} addresses data privacy  to integrate data collected from different data services. \cite{Tian2010} proposed an inter-cloud data integration system considering privacy requirements and the cost for protecting and processing data. Even if \cite{Tian2010,YauY08} tackle quality aspects of the integration,  other crucial aspects  should be studied, for example data consumers requirements and constraints, data providers, the associated infrastructures and the data quality itself. It is also important to include these criteria in the way services are composed to produce  query plans.

As traditional databases theory, data integration on cloud and service-oriented context deals with query rewriting issues. Existing work like~\cite{ba2014,Barhamgi2010,Benouaret2011,Umberto} have refered it as a service composition problem. Given a query, the objective is to lookup and compose data services that can contribute to produce a result. In general, these works must address performance issues, because they use algorithms that can become expensive according to the complexity of the query and on the number of available services. Although \cite{ba2014,Benouaret2011} have considered preferences and scores to produce rewritings, the multi-cloud context introduces new requirements and constraints to the integration process. Currently, the approaches are not sufficient to cover the new challenges. Thus, they should be revisited and adapted in order to make the integration efficient in this new environment. 

Service level agreements (SLA) have been widely adopted in different domains to specify what service consumer can expect from the service delivered by a service provider. Research contributions in cloud computing concern (i) SLA management; (ii) inclusion of security requirements on SLAs; (iii) SLA negotiation; (iv) SLA matching; and (v) monitoring and allocation of cloud resources to detect and avoid SLA violations. 
%Research contributions in cloud computing mainly concern (i) SLA negotiation phase (step in which the contracts are established between customers and providers) and (ii) monitoring and allocation of cloud resources to detect and avoid SLA violations.
We strongly believe that SLAs can be used  to explicitely introduce the notion of quality in the current data integration solutions. In this sense, the use of SLA's to guide the entire data integration in a multi-cloud context seems original and promising for providing new perspectives to the data integration problem.
