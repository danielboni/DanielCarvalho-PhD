In recent years, the cloud have been the most popular deployment environment for data integration~\cite{Carvalho2015}.
Researches in this domain have focused manly on performance~\cite{Correndo2010,ElSheikh2013} and privacy issues~\cite{Tian2010,YauY08}. 

\cite{Correndo2010} proposed a query rewriting method for achieving RDF data integration. The objective of the approach is: (i) solve the entity co-reference problem which can lead to ineffective data integration; and (ii) exploit ontology alignments with a particular interest in data manipulation. \cite{ElSheikh2013} introduced a system (called SODIM) which combine data integration, service-oriented architecture and distributed processing. The novelty of these approaches is that they perform data integration in service oriented contexts, particularly considering data services. They also take into consideration the requirement of computing resources for integrating data. Thus, they exploit parallel settings for implementation costly data integration processes. However, they are focused on performance aspects putting aside users' requirements regarding the quality of the data being integrated, the type of data services she wants to use, the economic cost she is ready to pay, among others.

\cite{YauY08} focused on data privacy in order to integrate data. Based on users' integration requirements, the repository supports the retrieval and integration of data across different services. 
\cite{Tian2010} proposed an inter-cloud data integration system that considers a trade-off between users' privacy requirements and the cost for protecting and processing data. According to the users' privacy requirements, the query plan in the cloud repository creates the users' query. This query is subdivided into sub-queries that can
be executed in service providers or on a cloud repository. Each option has its own  privacy and processing costs.
Thus, the query plan executor decides the best location to execute the sub-query to meet privacy and cost constraints. Although these approaches have tackled quality aspects of the integration, we believe there are other crucial aspects that should be studied to meet users' integration preferences such as data provenance and integrity, confidentiality, services reliability and availability, among others. Moreover, even with the cloud on-demand resources provisioning (which implies an  associated cost), the user is limited to her cloud subscription and maximum budget she is ready to pay for her desired integration. The economic cost should be taken into consideration.

Service level agreements (SLA) have been widely adopted in different domains. Research contributions in cloud computing mainly concern (i) SLA negotiation phase (step in which the contracts are established between customers and providers) and (ii) monitoring and allocation of cloud resources to detect and avoid SLA violations. As highlighted in our previous work~\cite{Carvalho2015}, we strongly believe that SLAs could be used in order to cover the limitations and enhance the quality in the current data integration solutions. Close to our idea, \cite{Nie07} proposed a data integration model based on SLAs in grid environments. Their work uses SLAs to define database resources. Then, resources can be evaluated (in terms of processing cost, amount of data and price of using the grid) and selected to the integration. In addition, a matching algorithm is proposed to produce query plans. The most appropriated solutions based on these QoS are selected as final results. Our work differs from \cite{Nie07} in some aspects: 
\begin{itemize}
\item Data is delivered as \textit{data services} in a multi-cloud context. \textit{Data services} and \textit{cloud providers} export their SLA defining the quality conditions under which the service is delivered.
\item SLAs are not limited only to describe the cost and amount of data, but also data quality aspects such its provenance, privacy, confidentiality, freshness, and service's delivery aspects such as response time, availability, reliability, among others.
\item Users are able to express queries associating quality integration requirements to them. Then, the service selection and rewriting process in terms of service compositions are guided by the user's requirements and the SLAs exported by \textit{data services} and \textit{cloud providers}.
\end{itemize}

In this context, current SLA models are not sufficient to cover the data integration requirements and multi-cloud context. Thus, we are current working on new models to tackle these aspects. To the best of our knowledge, we have not identified any other approach which uses SLA to guide the data integration in a cloud and multi-cloud context.
