This paper introduced a new data integration solution, adapted to the multi-cloud context. The solution is described thank to a metamodel describing the implied entities and a meta-process presenting the activities and the corresponding challenges. 

%This paper introduced a new vision of data integration guided by quality expressed through SLA's that can be executed in multi-cloud settings. It presented a data integration metamodel and meta-process in order to illustrate the data integration challenges imposed by this context and the guidelines we follow to develop our research.  
%Moreover, our research plan to achieve a better profit and efficiency in the instant data integration solutions is discussed.

%In addition, a query rewriting algorithm called \textit{Rhone} serves as proof for the feasibility of data integration process guided by cloud constraints and user preferences . Our first results are promisingly. The \textit{Rhone} reduces the rewriting number and processing time while considering user preferences and services' quality aspects extracted from SLAs to guide the service selection and rewriting. Furthermore, the integration quality is enhanced, and it is adapted to cloud economic model reducing the total cost of the integration.

%SLA incompatibilities are not treated in this paper. Currently we are working on this issue, and improving our SLA model and schema for data integration adapted to the multi-cloud context.  Another important part is how to make efficient the rewriting process by reducing the composition search space. Finally, how should be a parallel execution of the query plan to let the execution efficient in the multi-cloud. In addition, we are focusing on evaluating and validating the entire quality-based data integration approach on a multi-cloud environment. 