Data integration has evolved with the emergence of data services that deliver
data under different quality conditions related to data freshness, cost, reliability,
availability, among others. Data are produced continuously and on demand in huge
quantities and sometimes with few associated meta-data, which makes the
integration process more challenging. Some approaches express data integration
as a service composition problem where given a query the objective is to lookup and compose data services that can contribute to produce a result. Finding the best service composition that can answer a query can be computationally costly. Furthermore,  executing the composition can lead to retrieve and process data collections that can require important memory, storage and computing resources.
This problem has been addressed in the service-oriented
%domain~\cite{Benouaret2011,ba2014}.
domain~\cite{Barhamgi2010,Benouaret2011,ba2014}.
Generally, these solutions deal with query rewriting problems.
%\cite{Benouaret2011} introduced a service composition
%framework to answer preference queries based on a previous a query rewriting approach which processes queries
%on data provider services. 
%In that approach, two algorithms are presented to rank the best rewritings based on previously computed scores.
\cite{Barhamgi2010} proposed a query rewriting approach which processes queries
on data provider services. \cite{Benouaret2011} introduced a service composition
framework to answer preference queries. In that approach, two algorithms based
on~\cite{Barhamgi2010} are presented to rank the best rewritings based on previously computed scores.
%: one (i) to produce all possible rewritings before computing their scores and other (ii)
%which uses a quality metric that combines diversity and accuracy to,
%incrementally, rank services and to build the best rewritings.
\cite{ba2014} presented an algorithm that produces and order rewritings
according to user preferences. Yet, to our knowledge few works consider quality
measures associated both to data services and to user preferences in order to
guide the rewriting process. 
%In the context of the cloud, and also considering
%that queries can be executed in conditions in which energy, data transfer
%economic cost, memory and computing resources consumption can be limited or
%controlled by some economic model, it is important to design approaches and
%rewriting algorithms that consider these constraints.       


This paper introduces the early stages of our
ongoing work on developing the \textit{Rhone} service-based query rewriting
algorithm guided by SLA's. Our work addresses this issue and proposes the 
algorithm \textit{Rhone} with two original aspects: (i) the user can express
her quality preferences and associate them to her queries; and (ii)  service's
quality aspects defined on Service Level Agreements (SLA) guide service
selection and  the whole rewriting process.

 The remainder of this paper is organized as follows. Section~\ref{sec:rhone}
 describes the algorithm \textit{Rhone}, proposed in our work.
 Section~\ref{sec:implementationandresults} describes a running scenario and also
 implementation issues.
 Finally, section~\ref{sec:conclusions} concludes the paper and discusses our work perspectives.

%Section~\ref{sec:rhone} describes the algorithm \textit{Rhone}, proposed in our
%work. Section~\ref{sec:implementationandresults} describes a running scenario
%and also implementation issues.
%Finally, section~\ref{sec:conclusions} concludes the paper.