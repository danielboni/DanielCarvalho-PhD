\documentclass[12pt,a4paper,oneside]{article}
\usepackage[utf8]{inputenc}
\usepackage[english]{babel}
\usepackage{amsmath}
\usepackage{amsfonts}
\usepackage{amssymb}
\usepackage{graphicx}
\usepackage[left=2cm,right=2cm,top=2cm,bottom=2cm]{geometry}
\usepackage{titlesec}
\author{Daniel Aguiar da Silva Carvalho}
%\title{State of The Art}

\begin{document}

\section{State of the art}

\subsection{Related work on data integration and cloud services}

In \cite{Gonzalez:2010b}, a cloud-based data management and integration system called Fusion Tables is presented. It enables data sharing, integration and collaboration between different and multiple users. Users, that can be non-IT experts, can visualize and manipulate their data in the web in a easy way. The system enables users to (i) uploading of data files from different formats; (ii) visualizing the data in different ways; (iii) integrating data from different sources belonging to multiple users. The integration process consists in a join between tables; (iv) sharing and controlling data in at levels; and (v) interacting with data in a web interface or through an API. The authors described the design foundations of Fusion Tables (such as integration with the web, easy of use, incentives for sharing and facilitate collaboration) and some examples of applications that can take advantages from the system. 

The summaries for data integration are the ones I used in the first version of the related works on our paper~\cite{075,078,Nie07,096,Yau08}. I have no summaries of the data integration approaches I read before (I recently started doing the summaries), but this is the complete list I have read~\cite{066,067,070,072,113,077,Dustdar:2012,081,110,111,094,099,102}. Considering these papers, most of them are frameworks/systems for data integration. Excepting the articles concerning the Google fusion tables~\cite{Gonzalez:2010}, none of the works until now presented (clearly) how they integrate data (there is only a superficial description of the approach). The works focus on data quality aspects such as cost, privacy, protection and security of their integration approach. In~\cite{Lenzerini:2002}, a theoretical perspective of data integration is presented, focusing on aspects such as modeling data integration applications, inconsistencies between sources, reasoning on queries and query rewriting.

\bibliographystyle{plain}
\bibliography{bibliography}

\end{document}